\documentclass{report}

%====================== PACKAGES ======================

% region: packages

%\usepackage{extsize}
\usepackage[english]{babel}
\usepackage[utf8x]{inputenc}
%pour gérer les positionnement d'images
\usepackage{float}
\usepackage{amsmath}
\usepackage{graphicx}
\usepackage[colorinlistoftodos]{todonotes}
\usepackage{url}
%pour les informations sur un document compilé en PDF et les liens externes / internes
% \usepackage[hidelinks,colorlinks,linkcolor=black,]{hyperref}
\usepackage[hidelinks]{hyperref}
%pour la mise en page des tableaux
\usepackage{array}
\usepackage{tabularx}
%pour utiliser \floatbarrier
%\usepackage{placeins}
%\usepackage{floatrow}
%espacement entre les lignes
\usepackage{setspace}
%modifier la mise en page de l'abstract
\usepackage{abstract}
%police et mise en page (marges) du document
\usepackage[T1]{fontenc}
\usepackage[top=2.5cm, bottom=2.5cm, left=2.5cm, right=2.5cm]{geometry}
%Pour les galerie d'images
% \usepackage{subfig}

% endregion: packages

%====================== PACKAGES PERSO ======================

% region: packages perso

\usepackage{lmodern} % autre police d'écriture qui supporte le scaling
\usepackage[noabbrev]{cleveref} % références croisées [noabbrev] pour éviter les abréviations
\usepackage{acro} % acronymes (voir ici : https://tex.stackexchange.com/questions/492175/how-to-generate-list-of-abbreviations-in-latex)
\usepackage{mathrsfs}  % pour les polices mathématiques ex: \mathscr{X}
\usepackage{dsfont,amsfonts} % pour les polices mathématiques ex: \mathds{X}
% \usepackage{subfig} % pour les sous figures
\usepackage{lipsum} % pour générer du texte aléatoire
\usepackage{booktabs} % pour les tableaux
\usepackage{multirow} % pour les tableaux
\usepackage{etoc} % pour les tables des matières par chapitres 
\usepackage{algorithm} % pour les algorithmes
\usepackage{algorithmic} % pour les algorithmes
\usepackage[justification=justified,format=plain,labelfont=bf,width=0.9\textwidth]{caption}
\usepackage{bibentry} % Pour ajouter des refs qui ne sont pas citée directement
\usepackage{bm} % pour les polices mathématiques ex: \mathbf{X}
\usepackage{amssymb} % symboles mathématiques suplémentaires (comme \varnothing)
\usepackage{amsthm} % pour les théorèmes et surtout les preuves (voir ici: https://stackoverflow.com/questions/1449370/latex-error-environment-proof-undefined)
\usepackage[square,numbers]{natbib} % pour citer les auteurs directement [DEUX REFS] (1):https://tex.stackexchange.com/questions/69379/how-do-i-cite-author-in-latex (2) https://www.overleaf.com/learn/latex/Bibliography_management_with_natbib
\usepackage[left,modulo]{lineno} % pour numéroter les lignes
\usepackage{etoolbox} % for the \patchcmd command
\usepackage{subcaption}% pour avoir des sous tables et les sous figures (remplace subfig)
\usepackage{xcolor} % to use colors, specially for acronyms 
\usepackage{sectsty} % pour le style des sections (voir ici : https://tex.stackexchange.com/questions/311487/how-to-change-the-title-abstract-and-headings-font-to-sans-serif)
\usepackage{bbold} % pour les 1 avec 2 barres
\usepackage{stmaryrd} % pour les crochets d'intervalles
% \usepackage{libertine}  
% endregion: packages perso




%DIF LATEXDIFF DIFFERENCE FILE
%DIF DEL old.tex   Sat Jun 20 10:36:29 2015
%DIF ADD new.tex   Sat Jun 20 10:36:36 2015
%DIF PREAMBLE EXTENSION ADDED BY LATEXDIFF
%DIF UNDERLINE PREAMBLE %DIF PREAMBLE
\RequirePackage[normalem]{ulem} %DIF PREAMBLE
\RequirePackage{color}\definecolor{RED}{rgb}{1,0,0}\definecolor{BLUE}{rgb}{0,0,1} %DIF PREAMBLE
\providecommand{\DIFadd}[1]{{\protect\color{blue}\uwave{#1}}} %DIF PREAMBLE
\providecommand{\DIFdel}[1]{{\protect\color{red}\sout{#1}}}                      %DIF PREAMBLE
%DIF SAFE PREAMBLE %DIF PREAMBLE
\providecommand{\DIFaddbegin}{} %DIF PREAMBLE
\providecommand{\DIFaddend}{} %DIF PREAMBLE
\providecommand{\DIFdelbegin}{} %DIF PREAMBLE
\providecommand{\DIFdelend}{} %DIF PREAMBLE
%DIF FLOATSAFE PREAMBLE %DIF PREAMBLE
\providecommand{\DIFaddFL}[1]{\DIFadd{#1}} %DIF PREAMBLE
\providecommand{\DIFdelFL}[1]{\DIFdel{#1}} %DIF PREAMBLE
\providecommand{\DIFaddbeginFL}{} %DIF PREAMBLE
\providecommand{\DIFaddendFL}{} %DIF PREAMBLE
\providecommand{\DIFdelbeginFL}{} %DIF PREAMBLE
\providecommand{\DIFdelendFL}{} %DIF PREAMBLE
%DIF END PREAMBLE EXTENSION ADDED BY LATEXDIFF




%====================== COMMANDES PERSO ======================
\newcommand\myfunc[5]{%
  \begingroup
  \setlength\arraycolsep{0pt}
  #1\colon\begin{array}[t]{c >{{}}c<{{}} c}
             #2 & \to & #3 \\ #4 & \mapsto & #5 
          \end{array}%
  \endgroup}

% Pour pouvoir créer des propotision (voir ici : https://www.overleaf.com/learn/latex/Theorems_and_proofs)
\newtheorem{proposition}{Proposition}[section]
\newtheorem*{remark}{Remark} % spécifique pour les remarques


% Patch abstract pour ne pas utiliser de "titlepage" et ainsi avoir une
% numérotation de page qui ne se remet pas à 1 à chaque chapitre.
% Voir ici : https://tex.stackexchange.com/a/483381
\patchcmd{\abstract}{\titlepage}{\cleardoublepage}{}{}
\patchcmd{\endabstract}{\endtitlepage}{\clearpage}{}{}
%====================== INFORMATION ET REGLES ======================

%rajouter les numérotation pour les \paragraphe et \subparagraphe
\setcounter{secnumdepth}{4}
\setcounter{tocdepth}{4}

% Réglage propres au document PDF

\hypersetup{							% Information sur le document
pdfauthor = {Robin Dupont},			% Auteurs
pdftitle = { Thesis Manuscript of Robin Dupont},			% Titre du document
% pdfsubject = {Mémoire de Projet},		% Sujet
% pdfkeywords = {Tag1, Tag2, Tag3, ...},	% Mots-clefs
pdfstartview={FitH}}					% ajuste la page à la largueur de l'écran
%pdfcreator = {MikTeX},% Logiciel qui a crée le document
%pdfproducer = {}} % Société avec produit le logiciel


%====================== SECTION PERSO ====================== 
% acronym style

\definecolor{mycolor}{HTML}{363687} % define your color using a hex code


\acsetup{
    make-links = true,
}

\RenewAcroTemplate{long-short}{%
  \acroiffirstTF{%
    \acrowrite{long}%
    \acspace(%
    \acroifT{foreign}{\acrowrite{foreign}, }%
    \textcolor{mycolor}{\acrowrite{short}}% <-- Change is here
    \acroifT{alt}{ \acrotranslate{or} \acrowrite{alt}}%
    \acrogroupcite
    )%
  }%
  {\acrowrite{short}}%
}

\RenewAcroTemplate{short}{%
\textcolor{mycolor}{\acrowrite{short}}%
}


% acronyms

\DeclareAcronym{BN}{
    short=BN,
    long=Batch Normalisation,
}

\DeclareAcronym{CNN}{
    short=CNN,
    long=Convolutional Neural Network,
    long-plural=s,
}

\DeclareAcronym{GPU}{
    short=GPU,
    long=Graphics Processing Unit,
    long-plural=s,
}

\DeclareAcronym{TPU}{
    short=TPU,
    long=Tensor Processing Unit,
    long-plural=s,
}

\DeclareAcronym{CPU}{
    short=CPU,
    long=Central Processing Unit,
    long-plural=s,
}

\DeclareAcronym{amc}{
    short=AMC,
    long=AutoML for Model Compression,
}

\DeclareAcronym{nan}{
    short=\texttt{NaN},
    long=Not a Number,
}

\DeclareAcronym{FLOP}{
    short=FLOP,
    long=Floating Point Operation,
    long-plural=s,
}

\DeclareAcronym{ReLU}{
    short=ReLU,
    long=Rectified Linear Unit,
    long-plural=s,
}

\DeclareAcronym{STE}{
    short=STE,
    long=Straight Through Estimator,
}

\DeclareAcronym{LTH}{
    short=LTH,
    long=Lottery Ticket Hypothesis,
}

\DeclareAcronym{ASLP}{
    short=ASLP,
    long=Arbitrarily Shifted Log Parametrisation,
}

\DeclareAcronym{SGD}{
    short=SGD,
    long=Stochastic Gradient Descent,
}
\DeclareAcronym{LT}{
    short=LT,
    long=Lottery Ticket,
    long-plural=s,
}

\DeclareAcronym{GS}{
    short=GS,
    long=Gumbel-Softmax,
}

\DeclareAcronym{STGS}{
    short=STGS,
    long=Straight Through Gumbel-Softmax,
}

\DeclareAcronym{DWR}{
    short=DWR,
    long=Dynamic Weight Rescaling,
}

\DeclareAcronym{SR}{
    short=SR,
    long=Smart Rescale,
}

\DeclareAcronym{WR}{
    short=WR,
    long=Weight Rescaling,
}

\DeclareAcronym{SC}{
    short=SC,
    long=Signed Constant,
}

\DeclareAcronym{FS}{
    short=FS,
    long=Fan Scaling,
}

\DeclareAcronym{NAS}{
    short=NAS,
    long=Neural Architecture Search,
}

\DeclareAcronym{KD}{
    short=KD,
    long=Knowledge Distillation,
}

\DeclareAcronym{MAC}{
    short=MAC,
    long=Multiply-Accumulate,
}

\DeclareAcronym{SE}{
    short=SE,
    long=Squeeze-and-Excitation,
}

\DeclareAcronym{FFT}{
    short=FFT,
    long=Fast Fourier Transform,
}

\DeclareAcronym{FPGA}{
    short=FPGA,
    long=Field Programmable Gate Array,
}

\DeclareAcronym{ANN}{
    short=ANN,
    long=Artificial Neural Network,
    long-plural=s,
}

\DeclareAcronym{DNN}{
    short=DNN,
    long=Deep Neural Network,
    long-plural=s,
}

\DeclareAcronym{MLP}{
    short=MLP,
    long=Multilayer Perceptron,
    long-plural=s,
}

\DeclareAcronym{FP32}{
    short=FP32,
    long=single-precision floating-point format,
}

\DeclareAcronym{PTQ}{
    short=PTQ,
    long=Post-Training Quantisation,
}

\DeclareAcronym{QAT}{
    short=QAT,
    long=Quantisation-Aware Training,
}

\DeclareAcronym{TA}{
    short=TA,
    long=Teacher Assistant,
    long-plural=s,
}

\DeclareAcronym{SKD}{
    short=SKD,
    long=Scaled Kaiming distribution
}

\DeclareAcronym{DAG}{
    short=DAG,
    long=Directed Acyclic Graph,
    long-plural=s,
}

\DeclareAcronym{FC}{
    short=FC,
    long=Fully Connected,
}

\DeclareAcronym{CL}{
    short=Conv,
    long=Convolutional,
    long-plural=s,
}

\DeclareAcronym{OBS}{
    short=OBS,
    long=Optimal Brain Surgeon,
}

\DeclareAcronym{OBD}{
    short=OBD,
    long=Optimal Brain Damage,
}
% ----- Hyphenation
\hyphenation{re-para-me-tri-za-tion}
\hyphenation{re-para-me-tri-sa-tion}
% ----- Style des liens
% \hypersetup{
%   citecolor=blue,
% }

% ################  A CHANGER POUR REVUE ?
\linespread{1} % régler l'espacement entre les lignes

\renewcommand\linenumberfont{\normalfont\large} % régler la taille des numéros de ligne

%espacement entre les lignes d'un tableau


%======================== DEBUT DU DOCUMENT ========================

\begin{document}
\pagenumbering{roman} % numérotation des pages en chiffres romains

% TODO: adapter pour le confort de lecture
\fontsize{14}{16}\selectfont

% style des sections : fonte sans serifs
\allsectionsfont{\sffamily}

%page de garde
\title{\vspace{-3.0cm}Manuscript}
\author{Robin Dupont}
\date{}

%page blanche
%\newpage

%ne pas numéroter cette page
%\thispagestyle{empty}
%\newpage

% \maketitle % Pas besoin de maketitle si on gère le tout dans une page à part.


% abstract

\chapter*{Abstract}
Thanks to the miniaturisation of electronics, embedded devices have become
more and more ubiquitous, since the 2010s, realising various tasks all
around us. As their usage is developing, there is a growing demand for these
devices to process data and make complex decisions efficiently. Deep neural
networks are powerful tools to achieve this goal, however, these networks
are often too heavy and complex to fit on embedded devices. Thus, there is a
compelling need to devise methods to compress these large networks without
significantly compromising their efficacy. This PhD thesis introduces two
innovative methods, centred around the concept of pruning, aiming to
compress neural networks while ensuring minimal impact on their accuracy.

This PhD thesis first introduces a budget-aware method for compressing large
neural networks with weight reparametrisation and budget loss that does not
require fine-tuning. Traditional pruning methods often rely on post-training
saliency indicators to remove weights, disregarding the targeted pruning
rate. Our approach integrates a budget loss, driving the pruning process
towards a specific value during training, thereby achieving a joint
optimisation of topology and weights. By soft-pruning the smallest weights
using weight reparametrisation, our method significantly mitigates accuracy
degradation in comparison to traditional pruning techniques. We show the
effectiveness of our approach across various datasets and architectures.

This PhD thesis later focuses on the extraction of effective subnetworks
without weight training. Our goal is to identify the best subnetwork
topology in a large network without optimising its weights while still
delivering compelling performance. This is achieved using our novel
Arbitrarily Shifted Log Parametrisation, which serves as a differentiable
relaxation of discrete topology sampling, enabling the training of masks
that represent the probability of selection of the weights. Alongside, a
weight rescaling mechanism (referred to as Smart Rescale) is also
introduced, which allows enhancing the performance of the extracted
subnetworks as well as speeding up their training. Our proposed approach
also finds the optimal pruning rate after one training pass, thereby
circumventing computationally expensive gird-search and training across
various pruning rates. As shown through comprehensive experiments, our
method consistently outperforms closely related state-of-the-art techniques
and allows designing lightweight networks which can reach high sparsity
levels without significant loss in accuracy.


\addcontentsline{toc}{chapter}{Abstract} 
% ----- Tables des matières, figures et tableaux  + acronymes ------
\tableofcontents
\listoffigures
\addcontentsline{toc}{chapter}{List of Figures}
\listoftables
\addcontentsline{toc}{chapter}{List of Tables}
\newpage
\printacronyms
\addcontentsline{toc}{chapter}{List of Acronyms}
\newpage
\chapter*{Remerciements}

\indent La thèse est un défi aussi bien scientifique qu'humain et qui ne peut
être relevé sans l'aide de nombreuses personnes qui m'ont apporté leur temps, leur idées,
leur conseils et leur soutien. Je souhaite ici les remercier.\\

Je tiens tout d'abord à remercier mon directeur de thèse Hichem Sahbi pour son
encadrement durant ces quelques années.\\

Je remercie Jenny Benois-Pineau et Titus Bogdan Zaharia pour avoir accepté
d'être rapporteurs de cette thèse et pour le temps qu'ils ont consacré à la
lecture du manuscrit. Je remercie également Pierre Beauseroy, Nicolas Gac et
Vincent Gripon pour avoir accepté d'être membres du jury.\\

Cette thèse est une thèse CIFRE, menée en partenariat avec Netatmo où j'ai pu
rencontrer et travailler avec de brillants collègues. Je remercie tout
particulièrement Alice Lebois et Mohammed-Amine Alaoui qui ont tous les deux été de grands soutiens 
\addcontentsline{toc}{chapter}{Remerciements}
\newpage
\thispagestyle{empty}
%recommencer la numérotation des pages à "1"

%====================== INCLUSION DES PARTIES ======================
\renewcommand{\arraystretch}{1.5}
\setcounter{page}{1}
\pagenumbering{arabic} % numérotation des pages en chiffres normaux
% \linenumbers
\chapter{Introduction}\label{chap:intro}

% - révolutions industrielles de la première à la 4ème avec pour objectifs la mecanisation du travail dans la première (détailler les 3 autres)

% - composante de la 4ème révolution industrielle : 
%     - l'intelligence artificielle permise par le développement des capacités de calcul
%     - naissance d'objects intelligents et doté de capacités de calcul embarqués (smartphones, voitures autonomes, satellites, etc...)

% - la taille des modèles est un frein au déploiement de l'intelligence artificielle sur des objets embarqués.
% - CEPENDANT: la 4ème révolution industrielle est aussi celle des objets communiquants donc on pourrait déporter les calculs sur des serveurs distants.
% - MAIS: il existe des raisons de vouloir faire du calcul embarqué:
%     - confidentialité des données
%     - distribuer la charge de calcul et ne pas assumer le coût, ni de la communication, ni du calcul
%     - plus grand réactivité (temps de latence)
%     - coût de la communication
%     - pas necessairement d'accès à internet (autonomie: ex: rover sur Mars)

From the spinning jenny, blast furnace and steam engine that sparked the first
industrial revolution to the \ac{IOT} devices that drives the fourth, the
objective of mechanising labour and optimising productivity has been a
persistent theme throughout the past centuries. The first industrial revolution,
which dates back to 1760, introduced mechanisation through the use of water
wheels and steam engines. The second industrial revolution, starting towards the
end of the \textsc{XIX}th century, is linked to the development of automobiles,
crude oil extraction and assembly lines powered by electric energy. The third
industrial revolution, also called the digital revolution, took place in the
second half of the \textsc{XX}th century and brought electronics, information
and communication technology, and automated production. The Fourth Industrial
Revolution, often known as Industry 4.0, inaugurates the digital integration of
production chains as well as smart and connected devices that lead to more
efficient manufacturing systems. The fourth industrial revolution focuses on the
interconnectivity of devices and the development of their computational
capabilities. This track leads to the emergence of ever-connected \ac{IOT}
devices with embedded computing facilities, such as smartphones, autonomous
vehicles or satellites, that leverage \ac{AI} algorithms.\\

In parallel with these industrial revolutions, the field of \ac{AI} has seen
substantial growth and development. The term \emph{\acl{AI}} was first used at
the Dartmouth workshop in 1956 which is considered to be the founding event of
\ac{AI} as a research field \cite{dartmouth1956}. It launched decades of
research into machine learning and natural language processing among others
\cite{nilsson1998artificial}. In the subsequent decades, \ac{AI} saw significant
strides, including the development of rule-based systems, called expert systems
\cite{giarratano1994expert}, in the 1970s and the early exploration of machine
learning in the 1980s \cite{rumelhart1986learning}. These advancements occurred
alongside the third industrial revolution, setting the stage for further
progress in \ac{AI}. In the late XXth and early XXIst centuries, coinciding with
the premises of the fourth industrial revolution and helped with substantial
progress in computational power of \acp{GPU}, \ac{AI} started to draw tremendous
attention from both researchers and industrials with the advent of Deep
Learning. The latter is a subfield of machine learning which uses multi-layer
\ac{ANN} to learn and model complex patterns in datasets in an end-to-end
fashion, bringing significant improvement over manually engineered data
representation. The fast development of Deep learning has been driving advancements
in various domains such as natural language processing
\cite{DBLP:conf/emnlp/BudzianowskiV19,DBLP:conf/naacl/DevlinCLT19,DBLP:conf/nips/VaswaniSPUJGKP17},
image and speech recognition
\cite{DBLP:conf/nips/KrizhevskySH12,DBLP:journals/corr/SimonyanZ14a,DBLP:conf/cvpr/HeZRS16,DBLP:journals/corr/HannunCCCDEPSSCN14,DBLP:conf/icassp/ChanJLV16,DBLP:conf/icml/AmodeiABCCCCCCD16},
text and image generation
\cite{goodfellow2020generative,karras2019style,DBLP:conf/emnlp/BudzianowskiV19},
video game playing \cite{silver2016mastering,silver2018general} and molecule
folding \cite{jumper2021highly} to name a few.\\


The conquest of new fields and the quest for performance improvement of Deep
Learning models have led to a significant increase in their computational
complexity and size (see \cref{fig:sota:net_sizes_std_eff_nas}), particularly
regarding their number of parameters. The sheer size of modern \acp{ANN}, called
\acp{DNN}, presents a significant barrier to their deployment on embedded
devices or \ac{IOT} devices whose memory and computational resources are
inherently limited. To circumnavigate this hurdle, the prevalent approach is to
offload computations onto remote servers, leveraging the ever-interconnected
nature of modern \ac{IOT} devices and appliances.\\

Nonetheless, several compelling reasons exist for conducting embedded
computations instead of moving them to the cloud. First, processing the data
locally on premises ensures better data privacy, since the latter does not need
to leave the device to be processed on the cloud. Indeed cloud instances can be
located on various continents or countries where the legislation about data
privacy might be different from the one of the countries where the data is
collected. Second, local computations can distribute the processing and limit
communications. This is particularly relevant in more ways than one: first, it
can reduce the cost of communication and bandwidth, which are typically billed
to companies by cloud providers. Second, in some scenarios, the device might not
have access to a large bandwidth or cannot afford to transmit a lot of data,
which can be the case for remote areas or some devices with a low power budget.
Third, local computations can lead to greater responsiveness by reducing
latency, which might be critical in some applications such as autonomous
vehicles. Fourth, local computations can enable autonomy, which is particularly
relevant for devices that cannot rely on internet access, such as Mars rovers,
submarine drones or any other devices that need to process data in radio
silence.\\

The fourth industrial revolution and the rapid evolution in the field of \ac{AI}
have opened up a myriad of applications, with \ac{AI} algorithms and in
particular \acp{DNN}, offering significant potential to enhance the capabilities
of \ac{IOT} devices. However, the deployment of these advanced \acp{DNN} on
\ac{IOT} devices presents a significant challenge due to the inherent
computational and memory constraints of such devices. The sheer size and
complexity of modern \acp{DNN}, which have been instrumental in their success,
become a barrier when considering on-device deployment. This presents a
compelling case for the development of lightweight neural networks, tailored for
\ac{IOT} devices, that maintain the power of their larger counterparts while
being significantly reduced in size and computational requirements. Such
lightweight neural networks can also benefit all areas where saving
computational resources is of interest. Consequently, there is a need for
dedicated research efforts to design methods that yield lightweight neural
networks. This thesis aims to contribute to this effort by introducing pruning
methods that can reduce the size of neural networks while preserving their
performances, with a focus on topology selection. We introduce two new pruning
methods: The first performs joint topology and weight optimisation allowing for
a minimal loss in performance after pruning compared to standard methods. The
second approach does not require any weight training and instead focuses on
stochastic yet differentiable topology selection, achieving compelling results
overall and outperforming other related state-of-the-art methods that, again, do
not train the weights.

\section{Industrial Context}

% - Thèse CIFRE Netamo 
% - Netatmo : entreprise française spécialisée dans les objets connectés
% - en particulier les cameras de sécurité pour les particuliers qui font de la reconnaissance de visages et de la detection d'objets
% - le but est de faire du calcul embarqué sur les cameras pour éviter de devoir envoyer les données sur des serveurs distants
% - C'est utile pour les raisons évoquées à la section précédente, et en particulier pour la confidentialité des données etle fait de ne pas avoir à faire payer d'abonnement aux utilisateurs

This research work is a CIFRE thesis with Netatmo, a French company specialising
in smart devices that is now part of the Legrand Group. Notably, Netatmo
commercialises security cameras for individual use that perform tasks such as
face recognition and object detection using \acp{DNN}. The objective is to run
the \acp{DNN} directly on these cameras, sidestepping the need to send data to
distant servers. This approach aligns well with the reasons outlined in the
previous section, particularly in ensuring data privacy. Moreover, it allows for
a subscription-free business benefiting the end user, since there is no need to
pay for cloud infrastructures dedicated to running \acp{DNN}. Therefore, Netatmo
needs to develop lightweight neural networks that can be run on embedded devices
while maintaining the performance of their larger and more complex counterparts.
The models should be lightweight in order to, on the one hand, run on limited
hardware, and on the other hand, be fast enough to perform, for instance,
real-time intruder detection and alerting.\\

\section{Why Deep learning ?}

Deep learning is a subfield of machine learning that is the subject of intense
research efforts and numerous publications. It employs \aclp{ANN}, called
\acfp{DNN}, that aim to learn and model complex patterns in unstructured data in
an end-to-end fashion. Deep learning models have proven their effectiveness in
numerous domains and have been particularly performant in the field of computer
vision
\cite{DBLP:conf/cvpr/HeZRS16,DBLP:conf/nips/RenHGS15,DBLP:conf/eccv/LiuAESRFB16}.
Computer vision, which lies at the heart of Netatmo smart camera
functionalities, encompasses algorithms that enable computers to interpret and
understand the visual world and in particular detect and classify objects.\\

\acp{DNN} are the backbone of most advanced computer vision applications,
including Netatmo facial recognition and object detection features. More
specifically, \acp{CNN}, a specific type of \acp{DNN} can process images
directly, reducing the need for manual feature extraction, and their capacity
for hierarchical feature learning makes them particularly effective for tasks
such as object recognition and classification. Their architecture is such that
they perform well at recognising patterns in unstructured data and are able to
learn gradually more complex and abstract concept representations from raw data,
enabling them to outperform other machine learning models and humans in computer
vision tasks (see \cref{fig:intro:models_vs_humans}).\\

\begin{figure}[htbp]
      \centering
      \includegraphics[width=0.8\textwidth]{chapter_intro/assets/models_vs_human.pdf}
      \caption{Models top-5 accuracy on ImageNet \cite{deng2009imagenet} compared
            to human performance.}
      \label{fig:intro:models_vs_humans}
\end{figure}

Given the nature of tasks the Netatmo cameras are designed to perform, deep
learning and \aclp{DNN} are not just a choice but a necessity. They represent
the state of the art in computer vision tasks that outperforms other algorithms
and allows for accurate and reliable object detection and recognition.\\

\section{Challenges}

While deep learning, particularly through the use of \acp{CNN}, is the
technology of choice for computer vision applications, it comes with its
challenges that need to be addressed, especially in the context of deploying
these deep and large models on embedded devices. These challenges include model
complexity and computational requirements. The necessity of compressing neural
networks has been highlighted previously and also comes with its challenges that
include: preserving the performance and controlling the size of the compressed
model as well as training time.\\

One of the most significant challenges in deploying deep learning models and
especially \acp{CNN} on embedded devices is the large model size. These models
often have millions of parameters and this makes them computationally heavy and
challenging to fit into the limited memory of embedded devices. Secondly, these
complex models require substantial computational resources to operate. This
translates into slow computations which is a critical issue for devices which
aim to perform real-time tasks.\\

Compressing large neural networks is a necessity to deploy them on embedded
devices. However, this compression process comes with its challenges. First, the
compressed model should maintain the performance of the original model. However,
the original large model is trained with all its parameters and thus depends on
all of them. Consequently, removing more than a few can lead to degraded
performance. \\

Second, the compressed model should be small enough to fit into the
limited memory of embedded devices. It means that the compression process should
be controlled to ensure that the size of the compressed model does not exceed
the memory budget. However, compressing the model too much can lead to an
irrecoverable loss in performance. The compression procedure and hyperparameters
should be carefully chosen to ensure that the compressed model has enough
capacity to perform the task at hand. This is often achieved by grid-searching
the optimal set of hyperparameters, which can be time-consuming.\\

Third, the compression process should be fast enough to be practical. Indeed,
the compression process is often performed after the training of the original
model and often requires fine-tuning the compressed one to compensate for the
loss of performance. This fine-tuning step can be computationally expensive and
time-consuming, effectively doubling the training time of the model in some
scenarios.\\

To conclude, while deep learning and \acp{CNN} represent an exciting advancement
in computer vision applications, several challenges need to be addressed for
efficient and effective deployment on embedded devices. Addressing these
challenges forms the crux of this research, with a particular focus on model
compression techniques to reduce the size and complexity of neural networks
without significant loss in performance.\\


\section{Contributions}

This thesis tackles the challenge of compressing \acp{DNN} through pruning, a
technique that aims to reduce the size of a neural network by removing redundant
or unnecessary parameters, subsequently detailed in \cref{chap:sota}. The
contributions detailed in this manuscript focus on methods to identify the
parameters to prune as well as minimise the impact of their removal on the final
performance. These contributions are as follows:\\

\noindent \textbf{Budget-aware pruning with weight reparametrisation.} The two
main challenges when pruning a neural network are first, determining which
weights should be removed and then, mitigating the loss of performance
introduced by weight removal. The first challenge is often referred to as
determining the saliency of the weights, which is a score that reflects the
importance of the weights in the network. The second challenge is often
sidestepped and the pruned network is simply fine-tuned to recover the lost
performance. To address both of these challenges, we propose the following main
contributions:

\begin{itemize}
      \item A numerically stable reparametrisation function, used in both our
            weight reparametrisation and our budget regularisation loss
            (subsequently detailed), that acts as a surrogate differentiable
            $\ell_0$ norm.

      \item A weight reparametrisation that embeds the saliency score of the
            weight in its expression and therefore value. This reparametrisation
            allows to soft-prune the weights during training thereby significantly
            mitigating the performance drop that occurs after pruning. Moreover,
            this reparametrisation does not require the introduction of auxiliary
            variables to determine the saliency of the weights, leading to a
            minimal impact on memory and computational requirements.

      \item A budget regularisation loss that allows to drive the optimisation
            procedure to respect a given budget. This budget regularisation loss
            benefits directly from the aforementioned reparametrisation function
            to compute the current weight budget. It is optimised jointly with
            the original loss, leading to an optimal solution in terms of
            performance and budget.

      \item A comprehensive set of experiments that demonstrate the effectiveness
            of our method and validates each one of its components on various
            datasets and architectures.\\
\end{itemize}

\noindent These contributions have been published in the following article:
\begin{itemize}
      \item Robin Dupont, Hichem Sahbi, and Guillaume Michel. Weight
            reparametrization for budget-aware network pruning. In \emph{2021
            IEEE International Conference on Image Processing, ICIP 2021,
            Anchorage, AK, USA, September 19-22, 2021}, pages 789–793. IEEE,
            2021.\\
\end{itemize}



\noindent \textbf{Pruning without weight training with stochastic sampling.} As
mentioned above, a major hurdle in pruning is determining which weights to
remove. This is especially challenging since weights, and consequently their
saliency, can fluctuate throughout training. This implies that pruning should
either be reversible or performed at the end of training. We propose a different
approach that does not require training the network to determine the saliency of
the weights, the latter being fixed throughout the process. Instead, we sample a
subset of weights (effectively pruning the other weights) forming a subnetwork
of the original network and evaluate its performance. This allows us to search
for a topology that is both lightweight and performant inside the original
network without training its weights. The main contributions of this method are
as follows: \\

\begin{itemize}
      \item A stochastic weight sampling method that is computationally
            efficient, numerically stable, differentiable and allows sampling
            weights while training their probability of being selected,
            represented by latent masks. The optimisation of the latter allows
            to learn the saliency of the weights without training the network,
            and therefore identifying and extracting an effective subnetwork.

      \item A pruning strategy for the masks that freeze the topology and
            performs better than averaging methods previously used in the
            state-of-the-art. Moreover, this pruning strategy allows to discover
            the optimal pruning rate for the network, eliminating the need for
            costly grid search to determine it.

      \item An efficient learnt-based weight rescaling mechanism to compensate
            for the disruption in weight distribution statistics caused by
            stochastic sampling. This rescaling is less computationally
            intensive, more flexible and allows for smoother variations of the
            scaling factor than other rescaling methods.

      \item A comprehensive set of experiments that demonstrate the
            effectiveness of our method and validates each one of its components
            on various datasets and architectures, as well as comparison with
            other closely related state-of-the-art methods in various
            configurations.

      \item A public repository containing the implementation of our method and
            the methods we compare against, as well as detailed code and instructions to
            reproduce our results.\\
\end{itemize}

\noindent These contributions have been published in the following article:
\begin{itemize}
      \item Robin Dupont, Mohammed Amine Alaoui, Hichem Sahbi, and Alice
            Lebois. Extracting effective subnetworks with Gumbel-Softmax. In \textit{2022
                  IEEE International Conference on Image Processing, ICIP 2022, Bordeaux,
                  France, 16-19 October 2022,} pages 931–935. IEEE, 2022.\\
\end{itemize}

\section{Outline}

The rest of this thesis is organised as follows:\\

\Cref{chap:dlo} offers an introduction to deep learning, providing a detailed
overview of its foundational and core concepts. It first explores early
architectures, beginning with the \emph{Perceptron} and the \ac{MLP}. The focus
of the chapter then shifts towards neural network training, giving formal
definitions of the loss function, regularisation, and optimisation process. A
dedicated section delves into \aclp{CNN}, exposing and detailing their building
blocks, and the evolution of their architectures. Then, the architectures used
in the experiments of \cref{chap:chapter1,chap:chapter2} are detailed.
Additionally, this chapter lists and describes prominent datasets, namely
CIFAR-10, CIFAR-100, and TinyImageNet, and discusses their respective train,
validation, and test sets.\\

\Cref{chap:sota} introduces deep neural network compression and presents
state-of-the-art methods divided into different families. The chapter begins
with acceleration techniques and presents a range of methods whose goal is to
speed up matrix operations or convolutions. Then, it explores the teaching
paradigm, highlighting methods that rely on a large pre-trained network to
improve the training of lightweight ones. Furthermore, the chapter addresses the
design aspects of lightweight architectures introducing building blocks for
efficient architecture design and \acl{NAS}. Afterwards, the chapter discusses
methods to compress and optimise existing architectures and in particular
pruning. Finally, the chapter presents the positioning of our methods and the
rationale behind them.\\

\Cref{chap:chapter1} presents our pruning method based on weight
reparametrisation and budget regularisation. It starts by outlining closely
related work. Then, the core method components are examined, starting with our
weight reparametrisation and then our budget loss. Afterwards, a general
overview of the algorithm is provided. Furthermore, the chapter details
experiments assessing our method performance in various configurations as well
as experiments validating the components of our method and the choices of
hyperparameters. A conclusion summarises the key findings and highlights of our
method for neural network pruning.\\

\Cref{chap:chapter2} delves into our stochastic pruning method without weight
training. It starts with an introduction and examination of closely related
work. Then, it details the first core component of our method, namely
\acl{ASLP}, a method for extracting effective subnetworks using the
Gumbel-Softmax technique that solves various issues that arose from previous
methods. Afterwards, it introduces our weight-rescaling technique and presents
its main benefits, as well as our pruning strategy to freeze the stochastic
topology. Subsequently, a method and algorithm overview outlines the key points
of our methods. Furthermore, the chapter exposes a comprehensive set of
experiments that compares our method against other state-of-the-art methods in
various scenarios and validates the components of our method. The chapter
concludes by summarising our findings and results.\\
\chapter{Deep Learning Overview}\label{chap:dlo}

\localtableofcontents

\section{Introduction}

Deep Learning is a subfield of machine learning that focuses on the study of
\acp{DNN} which have their roots in \acp{ANN}. \acp{DNN} aim to learn a data
representation from unstructured data such as raw images
\cite{DBLP:conf/nips/KrizhevskySH12}, text
\cite{DBLP:conf/emnlp/BudzianowskiV19} or audio
\cite{DBLP:journals/corr/HannunCCCDEPSSCN14}, in an end-to-end fashion.
\acp{DNN} have been used to solve a wide range of tasks, including image and
speech recognition
\cite{DBLP:conf/nips/KrizhevskySH12,DBLP:journals/corr/SimonyanZ14a,DBLP:conf/cvpr/HeZRS16,DBLP:journals/corr/HannunCCCDEPSSCN14,DBLP:conf/icassp/ChanJLV16,DBLP:conf/icml/AmodeiABCCCCCCD16},
natural language processing
\cite{DBLP:conf/emnlp/BudzianowskiV19,DBLP:conf/naacl/DevlinCLT19,DBLP:conf/nips/VaswaniSPUJGKP17},
object detection \cite{DBLP:conf/cvpr/RedmonDGF16,DBLP:conf/nips/RenHGS15},
semantic segmentation \cite{long2015fully,DBLP:conf/cvpr/LiuCSAHY019}, text and
image generation
\cite{goodfellow2020generative,karras2019style,DBLP:conf/emnlp/BudzianowskiV19}
as well as exotic domains like video games
\cite{silver2016mastering,silver2018general} or molecules folding
\cite{jumper2021highly}. \acp{ANN} were initially conceptualised based on the
understanding of biological neural networks present in the brain
\cite{mcculloch1943logical,hebb2005organization}.
\citeauthor{rosenblatt1958perceptron} proposed in
\cite{rosenblatt1958perceptron} a theoretical model of a neuron, denoted the
\emph{perceptron}, which was capable of learning a linear decision boundary. The
perceptron model was later extended to multiple layers of neurons, giving rise
to the \ac{MLP} \cite{rosenblatt1961principles,rumelhart1986learning}. A
\acl{MLP} is a type of artificial neural network that extends the concept of a
single-layer perceptron by including one or more hidden layers of neurons
connected upstream to an input layer and downstream to an output layer. Each
layer is fully connected to the next, allowing the model to learn and represent
more complex, non-linear relationships in the input data. Although more capable
than the perceptron, the \ac{MLP} is still limited by its depth. The next
advance came from the stacking of multiple layers, leading to \aclp{DNN}.\\

In the context of \acp{DNN}, the term \emph{deep} denotes the stacking of many
layers within a neural network. The concept of \acp{DNN} is based on the idea
that the depth and the numerous layers can help in learning features at various
levels of abstraction, enabling the network to learn complex hierarchical
patterns. For instance, in the context of image recognition, lower layers learn
local features like edges and textures, while deeper layers learn to identify
more abstract concepts like shapes or objects.\\

The rise of \acp{DNN} was made possible by several factors. On the one hand the
increase in computational power, and in particular the use of \acp{GPU}, made
the training of large and deep networks feasible. Indeed, AlexNet, the first
\ac{CNN} to win the ImageNet Large Scale Visual Recognition Challenge
\cite{DBLP:conf/nips/KrizhevskySH12}, was trained on two \acp{GPU} in parallel
to accelerate computations. Nowadays, the use of \acp{GPU} or dedicated hardware
such as \acp{TPU} \cite{jouppi2017datacenter} is ubiquitous and supported by all
the major deep learning frameworks
\cite{DBLP:journals/corr/AbadiABBCCCDDDG16,DBLP:conf/nips/PaszkeGMLBCKLGA19}. On
the other hand, the availability of large-scale datasets such as ImageNet
\cite{deng2009imagenet} allowed to train or pre-train deep networks with
millions of parameters without overfitting.\\

This chapter aims to give an overview of the different neural network
architectures, building blocks, training techniques and datasets that are widely
used in Deep Learning for computer vision and in our experiments.
\Cref{sec:dlo:early_architectures} introduces the early neural network
architectures, namely the perceptron and the \ac{MLP}. \Cref{sec:dlo:training}
focuses on the functional definition of a neural network and its training.
\Cref{sec:dlo:cnn} presents the building blocks and architectures of various
\acp{CNN} for computer vision, and in particular the ones we benchmark our
methods with (see \cref{sec:chap1:experiments,sec:chap2:experiments}). Finally,
\Cref{sec:dlo:datasets} gives an overview of the most used datasets that we used
in our experiments.


\section{Early Architectures}\label{sec:dlo:early_architectures}

In this section, we present the perceptron \cite{rosenblatt1958perceptron} and
then the \acl{MLP} \cite{rosenblatt1961principles,rumelhart1986learning}. Both
are the two founding neural network architectures that led to the development of
\aclp{DNN}.

\subsection{Perceptron}\label{sec:dlo:perceptron}

The \emph{perceptron} is a model of artificial neuron, capable of learning a
linear decision boundary. It was proposed by
\citeauthor{rosenblatt1958perceptron} in 1958 \cite{rosenblatt1958perceptron}
and conceptualised based on the understanding of biological neural networks
present in the brain \cite{mcculloch1943logical,hebb2005organization}. The
perceptron is composed of inputs that are weighted and summed before being
passed through a nonlinear function referred to as an activation function. The
conceptual representation of the perceptron is displayed in
\cref{fig:dlo:perceptron} and its mathematical formulation is defined in
\cref{eqn:dlo:perceptron}: \\
% and can be express in vector form as written in \cref{eqn:dlo:perceptron_vector}.\\

\begin{equation}
  \label{eqn:dlo:perceptron}
\hat{y} = g(\sum_{i=1}^{n} w_i \cdot x_i + b)
\end{equation} \\


% \begin{equation}
%   \label{eqn:dlo:perceptron_vector}
%   \hat{y} = g(\mathbf{w}^T \mathbf{x} + b)
% \end{equation}\\

\noindent where $x_i$ is the $i$th input, $w_i$ its associated weight, $n$ is
the number of inputs, $b$ is the bias, $g$ is the activation function, and
$\hat{y}$ is the output of the perceptron. This formulation can also be written
in vector form as in \cref{eqn:dlo:perceptron_vector}: \\

\begin{equation}
  \label{eqn:dlo:perceptron_vector}
  \hat{y} = g(\mathbf{w}^T \mathbf{x} + b)
\end{equation}\\

\noindent where $\mathbf{x}$ is the vector of inputs and $\mathbf{w}$ is the
vector of weights. The activation function $g$ is typically a nonlinear
function, such as the sigmoid or the hyperbolic tangent (see
\cref{fig:dlo:activation_functions}). Due to its shallow architecture, the
perceptron cannot learn complex decision boundaries. Nevertheless, it is
possible to stack several perceptrons to learn nonlinear decision boundaries,
leading to a \acl{MLP}.\\

\begin{figure}[htbp]
  \centering
  \includegraphics[width=0.7\textwidth]{chapter_dlo/assets/perceptron_scheme.pdf}
  \caption{Conceptual scheme of the \emph{perceptron}. Each input $x_i$ is multiplied
  by its associated weight $w_i$ and summed to the other weighted inputs. The
  bias $b$ is added to the sum and the result is passed through an activation
  function $g$ to produce the output $\hat{y}$.}
  \label{fig:dlo:perceptron}
\end{figure}

\subsection{Multilayer Perceptron}\label{sec:dlo:mlp}

The \acf{MLP} is an extension of the perceptron model, comprising multiple
layers of perceptrons, also referred to as neurons \cite{rumelhart1986learning}.
A \ac{MLP} with one hidden layer is represented in \cref{fig:dlo:mlp}. In the
latter, the circles represent the neurons and the connections between them,
representing weights, are materialised by lines. The \ac{MLP} is the simplest
type of feedforward \ac{ANN}. Feedforward refers to the fact that the
connections between neurons in the \ac{MLP} form a \acf{DAG}, where the outputs
of the neurons from one layer are passed to the next, with no backward
connections or feedback. Using the same notations as in
\cref{eqn:dlo:perceptron_vector}, the vector form of the \ac{MLP} displayed in
\cref{fig:dlo:mlp} can be written as in \cref{eqn:dlo:mlp}, where the subscript
of activation functions $g_i$, weight matrices $\mathbf{w}_i$ and bias vectors
$\mathbf{b}_i$ denotes their belonging to the $i$th layer.\\

\begin{equation}
  \label{eqn:dlo:mlp}
  \hat{\mathbf{y}} = g_2(\mathbf{w}_2^T \cdot  g_1(\mathbf{w}_2^T \cdot \mathbf{x} + \mathbf{b}_1) + \mathbf{b}_2)
\end{equation}\\

Each layer of the \ac{MLP}, being fully connected to the next one, enables the
\ac{MLP} to handle problems that the perceptron cannot solve, such as problems
requiring nonlinear decision boundaries. Furthermore,
\citeauthor{cybenko1989approximation} proved in \cite{cybenko1989approximation}
that an \ac{MLP} can approximate continuous functions on compact subsets of
$\mathds{R}^n$. This result is known as the \emph{Universal Approximation
Theorem}. Before the emergence of Deep Learning, \acp{MLP} have been applied to
various domains, including voice recognition, image recognition, and machine
translation \cite{wasserman1988neural}.


\begin{figure}[htbp]
  \centering
  \includegraphics[width=0.7\textwidth]{chapter_dlo/assets/mlp_scheme.pdf}
  \caption{Conceptual scheme of a \ac{MLP} with one hidden layer. Each circle
  represents a neuron and each line a connection associated with a weight.}
  \label{fig:dlo:mlp}
\end{figure}

\section{Neural Network Training}\label{sec:dlo:training}

Neural Network Training revolves around the optimisation of a mapping function
that learns to predict an output given input data by adjusting its internal
parameters, also referred to as weights. This optimisation, also called
\emph{training}, involves iteratively tuning these weights so that the
discrepancy between the output predicted by the model and the reference output
is minimised. Weights tuning relies on gradient-based methods that hinge
around two core components: the \emph{backpropagation} algorithm to compute the
gradients and the \ac{SGD} algorithm to update the weights.

\subsection{Functional Definition}

Neural networks can be defined as a mapping function from an input space
$\mathcal{X}$ to an ouput space $\mathcal{Y}$. This mapping function $f$ is
characterised by a set of parameters $\theta$, often called \emph{weights}. The
training of a neural network consists in tuning the parameters $\theta$ so that,
given an input $X$, the mapping function $f$ output, denoted $\hat{y}$, is as
close as possible to the associated true output $y$. This training is done
iteratively by using example pairs $(X, y) \in \mathcal{X} \times \mathcal{Y}$,
where $X\in\mathcal{X}$ is the input and $y\in\mathcal{Y}$ is the output. In the
context of image classification, $X$ is an image and $y$ is a label that
indicates the class of the associated image. A functional representation of a
neural network is given in \cref{eqn:dlo:nn_functional_definition}, where $f$ is
the neural network, $\theta$ is the set of parameters of the network,
$X\in\mathcal{X}$ is the input given to the neural network and $\hat{y}$ is the
output.\\

\begin{equation}
  \label{eqn:dlo:nn_functional_definition}
  % \centering
  \begingroup
  \setlength\arraycolsep{0pt}
  f \colon\begin{array}[t]{c >{{}}c<{{}} l}
    \mathcal{X} & \to     & \mathcal{Y} \\
    X                     & \mapsto & f(X, \theta) = \hat{y}
  \end{array}
  \endgroup
\end{equation}\\


Considering image classification,  the output $\hat{y}$ is a probability vector
where the largest coefficient is the one whose index corresponds to the
predicted class of the input image. This vector is generally converted into a
one-hot vector, where the only non-zero coefficient is at the index of the
predicted class. The true label $y$, referred to as the ground truth, is the
class index so that $y\in\llbracket0;C-1\rrbracket$, where $C$ is the number of
classes considered. The ground truth can also be converted into a one-hot
vector.\\

% In the case of image classification, the input $X$ given to the neural network
% is an image, and the output $\hat{y}$ is a probability vector where the largest
% coefficient is the one whose index corresponds to the predicted class of the
% input image. $y_i$, often referred to as the ground truth is a one-hot vector
% where the only non-zero coefficient is at the index corresponding to the true
% class of the input image.\\

% A neural network is a function that maps an input $X$ to an output $y$ through
% a series of transformations. The functional definition of a neural network is
% given in \cref{eqn:dlo:nn_functional_definition}, where $f$ is the neural
% network, $\theta$ is the set of parameters of the network, and $y$ is the
% output.\\

\subsection{Loss Function and Regularisation}
Training a neural network aims at finding the optimal parameters $\theta$ that
maximises a performance, quantified by a metric $P$, often based on the
discrepancy between the predicted output $\hat{y}$ and the true output $y$.
However, optimising directly the metric $P$ might be intractable. To solve for
this issue, one may define a differentiable cost function and minimise the
latter as a proxy for optimising $P$. Considering $\delta$ as the empirical
distribution of the training data, the cost function $\mathcal{J}(\theta)$, also
referred to as the \emph{empirical risk}, is defined in the following equation:

\begin{equation}
  \label{eqn:dlo:cost_function}
  \mathcal{J}(\theta) = \mathds{E}_{(X, y) \sim \delta} \left[ \mathcal{L}(f(X,\theta), y) \right]
\end{equation}\\

\noindent where $\mathcal{L}$ is the loss function. Note that the true data
distribution is not known, and thus, the empirical distribution $\delta$ is used
instead. The minimisation of the empirical risk alone is not sufficient to
ensure good overall performance. Indeed, the neural network could learn to
perfectly predict the output of the training set but fail to generalise to
unseen data. This phenomenon is called \emph{overfitting}. To prevent
overfitting, we add a regularisation term to the empirical risk. The
regularisation term, denoted $\mathcal{R}$, is a function of the parameters
$\theta$ of the neural network which penalises the complexity of the model, and
thus prevents overfitting. To account for regularisation, the cost function in
\cref{eqn:dlo:cost_function} is updated to:

\begin{equation}
  \label{eqn:dlo:regularised_cost_fn}
  \mathcal{J}_r(\theta) = \mathds{E}_{(X, y) \sim \delta} \left[ \mathcal{L}(f(X,\theta), y) \right] + \mathcal{R}(\theta) 
\end{equation}\\

\noindent\textbf{Loss function.} In
\cref{eqn:dlo:cost_function,eqn:dlo:regularised_cost_fn}, the loss function
$\mathcal{L}$ is a measure of the discrepancy between the ground truth $y$ and
the predicted output. Contrary to the metric $P$ which might be
non-differentiable, the loss function is differentiable so that its minimisation
can be achieved using gradient-based methods, subsequently detailed in
\cref{sec:dlo:backpropagation}. The choice of the loss function depends on the
task at hand. For classification tasks (not only images), the loss function is
often the \emph{cross-entropy} loss. For a binary classification problem, the
ground truth is a binary variable $y\in \{0,1\}$ and the predicted output is a
scalar $f(X,\theta)=\hat{y}\in[0,1]$. The binary cross-entropy loss is defined
as follows:\\

\begin{equation}
  \label{eqn:dlo:binary_cross_entropy_loss}
  \mathcal{L}(\hat{y}, y) = - y \log(\hat{y}) - (1-y) \log(1-\hat{y})
\end{equation}\\

\noindent The binary cross-entropy loss defined in
\cref{eqn:dlo:binary_cross_entropy_loss} can be extended to problems with more
than two classes. For a classification problem with $C$ classes, the ground
truth is a one-hot vector $\mathbf{y}\in \{0,1\}^C$ and the output is a
$C$-dimensional vector $f(X,\theta)=\hat{\mathbf{y}}\in\mathds{R}^C$. The
multi-class cross-entropy loss is defined as follows:

\begin{equation}
  \label{eqn:dlo:multiclass_cross_entropy_loss}
  % \mathcal{L}(\hat{\mathbf{y}}, \mathbf{y}) = - \sum_{i=1}^c y_i \log \left( \displaystyle\frac{\exp(\hat{y}_i)}{\displaystyle\sum_{j=1}^c \exp(\hat{y}_j)} \right)
  \mathcal{L}(\hat{\mathbf{y}}, \mathbf{y}) = - \sum_{i=1}^C y_i \log \left( \phi(\mathbf{\hat{y}})_i \right)
\end{equation}\\


\noindent In the above equation, $\hat{\mathbf{y}}$ is the unnormalised raw
output vector of the neural network and $\phi$ is the softmax function, whose
expression is given in \cref{eqn:dlo:softmax}. The softmax function is used to
convert the raw output vector of real numbers into a probability distribution.
Note that some models which use the softmax as the activation function of their
last layer output directly a probability distribution, in which case the softmax
is not needed.\\

Considering a vector $\mathbf{z} = [z_1, \dots, z_n]$, the $j$-th component of
vector $\mathbf{z}$ normalised by the softmax function is given by: 

\begin{equation}
  \label{eqn:dlo:softmax}
  % \phi(\mathbf{z})_j = \frac{\exp(z_j)}{\displaystyle\sum_{k=1}^{\text{dim}(\mathbf{z})} \exp(z_k)}
  \phi(\mathbf{z})_j = \frac{\exp(z_j)}{\displaystyle\sum_{k=1}^{n} \exp(z_k)}
\end{equation}\\

\noindent \textbf{Regularisation.} The regularisation term $\mathcal{R}$ is a
differentiable function of the weights $\theta$. It acts as a control mechanism
to avoid overfitting by preventing the weights of the neural network from
becoming too large, which can lead to overly complex models that overfit the
training data. This is typically achieved by adding a penalty proportional to
the magnitude of the weights, thereby keeping them small.\\

Common types of regularisation include $\ell_1$ and $\ell_2$ regularisation,
whose expressions are shown in \cref{eqn:dlo:reg_l1,eqn:dlo:reg_l2}
respectively. $\ell_1$ regularisation \cite{tibshirani1996regression}, adds a
penalty equal to the absolute value of the magnitude of the weights. On the
other hand, $\ell_2$ regularisation \cite{hoerl1970ridge}, adds a penalty
equivalent to the square of the magnitude of the weights. Both methods aim to
reduce the magnitude of the weights, but $\ell_1$ regularisation is more
targeted towards feature selection, effectively pushing some weights to $0$,
whereas $\ell_2$ restrains globally their magnitude.\\

The regularisation term $\mathcal{R}$ is added to the cost function with a
regularisation coefficient, usually denoted as $\lambda$, which is a
hyperparameter that balances the trade-off between fitting the training data
(minimising the loss $\mathcal{L}$) and limiting the complexity of the model
(minimising $\mathcal{R}$).\\

For a network with L layers and parameters $\theta={\mathbf{w}_1, \dots,
\mathbf{w}_L}$, the $\ell_1$ and $\ell_2$ regularisation term is defined as
follows:\\

\begin{equation}
  \label{eqn:dlo:reg_l1}
  \mathcal{R}_{\ell_1}(\theta) = \lambda \sum_{i=1}^{L} \left| \mathbf{w}_i \right|
\end{equation}\\

\begin{equation}
  \label{eqn:dlo:reg_l2}
  \mathcal{R}_{\ell_2}(\theta) = \frac{\lambda}{2} \displaystyle \sum_{i=1}^{L} \| \mathbf{w}_i \|_2 
\end{equation}\\

\subsection{Loss Optimisation}\label{sec:dlo:backpropagation}

As mentioned before, the training of a neural network involves finding the
optimal set of parameters $\theta$ that minimises a cost function
$\mathcal{J}(\theta)$. This process of optimisation is typically carried out
using gradient-based methods which rely on the iterative adjustment of the
parameters in the opposite direction of the gradient of the cost function. The
gradient of a function provides the direction of the steepest ascent at a given
point \cite{boyd2004convex}. Thus, by moving the parameters in the opposite
direction of the gradient, we seek to descend to a local minimum of the
function.\\

\noindent \textbf{Backpropagation.} One critical step in the optimisation
process is the computation of the gradient of the cost function with respect to
the parameters, $\nabla \mathcal{J}(\theta)$. These gradients are computed with
the \emph{backpropagation} algorithm \cite{rumelhart1986learning} which is an
application of the \emph{chain rule} (see \cref{eqn:dlo:chain_rule}) to
efficiently compute these gradients. It involves a forward pass through the
network to compute the outputs and thus the loss, and a backward pass to
calculate the gradients. During the backward pass, the partial derivative of the
cost with respect to each parameter is computed, starting from the output layer
and going back to the input layer. The previously computed derivatives from the
subsequent layers are used to compute the ones of the earlier layers,  
making the backpropagation algorithm computationally efficient.\\

% Chain rule latex equation
\begin{equation}
  \label{eqn:dlo:chain_rule}
  \frac{\partial z}{\partial x} = \frac{\partial z}{\partial y} \frac{\partial y}{\partial x}
\end{equation}\\

\noindent \textbf{\acl{SGD}.} Once the gradients are calculated, they are used
to update the parameters. The most prevalent method for parameter updates is
\acf{SGD}, a derivative of the Robbins–Monro algorithm
\cite{robbins1951stochastic}. In \ac{SGD}, the gradient of the loss function is
computed for a random subset of the data (a \emph{batch} or \emph{mini-batch}),
and the weights are shifted in the direction that decreases the loss function.
This is achieved by subtracting the gradient of the cost function with respect
to that parameter multiplied by a learning rate $\eta$:\\

\begin{equation}
\label{eqn:dlo:sgd_update}
\theta_i^{(t+1)} = \theta_i^{(t)} - \eta \frac{\partial \mathcal{J}(\theta)}{\partial \theta_i}
\end{equation}\\

\noindent where $\theta_i^{(t)}$ is the $i$th parameter at iteration $t$. The
\ac{SGD} algorithm is detailed in \cref{alg:dlo:sgd}. The use of mini-batches in
\ac{SGD} leads to a trade-off between computational efficiency and estimation
accuracy. Indeed, the gradient is estimated using a subset of the entire
training set, which is, on the one hand, less accurate than using the whole
dataset, but on the other hand, less computationally intensive. The size of the
mini-batch, which is a hyperparameter of the training algorithm, determines this
trade-off and should also be chosen depending on the computational and memory
resources available. Note that the size of modern datasets, subsequently
detailed in \cref{sec:dlo:datasets}, makes it intractable to evaluate the
gradients on the whole dataset in one step, hence the use of mini-batches.\\

\begin{algorithm}
  \caption{Stochastic Gradient Descent Algorithm}
  \label{alg:dlo:sgd}
  \begin{algorithmic}
    \REQUIRE {Learning rate $\eta$, mini-batch size $m$, Initial parameters
    $\theta^{(0)}$, $n \geq m$ training pairs $(X,y)\in
    \mathcal{X}\times\mathcal{Y}$, Loss function $\mathcal{J}$} \WHILE {Stopping
    criterion not met} \STATE {Sample mini-batch of size $m$ from training set}
    \STATE {Compute gradient estimate on mini-batch: $\hat{g} \gets \nabla
    \mathcal{J}(\theta)$} \STATE {Update parameters: $\theta^{(t+1)} \gets
    \theta^{(t)} - \eta \hat{g}$} \ENDWHILE \RETURN {Optimal parameters
    $\theta$}
  \end{algorithmic}
\end{algorithm}


\noindent \textbf{Learning Rate.} The learning rate is a hyperparameter that
determines the step size of the update at each iteration while moving toward a
minimum of the loss function (see \cref{eqn:dlo:sgd_update}). Setting the
learning rate too high can cause the learning process to converge too quickly or
overshoot while setting it too low can make the learning process slow to
converge, as shown in \cref{fig:dlo:gradient_descent}.\\

\begin{figure}[htbp]
  \centering
  \includegraphics[width=0.7\textwidth]{chapter_dlo/assets/gradient_descent.pdf}
  \caption{Illustration of the effect of the learning rate on the convergence of
    the gradient descent. The gradient descent has been applied iteratively for
    20 epochs. On the one hand, a too-high learning rate ($\eta=1.01$) causes
    the gradient descent to overshoot the minimum of the loss function. On the
    other hand, a too-low learning rate ($\eta=0.01$) causes the gradient
    descent to converge slowly.}
  \label{fig:dlo:gradient_descent}
\end{figure}

\noindent \textbf{Alternative methods.} To enhance the performance of \ac{SGD},
various modifications and extensions have been proposed, such as \ac{SGD} with
momentum \cite{sutskever2013importance,polyak1964some}, RMSProp
\cite{hinton2012neural}, or Adam \cite{kingma2014adam}. These methods aim to
adjust the learning rate dynamically or dampen the oscillations in the gradient
descent to achieve faster and more stable convergence.\\

For instance, \ac{SGD} with momentum
\cite{sutskever2013importance,polyak1964some} uses a momentum coefficient
$\gamma$ and smoothes the variations of the descent direction and thus,
preventing the optimisation to get stuck in small local minimums. The momentum
term is a moving average of the gradient, here denoted $v$, and it is used to
update the parameters as shown in \cref{eqn:dlo:sgd_momentum_update}. In this
equation, the momentum coefficient $\gamma \in [0,1]$ is a hyperparameter that
is typically set close to 1, $0.9$ being a common value.\\

\begin{equation}
  \label{eqn:dlo:sgd_momentum_update}
  \begin{split}
    v_{t+1} &= \gamma v_t + \eta \nabla \mathcal{J}(\theta) \\
    \theta_{t+1} &= \theta_t - v_{t+1}
  \end{split}
\end{equation}\\

\section{Convolutional Neural Networks for Computer Vision}\label{sec:dlo:cnn}

In the field of computer vision, \acp{CNN} have emerged as effective
architectures that enable high performance on image classification tasks. The
effectiveness of \acp{CNN} lies in their architecture that leverages the \ac{CL}
layers to automatically learn abstract features from visual data in a
hierarchical fashion. In this section, we explore the building blocks of
\acp{CNN} and various architectures that have been widely used and became
\emph{de facto} standards in the literature.

\subsection{Building Blocks}

This section covers the most common building blocks of \acp{CNN} for computer
vision. These building blocks are organised in layers that are stacked to form
neural network architectures subsequently detailed in
\cref{sec:dlo:architectures,sec:dlo:architectures_used}.\\

\noindent\textbf{Convolutional layer.} \ac{CL} layers are one of the core
building blocks of \acp{CNN}. Each convolution layer performs a series of
spatial convolutions on the input data using a set of learnable filters or
kernels. These filters are designed to extract low-level features such as edges,
corners, and textures in the early layers, while they learn high-level features
like object parts or even whole objects in the deeper layers. Contrary to manual
feature engineering, the features learned by \ac{CL} layers are learned in a
\emph{end-to-end} fashion. The 2D convolution operation is defined in
\cref{eqn:dlo:convolution} :\\

\begin{equation}
  \label{eqn:dlo:convolution}
  % (x * k)(i,j) = \sum_{m} \sum_{n} x(i -m, j -n) \cdot k(m, n)
    Y_{ij} = \sum_{{a=0}}^{{k_h-1}} \sum_{{b=0}}^{{k_w-1}} X_{i-a, j-b} \cdot K_{ab}    
\end{equation}\\

\noindent where $\mathbf{X}$ is the input, $\mathbf{K}$ is the kernel of size
$k_h \times k_w$ and $(i,j)$ are the spatial coordinates in the output feature
map. Note that some Deep Learning frameworks implement \emph{cross-correlation}
instead of convolution. In the former, the kernel is not spatially flipped
leading to the cross-correlation not being commutative
\cite{goodfellow2016deep}. The \ac{CL} layer kernels are typically smaller than
the input along width and height dimensions (they are generally $3\times 3$
\cite{DBLP:conf/cvpr/HeZRS16}) but comprises as much channels as the input.
During the forward pass, each kernel is spatially convolved channel-wise with
the input and the convolution outputs are summed along the channel dimension to
yield a single scalar for each kernel position on the input (see also
\cref{fig:dlo:conv_layer}).\\

\ac{CL} layers are more computationally efficient than \ac{FC} layers, as they
have a form of weight sharing baked in. Indeed, the same kernel is applied to
every location of the input, which brings two main benefits: \emph{(i)} the
number of parameters is independent of the input size and \emph{(ii)} a single
learned kernel, acting as a feature detector, can be used in multiple locations.
This is especially useful for early feature detector that detects basic shapes
or textures. In addition, because of the kernels being convolved across the
whole input, \ac{CL} layers are also less sensitive to spatial translations that
might occur in different instances of the same class.\\

\noindent \textbf{Fully connected layer.} \ac{FC} layers, also known as
\emph{Dense} layers are often the last layers of a \ac{CNN}, effectively serving
as a classifier, whereas the \ac{CL} layers act as a feature extractor.
\ac{FC} layers perform high-level reasoning by conducting non-linear
transformations of the extracted features and combining them to make decisions.
In an FC layer, each neuron is connected to every neuron in the previous layer.
A \ac{FC} layer can be describe as a matrix-vector product as in
\cref{eqn:dlo:fc_layer} (see \cref{fig:dlo:dense_layer}).\\

\begin{equation}
  \label{eqn:dlo:fc_layer}
  \mathbf{y} = \mathbf{w}^T \cdot \mathbf{x} + \mathbf{b}
\end{equation}\\

\noindent where $\mathbf{x}$ is the input vector, $\mathbf{w}$ is the weight
matrix and $b$ is the bias. In the context of \acp{CNN}, before passing the
output of the last \ac{CL} layer to the first \ac{FC} layer, it needs to be
flattened or reshaped into a single column vector. The final layer in a \ac{CNN}
is a \ac{FC} layer that has a number of neurons equal to the number of output
classes, and it typically uses a softmax activation to output a probability
distribution over those classes.\\

%FIXME: rajouter phrase pour dire que l'on peut remplacer les layers FC par des
%layers conv pour des réseaux "fully convolutional"


\begin{figure}[htbp]
  \centering
  \subfloat[Convolutional Layer\label{fig:dlo:conv_layer}]{
    \includegraphics[width=0.49\textwidth]{chapter_dlo/assets/conv_layer.pdf}}
    \subfloat[Fully Connected Layer\label{fig:dlo:dense_layer}]{
    \includegraphics[width=0.49\textwidth]{chapter_dlo/assets/dense_layer.pdf}}
    \caption{Conceptual representation of a \acl{CL} and a \acl{FC} layer. The
    \acl{CL} layer (\cref{fig:dlo:conv_layer}) takes a multi-channel input and
    produces a multi-channel output. Each coefficient of the output is computed
    by applying a convolution operation at a corresponding location in the
    input. The \acl{FC} layer (\cref{fig:dlo:dense_layer}) takes a vector input
    and produces a vector output. Each connection is represented by a weight in
    the weight matrix.}
  \label{fig:sota:layers}
\end{figure}

\noindent \textbf{Activation function.} These functions are often applied to the
output feature map of a convolutional or fully connected layer, resulting in the
\emph{activation map} or \emph{activations}. This function introduces
non-linearity into the model, allowing it to learn more complex patterns
\cite{long2015fully}. A common activation function used in \acp{CNN} is the
\ac{ReLU}, represented as $f(x)=\max(0,x)$. Other functions like the sigmoid
$f(x)=1/(1+e^{-x})$ or tanh $f(x)=(e^{x} -e^{-x})/(e^{x}+e^{-x})$ functions have
been used (see \cref{fig:dlo:activation_functions}), however, the \ac{ReLU} is
preferred over the latter for its computational efficiency and its ability to
mitigate the vanishing or exploding gradient problem
\cite{hochreiter2001gradient,glorot2010understanding}.\\

\begin{figure}[htbp]
  \centering
  \includegraphics[width=0.7\textwidth]{chapter_dlo/assets/activation_functions.pdf}
  \caption{\ac{ReLU}, tanh and sigmoid activation functions. Best viewed in
  colours.}
  \label{fig:dlo:activation_functions}
\end{figure}

\noindent \textbf{Pooling.} Pooling is often employed after \ac{CL} layers in a
\ac{CNN} and aims at progressively reducing the spatial extent of the input
representation, thus reducing the number of parameters and computations in the
network. This also helps control overfitting and increases the receptive field
of the subsequent layers. The pooling operation is performed independently on
each input channel, so the number of channels remains unchanged. The two most
common types of pooling are \emph{max} and \emph{average pooling}. The former
selects the maximum value in each window (often of size $2\times 2$), while the
latter computes the average value of the window. Given an input matrix
$\mathbf{X}$, the output matrix $ \mathbf{Y} $ for a
certain spatial location $(i, j)$ is defined in \cref{eqn:dlo:max_pooling} for
\emph{max pooling} and \cref{eqn:dlo:avg_pooling} for \emph{average pooling}:\\

% \begin{equation}
%   \label{eqn:dlo:pooling}
%   Y_{ij} = p \left( \mathbf{X}(i⋅s:i⋅s+f,j⋅s:j⋅s+f) \right)
% \end{equation}\\

% Max Pooling
\begin{equation}
  \label{eqn:dlo:max_pooling}
  Y^{\text{max}}_{ij} = \max_{(a,b) \in [0, k_h-1] \times [0, k_w-1]} X_{i + a, j + b}
\end{equation}\\
  
  % Average Pooling
  \begin{equation}
  \label{eqn:dlo:avg_pooling}
  Y^{\text{avg}}_{ij} = \frac{1}{k_h \times k_w} \sum_{a=0}^{k_h-1} \sum_{b=0}^{k_w-1} X_{i + a, j + b}  
\end{equation}\\

\noindent where $k_h$ and $k_w$ represent the height and width of the pooling
windows respectively. Note that pooling has no learnable parameters. They only
downsample the input based on a fixed function.\\


\noindent \textbf{Batch Normalisation.} \ac{BN} is a technique introduced in
\cite{DBLP:conf/icml/IoffeS15} to combat the issue of internal covariate shift
in deep neural networks, thereby accelerating training and improving
generalization. Covariate shift refers to the changes in the distribution of
features in the training and test dataset, which can lead to slow convergence,
make the network harder to train or hinder its generalisation capabilities.
\ac{BN} normalises the input of the layer by adjusting and scaling the
activations of the previous one. For each mini-batch of inputs (for instance,
the activation map of the previous layer), it computes the mean and variance of
the activations and performs normalization. The transformation is defined as
follows:\\

\begin{equation}
  \label{eqn:dlo:batchnorm}
  \hat{x}_{i} = \frac{x_{i} - \mu_{B}}{\sqrt{\sigma_{B}^{2} + \varepsilon}}
\end{equation}\\

\noindent where $x_{i}$ is the input, $\mu_{B}$ is the mini-batch mean,
$\sigma_{B}^{2}$ is the mini-batch variance, and $\varepsilon$ is a small
constant for numerical stability. After normalization, the method allows the
network to learn an affine transformation for each activation, permitting the
network to control the mean and standard deviation of the input distribution,
formalised in \cref{eqn:dlo:batchnorm}:\\

\begin{equation}
  \label{eqn:dlo:batchnorm_affine}
  y_{i} = \gamma \hat{x}_{i} + \beta
\end{equation}\\

\noindent where, $\gamma$ and $\beta$ are the learnable parameters of the affine
transformation. \ac{BN} has the advantage of making the network less sensitive
to the initial weights, allowing higher learning rates, and reducing the need
for Dropout, among other regularisers. However, its effectiveness decreases in
the case of small batch sizes, as the estimate of the batch mean and variance
becomes less accurate.\\

\noindent \textbf{Dropout.} Dropout is a regularization technique used to
prevent overfitting in neural networks. Dropout was introduced in
\cite{DBLP:journals/jmlr/SrivastavaHKSS14} and works by randomly deactivating a
proportion of neurons in a layer during each training iteration. More
specifically, during the forward pass, each neuron has a probability $p$ of
being temporarily removed from the network, effectively breaking up
co-adaptations between neurons and forcing them to learn more robust and
independent features. The output of Dropout is given in \cref{eqn:dlo:dropout}:\\

\begin{equation}
  \label{eqn:dlo:dropout}
    r_i \sim \text{Bernoulli}(1 - p), \quad
    \mathbf{y} = \frac{\mathbf{x} \odot \mathbf{r}}{1 - p} 
\end{equation}\\

\noindent In the above equation, $\mathbf{x}$ denote the output of a layer
processed with dropout, $\mathbf{r}$ is a binary mask vector of the same shape
as $\mathbf{x}$, where each element of $r$ is independently drawn from a
Bernoulli distribution with probability $1-p$, leading to $r_i=1$ if the
associated weight is kept and a $r_i=0$ if not. The product $\mathbf{x} \odot
\mathbf{r}$ is scaled by $1-p$ to ensure that the expected value of $\mathbf{x}$
remains unchanged. During the evaluation, the dropout is changed to an identity
function.

\subsection{Architectures Evolution}\label{sec:dlo:architectures}

The evolution of \aclp{CNN} is characterised by a consistent increase in their
size and performance, alongside the introduction of new architectural
modifications to address the limitations of their predecessors (see
\cref{fig:dlo:net_sizes}). In this section, we present a historical overview of
the \acp{CNN} evolution and we subsequently detail the architectures that we
used in our experiments.\\

One of the earliest \ac{CNN} was introduced in 1998: LeNet-5 was developed for
digit recognition \cite{DBLP:journals/pieee/LeCunBBH98}, constituting a
relatively simple network with 5 layers with learnable parameters: 2 \ac{CL}
layers and 3 fully connected layers. Its size is significantly smaller compared
to the contemporary models (see \cref{fig:dlo:net_sizes}). With the introduction
of AlexNet \cite{DBLP:conf/nips/KrizhevskySH12} in 2012, the network size
considerably grew, comprising more layers and neurons to handle more complex
tasks, like large-scale image recognition. AlexNet tackled the overfitting issue
in LeNet-5 with the use of data augmentation and dropout techniques and
introduced it also introduced and popularised the \ac{ReLU} activation
function.\\

\begin{figure}[htbp]
  \centering
  \includegraphics[width=0.70\textwidth]{chapter_sota/assets/lenet.png}
  \caption{Architecture of LeNet-5, a \acl{CNN} used for handwritten digit
    recognition. Image taken from \cite{DBLP:journals/pieee/LeCunBBH98}}
  \label{fig:dlo:lenet5}
\end{figure}


The next advancement was the VGG networks family
\cite{DBLP:journals/corr/SimonyanZ14a} %, introduced in 2014 
which proposed much deeper architectures with up to 19 layers, which is a
significant increase over the 8 layers of AlexNet. However, the increased depth
led to the \emph{vanishing gradients} problems. The vanishing gradients problem
refers to the situation in training a deep neural network where gradients are
backpropagated through layers and become increasingly small, effectively
preventing the weights of earlier layers from learning and updating effectively.
The VGG networks also introduced the practice of stacking multiple convolutional
layers with small $3\times 3$ filters instead of using larger ones. The same
year, Google's Inception (or GoogLeNet) \cite{DBLP:conf/cvpr/SzegedyLJSRAEVR15}
was introduced, addressing the vanishing gradients  issue with its novel
inception modules, which allowed the network to learn at varying scales and
increased computational efficiency, without overly increasing the network size.
GoogleNet was also the first \ac{CNN} that was not a simple stack of layers and
processed a single input with different blocks in parallel before merging
them.\\

\begin{figure}[htbp]
  \centering
  \includegraphics[width=0.7\textwidth]{chapter_sota/assets/vgg16.png}
  \caption{Architecture of the VGG16 network introduced in
    \cite{DBLP:journals/corr/SimonyanZ14a}. Image taken from
    \cite{ferguson2017automatic}}
  \label{fig:dlo:vgg16}
\end{figure}

Later, the ResNet models family was proposed in \cite{DBLP:conf/cvpr/HeZRS16},
which effectively tackled the vanishing gradient problem by introducing skip (or
shortcut) connections, allowing gradients to backpropagate directly through
several layers. These shortcut connections also allowed the network to grow in
depth up to 152 layers without a significant increase in computational cost.
However, a challenge remained with the constant need for careful design to
manage feature-map sizes. Indeed, stacking numerous layers, with their channel
count increasing with depth, can lead to an explosion in the number of
parameters as well as increased memory consumption.\\

\begin{figure}[htbp]
  \centering
  \includegraphics[width=0.5\textwidth]{chapter_dlo/assets/skip_connection.pdf}
  \caption{A residual block and its skip connection used in
    ResNets\cite{DBLP:conf/cvpr/HeZRS16}. The identity skip connection allows
    for the gradient to be backpropagated directly through several layers, thus
    mitigating the \emph{vanishing gradients} problem.}
  \label{fig:dlo:skip_connection}
\end{figure}

In response, DenseNet \cite{huang2017densely} was proposed. It connects each
layer to every other following layer of the same block in a feed-forward
fashion. By reinforcing the propagation of features and gradients through the
network, the DenseNet architecture alleviates the vanishing-gradient problem and
further improves the information flow from earlier layers to later ones by
reusing earlier features in the deeper layers. Thus, through these chronological
advancements, neural networks not only grew in size but also improved in
performance, thereby becoming more efficient and capable of handling more
complex tasks.\\

\begin{figure}[htbp]
  \centering
  \includegraphics[width=0.7\textwidth]{chapter_sota/assets/network_sizes_normal.pdf}
  \caption{Networks size comparison. The \emph{x-axis} represents the number of
    \acp{FLOP} required to process a single image. The \emph{y-axis} represents
    the Top-1 accuracy on the ImageNet \cite{deng2009imagenet} dataset and the
    size of the circles represents the number of parameters in the network.
    Numbers are taken from \cite{pytorch_vision}}
  \label{fig:dlo:net_sizes}
\end{figure}

\subsection{Architectures Used in Experiments}\label{sec:dlo:architectures_used}

In the subsequent paragraphs, we detail the architectures that we used in our
experiments. We chose these architectures because they are representative of the
state-of-the-art in image classification and they are widely used in the pruning
literature. \Cref{tab:dlo:networks_size} gives an overview of the different
network architectures.\\

\begin{table}[ht!]
  \centering
  \begin{tabular}{lcccccc}
    \cline{2-7}
    & \textbf{Conv2} & \textbf{Conv4} & \textbf{Conv6} & \textbf{VGG16} & \textbf{ResNet20} & \textbf{ResNet18} \\ \hline
    Number of Parameters & 4,301,642 & 2,425,930  & 2,262,602      & 14,728,266
    & 269,034           & 11,685,608 \\
    Number of layers & 5 & 7 & 9 & 14 & 20 & 18 \\
    Number of \ac{CL} layers & 2 & 4 & 6 & 13 & 19 & 17 \\
    Number of \ac{FC} layers & 3 & 3 & 3 & 1 & 1 & 1 \\ \hline
  \end{tabular}
  \caption{Number of parameters for the used neural network architectures. The
  number of parameters is given for the CIFAR-10 dataset, except for the
  ResNet18 architecture, whose number of parameters is given for the
  TinyImageNet dataset.}
  \label{tab:dlo:networks_size}
\end{table}

% VGG16
\noindent \textbf{VGG16.} The VGG16 network
\cite{DBLP:journals/corr/SimonyanZ14a} is a 16-layer \ac{CNN} composed of 13
\ac{CL} layers and 3 fully connected layers. VGG16 was originally designed
for ImageNet \cite{deng2009imagenet} and in our experiments with CIFAR-10 and
CIFAR-100 (described in \cref{sec:dlo:datasets}) we use a slightly modified
version of VGG16 where we replace the 3 \ac{FC} layers with an average pooling
layer and a single \ac{FC} layer \cite{vggcifar}. The \ac{CL} layers
filters are of size $3\times 3$ with a stride of 1. The max-pooling layers are
of size $2\times 2$ with a stride of 2. Each \ac{CL} layer is followed by
a \ac{ReLU} activation function. The VGG16 network is illustrated in
\cref{fig:dlo:vgg16_cifar}.\\

\begin{figure}[htbp]
  \centering
  \includegraphics[width=0.30\textwidth]{chapter_dlo/assets/vgg16_cifar.pdf}
  \caption{VGG16 adapted for CIFAR-10 and CIFAR-100.}
  \label{fig:dlo:vgg16_cifar}
\end{figure}

% ResNet18 and ResNet20
\noindent \textbf{ResNet\{18,20\}.} The ResNet18 and ResNet20 networks
\cite{DBLP:conf/cvpr/HeZRS16} are 18 and 20-layer \acp{CNN} respectively. These
layers are organized into stages, with ResNet18, represented in
\cref{fig:dlo:resnet18}, consisting of 4 stages with 2 \emph{Basic Blocks}
(detailed subsequently) 2 each, while ResNet20, represented in
\cref{fig:dlo:resnet20}, is structured into 3 stages, each containing 3
\emph{Basic Blocks}. The \emph{Basic Blocks}, also referred to as \emph{Residual
Blocks}, are composed of \ac{CL} layers (see
\cref{fig:dlo:resnet18,fig:dlo:resnet20}) and follow the principle of learning
the residual function:\\

\begin{equation}
  f(x) = h(x) - x
\end{equation} \\

\noindent where $h(x)$ is the mapping usually learned by previous architectures
such as VGG16. The representation of a residual block is given by
\cref{eqn:dlo:residual_block} (see also \cref{fig:dlo:skip_connection}): \\

\begin{equation}
  \label{eqn:dlo:residual_block}
  y = f(x,\theta) + x
\end{equation}\\

\noindent where $x$ is the input, $f$ represents the residual function, $\theta$
are the weights of the block, and $y$ is the output. In
\cref{eqn:dlo:residual_block} $+ x$ denotes the skip connection, which enables
direct backpropagation of the gradient to earlier layers.\\

\begin{figure}
\centering
\subfloat[ResNet20\label{fig:dlo:resnet20}]{
  \includegraphics[width=0.35\textwidth]{chapter_dlo/assets/resnet20.pdf}}
  \hfill
\subfloat[ResNet18\label{fig:dlo:resnet18}]{
  \includegraphics[width=0.60\textwidth]{chapter_dlo/assets/resnet18.pdf}}

\caption{ResNet20 and ResNet18 architectures. ResNet20 (\cref{fig:dlo:resnet20})
is tailored for CIFAR-10 and comprises 3 stages encompassing 3 \emph{Basic
Blocks} of 2 \ac{CL} layers each, with an identity skip connection in each
block. ResNet18 (\cref{fig:dlo:resnet18}) is tailored for ImageNet and is
composed of 4 stages encompassing 4 \emph{Basic Blocks} of 2 convolutional
layers each. There are two types of blocks: $\mathbf{B}_\text{I}$ with an
identity skip connection and $\mathbf{B}_\text{P}$ with a projection skip
connection. The projection skip connection is used to match the dimensions
between the input and the output of the block.}
\label{fig:dlo:resnets}
\end{figure}

\noindent \textbf{Conv\{2,4,6\}.} Conv2, Conv4 and Conv6 are shrunk down
versions of the VGG16 network architecture, composed of 2, 4 and 6 \ac{CL}
layers respectively and 3 \ac{FC} layers. Although Conv2, Conv4 and Conv6,
introduced by \citeauthor{DBLP:conf/iclr/FrankleC19} in
\cite{DBLP:conf/iclr/FrankleC19}, are not widely featured in existing
literature, we chose to employ them due to their use in the methods we benchmark
against. The \ac{CL} layers are stacked in increasing depth, and their
convolutional filters are of size $3\times 3$ with a stride of 1. The
max-pooling layers are of size $2\times 2$ with a stride of 2. They are
represented in \cref{fig:dlo:conv246}.\\

\begin{figure}[htbp]
  \centering
  \includegraphics[width=0.7\textwidth]{chapter_dlo/assets/conv246.pdf}
  \caption{Conv2, Conv4 and Conv6 architectures. The number of flat features
  $\mathbf{F}$ corresponds to the size of the feature map of the last block
  $\mathbf{B}$, once vectorised. $\mathbf{F}=16384, ~8192~ \text{and}~ 4096$ for Conv2, Conv4
  and Conv6, respectively for input images of size $32\times 32$.}
  \label{fig:dlo:conv246}
\end{figure}


\section{Datasets}\label{sec:dlo:datasets}

In this thesis, we focus on image classification and supervised learning, a
machine learning paradigm in which the model is trained using labelled data. In
the context of image classification, the labelled data are pairs of images and
labels which are the class of their associated image. We denote an input image
$X$ and its corresponding label $y$. Each image $X$ belongs to the set of all
images of the dataset $\mathcal{X}$, and each label $y$ belongs to the set of
all labels of the dataset $\mathcal{Y}$. The ensemble of the image-label pairs
are gathered in a dataset, denoted $\mathcal{D}$, which is formally a set of
pairs $(X, y)$, where $X \in \mathcal{X}$ and $y \in \mathcal{Y}$, so that $D
\subset \mathcal{X} \times \mathcal{Y}$. \\

In our experiments, we evaluated our methods on three different datasets
tailored for image classification: CIFAR-10 \cite{CIFARdataset}, CIFAR-100
\cite{CIFARdataset} and TinyImageNet \cite{TinyImageNet}. The following
paragraphs give details about these datasets and \cref{tab:dlo:datasets} sums
up their main characteristics.\\

\begin{table}[ht!]
  \centering
  \begin{tabular}{lcccc}
    \toprule
    \textbf{Dataset}    & \textbf{Number of images} & \textbf{Number of classes} &
    \textbf{Image size} & \textbf{Size of test set}                                               \\
    \hline
    CIFAR-10            & 60,000                    & 10                         & 32x32 & 10,000 \\
    CIFAR-100           & 60,000                    & 100                        & 32x32 & 10,000 \\
    TinyImageNet        & 100,000                   & 200                        & 64x64 & 10,000 \\
    \bottomrule
  \end{tabular}
  \caption{The number of images, of classes, image size and size of the test
    set for the three datasets used: CIFAR-10, CIFAR-100 and TinyImageNet.}
  \label{tab:dlo:datasets}
\end{table}

\subsection{CIFAR-10}

CIFAR-10 \cite{CIFARdataset} is a widely used dataset in machine learning and
computer vision. This is a labelled subset of the \emph{80 Million Tiny Images}
dataset \cite{4531741}. CIFAR-10 is a simple yet challenging dataset that allows
for quicker iteration or hyperparameter tuning than larger datasets such as
ImageNet \cite{DBLP:journals/ijcv/RussakovskyDSKS15}, but it is significantly
more complex than the MNIST dataset \cite{6296535}, which contains grayscale
handwritten digits images. the CIFAR-10 dataset contains 60,000 colour images of
size 32x32 pixels, split into 10 classes, namely: plane, car, bird, cat, deer,
dog, horse, ship, and truck. Each class contains 6,000 images. The dataset is
divided into two sets: a training set, composed of 50,000 images and a test set
containing 10,000 of them.\\

\begin{figure}[ht!]
  \centering
  \includegraphics[width=0.7\textwidth]{chapter_dlo/assets/cifar-10_example.png}
  \caption{ A sample of images from CIFAR-10. Each row contains images from one
    of the 10 classes: plane, car, bird, cat, deer, dog, frog, horse, ship, and
    truck}
  \label{fig:intro:cifar10_examples}
\end{figure}


\subsection{CIFAR-100}

CIFAR-100 \cite{CIFARdataset} is a more challenging
version of CIFAR-10. Like the latter, it is a labelled subset of the \emph{80
  Millions Tiny Images} and  is composed of 60,000 colour images of size 32x32
pixels. However, instead of 10 classes, CIFAR-100 contains 100 classes of 600
images each. As a result, each class has far fewer images than in CIFAR-10.
CIFAR-100 is also divided into two sets: a training and a test set, composed of
50,000 and 10,000 images respectively.\\

\begin{figure}[ht!]
  \centering
  \includegraphics[width=0.7\textwidth]{chapter_dlo/assets/cifar-100_example.png}
  \caption{A sample grid of images from the CIFAR-100 dataset. The images
    represent a subset of the 100 available classes, each image represents
    a unique class.}
  \label{fig:intro:cifar100_examples}
\end{figure}


\subsection{TinyImageNet}

TinyImageNet dataset is another popular dataset in machine learning and computer
vision, conceived as a subset of the larger ImageNet dataset
\cite{DBLP:journals/ijcv/RussakovskyDSKS15}. It comprises 100,000 colour images
of size 64x64 pixels, split into 200 classes, whereas ImageNet contains 1.2
million images of size 256x256 pixels, split into 1,000 classes. The dataset is
divided in 3 sets: the train set, which contains 500 images per class, the
validation and test sets, which both contain 50 images. The scaled-down image
size and the reduced number of images make TinyImageNet more computationally
manageable than ImageNet while still being challenging by offering diversity in
the image classes.\\


\begin{figure}[ht!]
  \centering
  \includegraphics[width=0.9\textwidth]{chapter_dlo/assets/tinyimagenet_example.png}
  \caption{A sample of images from the Tiny ImageNet dataset. Each image
    represents an instance of one of the  200 distinct classes.}
  \label{fig:intro:tinyimagenet_examples}
\end{figure}

\subsection{Train, Validation and Test Sets}

In our experiments, for each dataset, we use 3 distinct sets for training,
validation and testing. The training set serves to train the model, while the
validation set is used to monitor the evolution of the performance metric on
unseen data throughout the training. The validation metric provides the
necessary triggers for the early stopping policy (\emph{i.e.} interrupting the
training prematurely if the validation metrics do not change over a given number
of iterations). The test set, on the other hand, is used to evaluate the model's
performance on entirely new data and to report the final test accuracy. When
utilizing datasets like CIFAR-10 and CIFAR-100, only training and testing sets
are available. For these datasets, we split the given train set using the
following proportions: 90\% is used for training the network and the remaining
10\% is for validation. On the other hand, the TinyImageNet dataset does provide
training, validation, and testing sets, but the test set lacks annotations.
Hence, we use 90\% of the original training set for model training and the
remaining 10\% for validation. Instead of the original unannotated test set, we
repurpose the original validation set to serve as the test set. This is a common
strategy employed by other implementations
\cite{hanyuanxu2018tinyimagenet,nbdt,alvinwan2020nbdt}.\\

\chapter{Deep Neural Network Compression}\label{chap:sota}

\localtableofcontents

\section{Introduction}\label{sec:sota:intro}

% region: intro
The fast development of neural networks has led, on the one hand, to the
enhancement of their performance, but also, on the other hand, to a significant
growth in size and parameter count. The rapid evolution and adoption of these
networks has given rise to various applications
\cite{DBLP:conf/nips/KrizhevskySH12,DBLP:conf/emnlp/BudzianowskiV19,silver2018general,jumper2021highly},
particularly embedded ones\cite{kim2020review,kuutti2020survey}, whose resources
are highly constrained in terms of computing power, energy consumption and
memory footprint. Alongside the increase in the size of these networks,
compression techniques
\cite{DBLP:conf/nips/CunDS89,DBLP:journals/corr/HanMD15,DBLP:conf/nips/HanPTD15}
have been devised, in order to enable the use of these algorithms in embedded
applications or resource-constrained environments.\\

This chapter focuses on state-of-the-art neural network compression methods,
predominantly based on various operations applied to the weights of an already
existing large neural network. This chapter is organised as follows:
\Cref{sec:sota:fast_convolutions} examines fast convolution techniques, which
aim to accelerate the computation of convolutions in neural networks, thereby
reducing both the runtime and computational resources required. Thereafter,
\cref{sec:sota:teaching_paradigm} delves into \ac{KD}, a process by which the
knowledge of a larger, more complex network (referred to as the \emph{teacher})
is transferred to a smaller and more efficient network (called the
\emph{student}), enabling the latter to achieve comparable performance with a
reduced footprint. Subsequently, \cref{sec:sota:archi_design} explores
architecture design methods that aim at producing more efficient and effective
networks. \Cref{sec:sota:efficient_archi} details ad-hoc architectures, referred
to as \emph{Efficient Architectures}. These architectures are lightweight
networks that revolve around a core technique to reduce their size while
preserving performance as much as possible. Hence, \cref{sec:sota:nas} discusses
\ac{NAS}, a method that automates the discovery of optimal network architectures
tailored to specific tasks or constraints, potentially leading to more compact
and efficient designs. Afterwards, \cref{sec:sota:refining_existing} presents
two categories of techniques that harness an existing neural network and refine
its architecture to produce a more compact and efficient model. First,
\cref{sec:sota:quantisation} focuses on \emph{quantisation} and
\emph{binarisation} techniques, which aim to lower the numerical precision of
weights and activations of networks in order to speed up their computation and
reduce their memory footprint. Lastly, \cref{sec:sota:pruning} considers neural
network pruning, which seeks to remove redundant or insignificant connections and
weights from networks, resulting in sparser and more computationally efficient
models.\\

%TODO: faire une mindmap des méthodes

% endregion: intro

\section{Accelerating Computations in Neural Networks}\label{sec:sota:fast_convolutions}

Among various operations and functions used in neural networks, two fundamental
mathematical operations, convolution and matrix multiplication are used
extensively and are the backbone of most computations in neural networks.
However, performing these operations can be computationally demanding,
particularly with large and complex networks. This may lead to long computation
times, posing a challenge for real-time or resource-limited applications. To
mitigate this issue, some research efforts have focused on developing techniques
to speed up these operations. These strategies encompass optimizing the
underlying algorithms to leveraging hardware acceleration, with the objective of
enhancing the speed and efficiency of neural network computations.\\


\subsection{\acl{FFT}}\label{sec:sota:fft}
The most popular algorithms for
accelerating convolution operations rely on the \ac{FFT}
\cite{DBLP:conf/nips/ChiJM20,DBLP:journals/npl/LinY19,DBLP:conf/pkdd/PrattWCZ17},
and leverage the Convolution Theorem. The Convolution Theorem states that the
convolution of two signals in the source domain is the product of the two
signals in the Fourier domain, as shown in \cref{eqn:sota:conv_theorem_fourier}:\\

\begin{equation}
  \label{eqn:sota:conv_theorem_fourier}
  F(x * y) = F(x) \cdot F(y)
\end{equation}\\

\noindent where $x$ and $y$ are the two signals in the source domain, $x*y$ is
the convolution of $x$ and $y$ and finally $F(x)$ and $F(y)$ are the Fourier
transforms of $x$ and $y$, respectively. Then, to obtain the result of the
convolution in the source domain, the inverse Fourier transform, denoted
$F^{-1}$, is applied as follows:\\

\begin{equation}
  x * y = F^{-1}(F(x) \cdot F(y))
\end{equation}\\

\noindent The convolution theorem allows for faster computation of the 2D
convolution by using the \ac{FFT} to compute the convolution in the frequency
domain, and the inverse \ac{FFT} to convert the result back to the source domain
\cite{oppenheim1997signals}.

\subsection{Optimised Matrix Multiplication Algorithms}\label{sec:sota:matrix_multiplication}

It is possible to accelerate matrix multiplication by directly optimising the
underlying algorithm. The  Strassen algorithm \cite{strassen1969gaussian}, used
in \cite{DBLP:conf/icann/CongX14}, is a fast method for matrix multiplication
that reduces the computational complexity from the standard $\mathcal{O}(n^{3})$
to approximately $\mathcal{O}(n^{2.807})$ by recursively dividing the matrices
of size $n$ into 4 submatrices of size $\frac{n}{2} \times \frac{n}{2}$,
reorganising and combining these multiplications to perform only 7 instead of 8
matrix multiplications (see
\cref{eqn:sota:matrix_multiplication_strassen,eqn:sota:matrix_multiplication_strassen_result}).\\

Considering a matrix multiplication of two square matrices $A$ and $B$ of size
$2n$ with $n\in\mathbb{N}$, defined in \cref{eqn:sota:matrix_multiplication},
the output of the standard bloc matrix multiplication, referred to as $C$, is
defined in \cref{eqn:sota:matrix_multiplication_result}. Note that $A_{ij}$ and
$B_{ij}$ are either a scalar if $n=1$, or a matrix of size $\frac{n}{2} \times
\frac{n}{2}$.\\

\begin{equation}
  \label{eqn:sota:matrix_multiplication}
  A = \begin{bmatrix} A_{11} & A_{12} \\ A_{21} & A_{22} \end{bmatrix},
  B = \begin{bmatrix} B_{11} & B_{12} \\ B_{21} & B_{22} \end{bmatrix}
\end{equation}\\

\noindent The computation of $C$ requires 8 matrix multiplications, as shown in
\cref{eqn:sota:matrix_multiplication_result}.\\
\begin{equation}
  \label{eqn:sota:matrix_multiplication_result}
  C = A \cdot B = \begin{bmatrix} A_{11}B_{11} + A_{12}B_{21} & A_{11}B_{12} + A_{12}B_{22} \\ A_{21}B_{11} + A_{22}B_{21} & A_{21}B_{12} + A_{22}B_{22} \end{bmatrix}
\end{equation}\\

\noindent The Strassen algorithm reduces the number of multiplications to 7 by
defining the following 7 products, referred to as $P_{i}$, with $i\in \llbracket
1;7 \rrbracket$:\\

\begin{equation}
  \label{eqn:sota:matrix_multiplication_strassen}
  \begin{aligned}
    P_{1} & = (A_{11} + A_{22})(B_{11} + B_{22}) \\
    P_{2} & = (A_{21} + A_{22})B_{11}            \\
    P_{3} & = A_{11}(B_{12} - B_{22})            \\
    P_{4} & = A_{22}(B_{21} - B_{11})            \\
    P_{5} & = (A_{11} + A_{12})B_{22}            \\
    P_{6} & = (A_{21} - A_{11})(B_{11} + B_{12}) \\
    P_{7} & = (A_{12} - A_{22})(B_{21} + B_{22})
  \end{aligned}
\end{equation}\\

\noindent The result of the matrix multiplication is then obtained by combining
these products, as shown in
\cref{eqn:sota:matrix_multiplication_strassen_result}.\\

\begin{equation}
  \label{eqn:sota:matrix_multiplication_strassen_result}
  C = \begin{bmatrix}
    P_{1} + P_{4} - P_{5} + P_{7} & P_{3} + P_{5}                 \\
    P_{2} + P_{4}                 & P_{1} - P_{2} + P_{3} + P_{6}
  \end{bmatrix}
\end{equation}\\

\noindent The Strassen algorithm has later been refined by
\citeauthor{coppersmith1987matrix}, who introduced the Coppersmith-Winograd
algorithm \cite{coppersmith1987matrix}. The latter brings down the complexity to
$\mathcal{O}(n^{2.376})$. This algorithm is used in various works, mostly
targeted towards a specific \ac{FPGA} \ac{CPU}
\cite{liu2018efficient,lu2018spwa,wang2020winonn}.\\

\subsection{Leveraging Matrix Structures}\label{sec:sota:matrix_structures}
Using a particular matrix structure also speeds up the standard operations used
in a neural network. Fully connected layers can be effectively accelerated by
forcing the use of specific matrix structures. For instance,
\citeauthor{DBLP:conf/iccv/ChengYFKCC15} devised a method where dense layers
standard operation is replaced by a circulant projection
\cite{DBLP:conf/iccv/ChengYFKCC15}. The circulant matrix can be stored in a
memory-efficient way and can be further sped up with \ac{FFT}. $C$ is an example
of a circulant matrix (see \cref{eqn:sota:circulant_matrix}).\\

\begin{equation}
  \label{eqn:sota:circulant_matrix}
  C =
  \begin{pmatrix}
    a & b & c & d \\
    d & a & b & c \\
    c & d & a & b \\
    b & c & d & a
  \end{pmatrix}
  \quad
  T =
  \begin{pmatrix}
    a & b & c & d \\
    e & a & b & c \\
    f & e & a & b \\
    g & f & e & a \\
  \end{pmatrix}
\end{equation}\\

Likewise, convolutional operations can be accelerated thanks to Toeplitz
matrices. A Toeplitz matrix, or diagonal-constant matrix, has the unique
characteristic of each descending diagonal from left to right being constant.
$T$ is an example of a Toeplitz matrix (see \cref{eqn:sota:circulant_matrix}).
This property is particularly useful for convolutions, as they can be expressed
as a multiplication by a Toeplitz matrix \cite{gray2006toeplitz}, as shown in
\cref{eqn:sota:toeplitz_matrix_conv}. This algorithm has been used in
\cite{liao2019compressing}, focusing on \ac{FPGA} architectures. Note that the
representation of convolution as a product with a Toeplitz matrix can further be
accelerated by using the aforementioned optimisations to the matrix
multiplication algorithm, such as the Strassen or Coppersmith-Winograd
algorithm.\\


Let $x$ be a signal of length $N$ and $h$ be a kernel of length $M$, expressed
as a Toeplitz matrix $H$. The convolution of $x$ and $h$ can be expressed as:\\

\begin{equation}
  \label{eqn:sota:toeplitz_matrix_conv}
  h * x = Hx = \begin{bmatrix}
    h_0     & 0       & \cdots & 0       \\
    h_1     & h_0     & \cdots & 0       \\
    \vdots  & \vdots  & \ddots & \vdots  \\
    h_{M-1} & h_{M-2} & \cdots & h_0     \\
    0       & h_{M-1} & \cdots & h_1     \\
    \vdots  & \vdots  & \ddots & \vdots  \\
    0       & 0       & \cdots & h_{M-1}
  \end{bmatrix}
  \begin{bmatrix}
    x_0     \\
    x_1     \\
    \vdots  \\
    x_{N-1} \\
  \end{bmatrix}
\end{equation}\\

\subsection{Practical Applications and Limitations}
The algorithms presented in
\cref{sec:sota:fft,sec:sota:matrix_multiplication,sec:sota:matrix_structures}
offer significant acceleration in the computation of convolution operations and
are widely implemented and used in state-of-the-art software
\cite{pytorch_vision,DBLP:journals/corr/AbadiABBCCCDDDG16}. In particular, the
Coppersmith-Winograd algorithm is used in various Deep Learning frameworks
\cite{DBLP:journals/corr/AbadiABBCCCDDDG16,DBLP:conf/nips/PaszkeGMLBCKLGA19} or
neural network \ac{GPU} libraries \cite{nvidia_cudnn} where the fastest
algorithm is automatically selected based on the tensor sizes and the hardware.
However, depending on the operand size, the total processing speed can be bound
to the hardware and more specifically, to the memory throughput and data access
time, more than the computation time
\cite{DBLP:journals/pc/WhaleyPD01,DBLP:journals/cca/DrevetIS10}.\\

\section{Teaching Paradigm}\label{sec:sota:teaching_paradigm}

The teaching paradigm embraces a class method that aims to transfer the
knowledge of a large, complex and accurate network, referred to as the
\emph{teacher}, to a smaller and more efficient one called the \emph{student}.
The student is trained with a combination of the main task loss as well as a
supplementary supervision signal which is derived from the feature maps of the
teacher network at various depths.\\


\subsection{Knowledge Distillation}
Methods in the teaching paradigm are mostly based on the seminal work of
\citeauthor{DBLP:journals/corr/HintonVD15} \cite{DBLP:journals/corr/HintonVD15},
better known as \acf{KD}. The latter seeks to train simple networks with \ac{KD}
yielding better performances compared to those trained from scratch. \ac{KD}
relies on teacher and student networks, where the logits of the former are used
as an additional supervision signal for the latter. When trained separately, the
student network can only rely on classification labels in order to learn its own
data representation while \ac{KD} relies on the logits of the trained teacher
network which provide more insight about the latent data representation.\\

For a classification problem, the loss used to train the student network with
\ac{KD} can be expressed as:\\

\begin{equation}
  \mathcal{L}_{\text{total}} = \underbrace{\mathcal{L}_{\text{CE}}(\hat{y}_{s}, y)\vphantom{\left(\frac{\hat{y}_{s}}{T}, \frac{\hat{y}_{t}}{T}\right)}}_{\text{Task loss}} +
  \lambda \frac{T^{2}}{2}\underbrace{\mathcal{L}_{\text{CE}}\left(\frac{\hat{y}_{s}}{T}, \frac{\hat{y}_{t}}{T}\right)}_{\text{Distillation loss}}
\end{equation}
\\

\noindent where $\mathcal{L}_{\text{CE}}$ is the cross-entropy loss,
$\hat{y}_{s}$ and $\hat{y}_{t}$ are the logits of the student and teacher
networks respectively, $y$ is the ground truth label, $T$ is the temperature
parameter and $\lambda$ is a mixing coefficient balancing the two losses. Note
that the distillation loss is scaled by $\frac{T^2}{2}$ to ensure that the
relative contribution of the task loss and distillation loss stays balanced if
the temperature changes.\\

\subsection{Feature-Map Matching}
Inspired by \ac{KD}, \cite{DBLP:journals/corr/RomeroBKCGB14} introduced FitNet,
a two-stage training algorithm, where an intermediate layer of the teacher is
chosen as a \emph{hint}\footnote{\emph{Hint} is the terminology used by
  \citeauthor{DBLP:journals/corr/RomeroBKCGB14}
  \cite{DBLP:journals/corr/RomeroBKCGB14} to denote a feature map used as a
  target for the student network.} for an intermediate layer of the student.
  Initially, the first layers of the student are trained to mimic the hint
  feature map. Then, the whole student network is trained with standard \ac{KD}
  against the whole teacher. In the first step, a regressor is needed in order
  to adapt the dimensions of the feature map, which may differ from the teacher
  to the student networks, as illustrated in \cref{fig:sota:kd_frameworks}.
  \citeauthor{DBLP:conf/cvpr/YimJBK17} argue that the direct feature map
  matching utilised by FitNets is overly restrictive. Drawing inspiration from
  the techniques used in \cite{DBLP:journals/corr/GatysEB15a} for style
  transfer, they propose an alternative method. In the context of style
  transfer, the Gram matrix of the feature maps is employed to encapsulate the
  texture information of an image. Adapting this approach, the method presented
  in \cite{DBLP:conf/cvpr/YimJBK17} calculates the Gram matrix across the
  feature maps of multiple layers. This computed Gram matrix, dubbed the Flow of
  Solution Procedure matrix, then serves as a \emph{hint} for the student
  network, guiding its training process. In practice, handling full-dimensional
  feature maps is cumbersome. That is why, in order to avoid this issue,
  \cite{DBLP:conf/iclr/ZagoruykoK17} use an attention map generated by squashing
  the feature maps to a 2D map allowing for a smaller 2D regressor to match
  attention map dimensions.\\

\begin{figure}[htbp]
  \centering
  \includegraphics[width=0.75\textwidth]{chapter_sota/assets/kd_frameworks.pdf}
  \caption{Overview of various knowledge distillation frameworks. From top to
    bottom, left to right: Deep Mutual Learning
    \cite{DBLP:conf/cvpr/ZhangXHL18}, FitNet
    \cite{DBLP:journals/corr/RomeroBKCGB14}, Attention Transfer
    \cite{DBLP:conf/iclr/ZagoruykoK17}, Teacher Assistant
    \cite{DBLP:conf/aaai/MirzadehFLLMG20} and Knowledge Distillation
    \cite{DBLP:journals/corr/HintonVD15}.}
  \label{fig:sota:kd_frameworks}
\end{figure}

\subsection{Deep Mutual Learning}

Note that the aforementioned knowledge transfer methods require teacher-student
pairs and assume that teachers are large trained models.
\cite{DBLP:conf/cvpr/ZhangXHL18} relax this assumption by proposing \emph{Deep
Mutual Learning}, which enables a pool of networks of different architectures to
learn together, provided that they have the same logit dimensions, and none of
the models in the pool requires a pretraining step. The uncertainty of each
model is distilled into each other, which creates additional knowledge.\\

\subsection{Teacher Assistant}
In all the aforementioned methods, the efficacy of knowledge distillation, and
consequently, the final performance of the student network, is significantly
influenced by the disparity in size between the student and teacher networks.
This size discrepancy, when excessive, may cause the student network to
encounter difficulties in aligning with the teacher logits, thus preventing
optimal knowledge distillation. To tackle this issue,
\citeauthor{DBLP:conf/aaai/MirzadehFLLMG20} introduced the concept of
\emph{Teacher Assistant}: networks of intermediary dimensions aiming at bridging
the size gap between student and teacher \cite{DBLP:conf/aaai/MirzadehFLLMG20}.
The \ac{TA} approach proposes to ensure effective knowledge transfer through a
stepwise transfer of knowledge, starting from the teacher to the \ac{TA}, and
finally from the \ac{TA} to the student. This technique allows each model to
learn from a slightly simpler model than itself. Empirical evidence shows that
the \ac{TA} approach tends to outperform traditional one-step distillation in
various experiments and across different network architectures, resulting in
improved performances. However, it is important to note that it does introduce
additional computational overhead due to the necessity of additional training
steps for the \ac{TA}, and careful selection of the size and number of \acp{TA}.
These considerations underscore that while the \ac{TA} strategy is effective in
managing the size disparity problem, it also adds complexity to the distillation
process.\\


\begin{figure}[htbp]
  \centering
  \includegraphics[width=0.7\textwidth]{chapter_sota/assets/variational_info_distillation.pdf}
  \caption{Conceptual scheme of \cite{DBLP:conf/cvpr/AhnHDLD19}. The student
    network efficiently learns the main task while retaining high mutual
    information with the teacher network. The mutual information is
    maximised by learning to estimate the distribution of the activations in
    the teacher network, provoking the transfer of knowledge. Adapted from
    the original scheme found in \cite{DBLP:conf/cvpr/AhnHDLD19}.}
  \label{fig:sota:vid_scheme}
\end{figure}

\subsection{Alternative Distillation Losses}
Other approaches that do not rely on direct feature map or logit matching have
been proposed. \cite{DBLP:conf/cvpr/AhnHDLD19} introduced \emph{Variational
  Information Distillation}, which indirectly maximises the mutual information
between the student and the teacher. This is done by using \emph{variational
  information maximisation} \cite{barber2004algorithm} to maximise a variational
lower bound of the mutual information, since directly maximising the latter is
intractable in practice (see \cref{fig:sota:vid_scheme}). Likewise,
\cite{DBLP:conf/eccv/PassalisT18} proposed a \emph{Probabilistic Knowledge
  Transfer} method that does not match logits or feature maps, but rather
represents the latter as a probability distribution and minimises divergence
between the two (see \cref{fig:sota:pkt_scheme}).\\


\begin{figure}[htbp]
  \centering
  \includegraphics[width=0.7\textwidth]{chapter_sota/assets/pkt_diagram.pdf}
  \caption{Conceptual scheme of the Probabilistic Knowledge Transfer method.
    Both the student and the teacher feature maps are modelled using probability
    distributions. The divergence of the latter is minimised in order to
    transfer knowledge from the teacher to the student. Illustration taken from
    \cite{DBLP:conf/eccv/PassalisT18}.}
  \label{fig:sota:pkt_scheme}
\end{figure}

\section{Architecture Design}\label{sec:sota:archi_design}

The architectural design of neural networks, while contributing significantly to
their performance, often inflates their computational and memory requirements.
This increased complexity, although beneficial for the final performance, could
limit the deployment of these networks in resource-constrained environments.
Thus, formulating effective and efficient neural network architectures is of
significant importance. The design of neural networks is a problem that not only
involves designing suitable building blocks but also determining their
organization and interconnections. This section scrutinizes these aspects by
focusing on handmade and automatic efficient architecture design.\\

\Cref{sec:sota:efficient_archi} introduces building blocks to design efficient
architectures. These building blocks have been meticulously engineered in the
state-of-the-art to strike a balance between computational efficiency and
performance. Properly incorporating these blocks can result in architectures
better suited to their operating environments, enhancing efficiency while
maintaining the desired level of performance.\\

Thereafter, \cref{sec:sota:nas} delves into the field of \acl{NAS}. The primary
aim of \ac{NAS} is to design, in an automatic fashion, network architectures
that demonstrate a high level of efficiency and performance for a given task. By
doing so, it eliminates the need for manual design and the associated iterative
trial-and-error approaches that would otherwise be necessary to assess and
evaluate the impact and effectiveness of each design decision
\cite{DBLP:journals/corr/HowardZCKWWAA17,DBLP:conf/cvpr/SandlerHZZC18,DBLP:conf/iccv/HowardPALSCWCTC19}.
Although \ac{NAS} was not initially targeted at generating lightweight
architectures, the principles and methods described in this section can be
adapted to optimise the architecture search for efficiency and compactness. \\

This section explores techniques aimed at the creation of efficient and
effective neural networks through the careful selection and assembly of
optimised building blocks. The organization of these components plays an
important role in network compression and optimization, highlighting that high
performance can also be reached with designs that are less resource-demanding.\\

\subsection{Building Blocks for Efficient Architecture Design}\label{sec:sota:efficient_archi}
% region: depthwise separable convolutions

\textbf{Depthwise Separable Convolutions.} One of the initial strategies towards
achieving efficiency in neural network architectures is the use of depthwise
separable convolutions. This technique, used in MobileNet
\cite{howard2017mobilenets} and EfficientNet \cite{DBLP:conf/icml/TanL19},
separates the standard convolution operation into two distinct steps: a
depthwise convolution and a pointwise convolution (see
\cref{fig:sota:depthwise_conv_vs_standard_conv}). By decomposing the operations
in this manner, the computational complexity is markedly reduced while still
retaining the ability to capture spatial and channel-wise information. Consider
an input feature map with $C_\text{in}$ channels of arbitrary width and height
and $C_\text{out}$ convolution kernels of size $k\times k \times C_\text{in}$. A
standard convolution algorithm will need $C_\text{in} \times C_\text{out} \times
  k \times k$ \ac{MAC} operations to produce a $1 \times 1 \times C_\text{out}$
element of the output feature map. In contrast, a depthwise separable
convolution algorithm will first apply a $k\times k \times 1$ convolution kernel
to the $C_\text{in}$ channels and then perform $C_\text{out}$ pointwise
convolutions with $1\times 1 \times C_\text{in}$ kernels to produce the same
$1\times 1 \times C_\text{out}$ element. This effectively reduces the number of
parameters to $C_\text{in} \times (C_\text{out} + k \times k)$, essentially
reducing the number of computations required to produce a $1 \times 1 \times
  C_\text{out}$ element by a factor of\\

$$\displaystyle\frac{C_\text{out}\times k \times k}{C_\text{out} + k \times k}.$$\\

\begin{figure}[htbp]
  \centering
  \subfloat[Standard Convolution\label{fig:sota:standard_convolution}]{
    \includegraphics[width=0.70\textwidth]{chapter_sota/assets/standard_conv_scheme.pdf}}\\
  \vspace{1cm}
  \subfloat[Depthwise Separable
    Convolution\label{fig:sota:depthwise_convolution}]{
    \includegraphics[width=0.70\textwidth]{chapter_sota/assets/depthwise_sep_conv_scheme.pdf}}
  \caption{Illustration schemes of the standard and depthwise separable
    convolution. The standard convolution uses $C_\text{out}$ kernels of size
    $k\times k \times C_\text{in}$. The depthwise separable convolution is
    split into two steps: \emph{(i)} a convolution with $C_\text{in}$ kernels
    of size $k \times k$ and \emph{(ii)} a convolution with $C_\text{out}$
    kernels of size $1\times 1 \times C_\text{in}$.
    Best viewed in colours.}
  \label{fig:sota:depthwise_conv_vs_standard_conv}
\end{figure}

% endregion: depthwise separable convolutions

% region: fire module
\noindent \textbf{Fire Module.} An alternative approach for designing efficient
architectures involves the integration of \emph{fire modules}, as proposed in
\cite{DBLP:journals/corr/IandolaMAHDK16}. These modules, represented in
\cref{fig:sota:fire_module}, aim to minimise computational requirements by
employing two distinct strategies: \emph{(i)} diminishing the number of input
channels supplied to the following conventional $k\times k$ convolutions and
\emph{(ii)} substituting a portion of the resource-intensive $k\times k$
convolutions with pointwise convolutions, which possess $k^2$ times fewer
parameters. The initial strategy is applied within the \emph{Squeeze Layer} of
the \emph{fire module}, which decreases the number of input channels delivered
to the \emph{Expand Layer}, subsequently reducing the number of parameters in
the \emph{Expand Layer} kernels. The second strategy is implemented in the
\emph{Expand Layer}, where some $3\times3$ convolutions are replaced with
$1\times1$ variants. Although the $1\times1$ convolutions capture less spatial
information, they are significantly less computationally demanding than the
$3\times3$ ones.\\

\begin{figure}[htbp]
  \centering
  \includegraphics[width=0.70\textwidth]{chapter_sota/assets/fire_module.pdf}
  \caption{Illustration scheme of the fire module. The fire module is composed
    of a \emph{squeeze layer} (pointwise convolution designed to reduce the
    number of channels fed to the following layer) and an \emph{expand layer}
    (convolution with mixed $1\times1$ and $3\times3$ kernels. The $1\times1$
    kernels replace some of the $3\times3$ kernels, being less computationally
    intensive.). Best viewed in colours.}
  \label{fig:sota:fire_module}
\end{figure}

% endregion: fire module

% region: shufflenet

\noindent\textbf{ShuffleNet.} Pushing the concept of depthwise separable
convolutions further, \cite{ZhangShuffleNet} introduces pointwise group
convolutions and channel shuffle operations to enhance efficiency while
maintaining accuracy. Pointwise group convolutions were initially introduced in
\cite{DBLP:conf/nips/KrizhevskySH12}, though their original purpose was not for
compression. Instead, group convolutions in \cite{DBLP:conf/nips/KrizhevskySH12}
were used to enable distributed training across multiple \acp{GPU} with limited
memory. However, ShuffleNet \cite{ZhangShuffleNet} leverages this concept for
network efficiency by dividing the input channels into groups and performing
convolutions on each group independently. This approach reduces the number of
operations and the computational cost compared to traditional convolutions. To
counteract the potential loss of expressive power caused by the separation of
channels into groups, ShuffleNet incorporates \emph{channel shuffle operations}
as shown in \cref{fig:sota:shuffle_net}. This technique allows for information
exchange between groups, effectively maintaining accuracy by ensuring that
different groups can capture diverse features in the input.\\

\begin{figure}[htbp]
  \centering
  \includegraphics[width=0.70\textwidth]{chapter_sota/assets/group_conv_and_channel_shuffling.pdf}
  \caption{Illustration scheme of grouped convolution with channel shuffling.
    Each filter only acts on a subset of the input tensor (here represented by a
    matching colour). The channels of the yielded tensor are shuffled to ensure
    the subsequent groups can access information from all the previous groups.
    Best viewed in colours.}
  \label{fig:sota:shuffle_net}
\end{figure}

\noindent\textbf{Learned group convolutions.} Following ShuffleNet, CondenseNet was
introduced in \cite{huang2018condensenet},
incorporating learned group convolutions to further enhance efficiency. Unlike
the predefined group convolutions in ShuffleNet, CondenseNet learns which
channels should be grouped together, enabling the network to adapt its structure
for a specific task. This results in better utilisation of network capacity and
reduces redundancy. CondenseNet leverages the DenseNet architecture
\cite{huang2017densely} to further improve performance. Thanks to the densely
connected architecture, features discarded in any layer can still be recovered
in subsequent ones.\\

\begin{figure}[htbp]
  \centering
  \includegraphics[width=0.7\textwidth]{chapter_sota/assets/channel_split.pdf}
  \caption{Illustration scheme of the path taken by the feature maps after the
    channel split block. Adapted from the original scheme found in \cite{MaShuffleNetV2}.}
  \label{fig:sota:channel_split}
\end{figure}

\noindent\textbf{ShuffleNetV2.} Building on the success of ShuffleNet,
ShuffleNetV2 was introduced in \cite{MaShuffleNetV2}, focusing on enhancing
network efficiency through the combination of strided convolution and channel
split. Strided convolution helps to reduce the spatial extent of feature maps,
thereby reducing the computation cost. The Channel Split technique efficiently
processes the input feature maps while maintaining the expressive power of the
architecture. Channel Split works by dividing the input feature maps into two
equal parts. One part is passed through the main branch of the ShuffleNet unit,
while the other part is sent through the identity branch, which leaves its input
unchanged. In the main branch, a sequence of pointwise and $3\times 3$
convolutions are performed. After both the main branch and the identity branch
complete their respective operations, the two parts are concatenated along the
channel dimension and the channels are shuffled. Finally, the output feature
maps are passed to the next ShuffleNet unit in the network. This process is
represented in \cref{fig:sota:channel_split}. This approach balances
computational efficiency with the expressive capacity of the model.\\
% endregion: shufflenet

% region: mobilenetv2
\noindent\textbf{Inverted residual and Linear bottlenecks.} Depthwise Separable
Convolutions were employed in MobileNet \cite{howard2017mobilenets}.
\citeauthor{DBLP:conf/cvpr/SandlerHZZC18} introduced skip connections and
residual blocks into the MobileNetV2 architecture
\cite{DBLP:conf/cvpr/SandlerHZZC18}, initially proposed in
\cite{DBLP:conf/cvpr/HeZRS16}. They also introduced the concept of inverted
residuals and linear bottlenecks. In conventional residual blocks, the input is
first compressed, then expanded, and finally compressed again after being added
to the original input. With inverted residual bottlenecks, on the other hand,
this process is reversed: the input is first expanded, then a depthwise
separable convolution is applied, and finally, it is compressed again. In this
architecture, the skip connections link the feature maps of smaller size,
instead of the larger ones. This allows for a more memory-efficient
architecture. The standard residual blocks and the inverted residual blocks are
shown in \cref{fig:sota:inverted_vs_residual_blocks}. The linear bottlenecks, on
the other hand, are convolutions with a linear activation function. This takes
advantage of the property that high-dimensional feature maps can be embedded in
a lower-dimensional manifold. To do this, it is necessary to use linear
transformations since non-linear ones could potentially destroy information as
reported in \cite{DBLP:conf/cvpr/SandlerHZZC18,DBLP:conf/cvpr/HanKK17}.\\

\begin{figure}[htbp]
  \centering
  \subfloat[Standard Residual Block\label{fig:sota:residual_block}]{%
    \includegraphics[width=0.49\textwidth]{chapter_sota/assets/mobilenet_v2_residual.png}}
  \subfloat[Inverted Residual Block\label{fig:sota:inverted_residual_block}]{%
    \includegraphics[width=0.49\textwidth]{chapter_sota/assets/mobilenet_v2_inverted_residual.pdf}}
  \caption{Illustration scheme of the residual block and the inverted residual
  block. Note that on the inverted residual block, the feature maps with the
  lower number of channels are the ones connected via the skip connection,
  whereas it is the opposite on the standard residual block. Diagonally hatched
  layers do not use non-linearities. The grey colour indicates the beginning of
  the next block. Both illustrations are taken from \cite{DongMobileNetV2}. Best
  viewed in colours.}
  \label{fig:sota:inverted_vs_residual_blocks}
\end{figure}

% endregion: mobilenetv2

% region: mobilenetv3
\noindent\textbf{Squeeze and excitation modules.} Advancing from MobileNet and
MobileNetV2, its third version \cite{DBLP:conf/iccv/HowardPALSCWCTC19}
incorporated \ac{SE} modules initially introduced in
\cite{DBLP:conf/cvpr/HuSS18}. These modules adaptively recalibrate channel-wise
feature responses, amplifying important features and suppressing less relevant
ones. The \ac{SE} module (represented in \cref{fig:sota:se_module}) performs
\emph{squeeze} and \emph{excitation} operations. The squeeze operation uses
global average pooling to create a channel descriptor that summarises the
spatial information for each channel. The excitation operation uses this
descriptor to learn non-linear interactions between channels through two fully
connected layers. The outputs of this mini-network are per-channel modulation
weights that recalibrate the original feature maps, scaling or "exciting" them
by these weights.\\

% endregion: mobilenetv3

\begin{figure}[htbp]
  \centering
  \includegraphics[width=0.70\textwidth]{chapter_sota/assets/SE_module.pdf}
  \caption{Illustration scheme of the \acf{SE} module. The original feature
    map is \emph{squeezed} into a channel descriptor through global average
    pooling. This descriptor is then used to learn the interdependencies between
    the channels through two fully connected layers. The output is then
    multiplied layerwise with the original feature map (\emph{excitation}). Best
    viewed in colours.}
  \label{fig:sota:se_module}
\end{figure}


% region: transition paragraph
The architectures we reviewed in this section revolve around specific key
techniques such as depthwise separable convolutions, fire modules, channel
shuffling, and \ac{SE} modules, among others. These architectures, while highly
efficient, are manually crafted and require a significant degree of human
expertise, intuition, and time to develop, optimise, and fine-tune. The manual
design of these architectures often relies on a deep understanding of the tasks
at hand, the data they will process, and the constraints of the environment in
which they will operate. However, the process of designing these efficient
architectures can be automated, which is the subject of the next section. Sizes
and performance of network architectures detailed in this section can be
compared to standard architecture sizes in \cref{fig:sota:net_sizes_std_eff}.\\

\begin{figure}[htbp]
  \centering
  \includegraphics[width=0.70\textwidth]{chapter_sota/assets/network_sizes_normal_eff.pdf}
  \caption{\Cref{fig:dlo:net_sizes} updated with the size and performance of the
    efficient architectures detailed in \cref{sec:sota:efficient_archi}. Best
    viewed in colours.}
  \label{fig:sota:net_sizes_std_eff}
\end{figure}




% endregion: transition paragraph

\subsection{Automatic Architecture Design Through Neural Architecture Search}\label{sec:sota:nas}

\acf{NAS} is a method that automates the discovery of neural network
architectures, potentially leading to more compact, efficient designs and
reducing the need for manual intervention. Although \ac{NAS} might not
explicitly aim at producing lightweight architectures, it can still yield
designs that strike a good balance between performance and computational cost
\cite{DBLP:conf/cvpr/TanCPVSHL19,DBLP:conf/icml/TanL19}. By using automated
methods to search for optimal architectures, it is possible to further enhance
the efficiency of neural networks, opening up new possibilities for their
deployment in resource-constrained environments. \ac{NAS} has emerged as an
essential paradigm, aiming to automate the traditionally manual and
labour-intensive process of designing efficient neural networks
\cite{DBLP:journals/corr/MiikkulainenLMR17}. Early network architectures were
indeed entirely handcrafted, requiring significant human effort and expertise.
However, these manual methods are being replaced by \ac{NAS} techniques, which
seek to automatically determine the optimal network structure given a training
set \cite{DBLP:journals/corr/abs-2301-08727,elsken2019neural}.\\

The performance and efficiency of \ac{NAS} are fundamentally determined by two
key aspects: the \emph{search space} and the \emph{search strategy}. The search
space, as the name implies, defines the set of all possible architectures that
can be discovered by the \ac{NAS} algorithm. It could be as broad as all
possible configurations of a certain type of network, such as \acp{CNN}, or as
narrow as different arrangements of a specific set of layers
\cite{DBLP:conf/cvpr/LiuCSAHY019}. The search strategy, on the other hand,
determines how the \ac{NAS} algorithm navigates through this search space in
order to optimise its given objective. This could involve gradient-based
strategies \cite{DBLP:conf/iclr/LiuSY19,DBLP:conf/iclr/XuX0CQ0X20}, or
stochastic methods, such as evolutionary algorithms and reinforcement learning
\cite{DBLP:conf/iclr/ZophL17,DBLP:conf/icml/RealMSSSTLK17}. The choice of search
space and search strategy significantly influences the ability of \ac{NAS} to
discover effective and efficient architectures and is thus a critical aspect of
NAS research. In the following paragraphs, we will delve deeper into some of the
major strategies and their impact on the field of \ac{NAS}.\\



\noindent\textbf{Search space.} The search space is a critical aspect of
\ac{NAS} as it bounds the possibilities of architectures and significantly
influences the outcome of the search. The search space could be as broad as all
possible configurations of a certain network type or as specific as various
arrangements of a predefined set of layers or blocks. For instance,
\cite{DBLP:conf/iclr/ZophL17} define their search space as a set of repeatable
sub-structures composed of basic layers (convolution layers, fully connected
layers, \ac{BN} layers, etc...) often called \emph{cells} that are stacked to
form the final architecture, while \cite{DBLP:conf/iclr/XieZLL19} design their
search space based on the connectivity patterns between network blocks. DARTS
\cite{DBLP:conf/iclr/LiuSY19} propose a continuous search space where the
architecture is parameterized as a differentiable function, allowing for
efficient search using gradient-based methods. Hierarchical search spaces, on
the other hand, offer a strategic approach which allows to manage the complexity
of the architecture search in \ac{NAS}
\cite{DBLP:conf/cvpr/LiuCSAHY019,DBLP:conf/cvpr/TanCPVSHL19}. In such a setup,
the architecture is divided into several levels of hierarchy, with each one
searched independently. This structure enables a more systematic and organized
exploration of the search space, allowing the algorithm to uncover useful
patterns and configurations at different levels of the network. The EfficientNet
models are exemplary of innovative architecture search strategies
\cite{DBLP:conf/icml/TanL19}. This series utilizes both \ac{NAS} and
\emph{compound scaling}. A baseline, EfficientNet-B0, was developed through
multi-objective \ac{NAS}, optimizing both accuracy and \acp{FLOP}. Subsequently,
a compound scaling method was applied to this baseline, uniformly scaling depth,
width, and resolution via a \emph{compound coefficient}. This approach yielded a
series of progressively larger EfficientNet models, whose performances are shown
in \ref{fig:sota:efficientnet_perfs}.\\\


\begin{figure}[htbp]
  \centering
  \includegraphics[width=0.70\textwidth]{chapter_sota/assets/efficientnet_perfs_overview.pdf}
  \caption{ImageNet top-1 accuracy vs model size (in millions of parameters).
    The EfficientNet family of models significantly outperforms other models of
    similar size, obtained either by \ac{NAS} or manual design. This graph is
    taken from \cite{DBLP:conf/icml/TanL19}.
  }
  \label{fig:sota:efficientnet_perfs}
\end{figure}


\noindent\textbf{Search strategy.} The search strategy is another major
component of \ac{NAS}, dictating how the algorithm explores the search space to
find the optimal architecture. A wide range of search strategies have been
proposed. Evolutionary algorithms \cite{DBLP:conf/icml/RealMSSSTLK17} use
principles of natural evolution such as mutation, crossover, and selection to
explore the search space. Despite their potential to find high-quality
solutions, these methods often require substantial computational resources due
to the large number of evaluations needed. Reinforcement Learning-based methods
\cite{DBLP:conf/iclr/ZophL17} employ a policy network to generate architectures
and a reward signal, typically validation accuracy, to guide the search. While
reinforcement learning methods can effectively navigate large search spaces,
their success heavily depends on the quality of the reward signal.
Gradient-based methods \cite{DBLP:conf/iclr/LiuSY19,DBLP:conf/iclr/XuX0CQ0X20}
make the search space continuous and use gradient descent for optimization,
which enables efficient exploration of the search space but requires careful
regularization to prevent overfitting. \cite{DBLP:conf/nips/BergstraBBK11} uses
Bayesian optimization to build a probabilistic model of the objective function
and uses it to select promising architectures, balancing exploitation and
exploration. This method can be sample-efficient but might struggle with
high-dimensional spaces. These diverse strategies offer multiple paths to
navigate the complex landscape of architecture search, each with its unique
trade-offs between efficiency, effectiveness, and computational demands.\\


\begin{figure}[htbp]
  \centering
  \includegraphics[width=0.70\textwidth]{chapter_sota/assets/network_sizes_normal_eff_nas.pdf}
  \caption{\Cref{fig:sota:net_sizes_std_eff} updated with the size and
    performance of architectures detailed in \cref{sec:sota:nas}. Best viewed in
    colours.}
  \label{fig:sota:net_sizes_std_eff_nas}
\end{figure}

\section{Compressing and Optimising an Existing Architecture}\label{sec:sota:refining_existing}

While the prior sections have primarily focused on constructing new, efficient
network architectures and mechanisms for automatic architecture discovery, this
part of the chapter transitions towards compressing and optimising existing neural
networks. The methods discussed in this section, namely \emph{quantisation},
\emph{binarisation} and \emph{pruning}, are specifically geared towards
leveraging and enhancing already existing architectures or trained models.
Instead of developing a new architecture, these techniques seek to make an
existing architecture more efficient by modifying its weights and connections.\\

\Cref{sec:sota:quantisation} delves into the methodologies of quantisation and
binarisation. These methods endeavour to reduce the numerical precision of
weights and activations in a network, without a significant drop in overall
performance. This process can significantly speed up computations and decrease
memory usage, contributing to the increased efficiency of a pre-existing
network, especially in environments with limited hardware or memory resources.\\

Subsequently, \cref{sec:sota:pruning} examines the application of
pruning techniques. Pruning refers to the elimination of redundant or
insignificant weights and connections in a network, leading to a sparser and
more effective architecture. Pruning an existing network can further enhance
efficiency by reducing the computational resources needed with a minimal or
controlled impact on the performance.\\

Through these methods, this section aims to demonstrate how the effectiveness of
existing neural networks can be optimised, thereby offering another viable path
towards generating more efficient models without creating new architectures from
scratch.\\

\subsection{Lower Precision Weights and Activations Representation}\label{sec:sota:quantisation}

Quantization is the process of converting continuous, high-resolution input
values into a lower-resolution and typically discrete representation.
Historically, the training of neural networks has largely relied on the use of
\ac{FP32}. FP32 has been the default choice due to its wide support across
various hardware platforms and software libraries, which has made it a practical
and convenient choice for the majority of machine learning tasks
\cite{sze2017efficient}. However, using \acl{FP32} is not always necessary, and
it is possible to constrain neural networks to use lower precision values,
effectively quantising its parameters or feature maps, while maintaining
compelling performances. Quantising a neural network can result in a reduced
memory footprint as well as faster computation if the operations are implemented
to leverage the specificity of the quantisation or paired with appropriate
hardware.\\

Quantising a neural network has been proposed as early as the 1990s
\cite{balzer1991weight,fiesler1990weight}. This later regains traction as
\citeauthor{37631} leveraged Single Input Multiple Data instructions (SIMD) of
x86 \ac{CPU} to speed up the fixed-point 8-bit operations \cite{37631}.
\citeauthor{gupta2015deep} \cite{gupta2015deep} used uniform quantisation with
fixed-point 16-bit representation and stochastic rounding to train neural
networks. Quantisation has also been applied together with K-means clustering
\cite{steinhaus1956division}. \cite{DBLP:journals/corr/HanMD15} uses K-means
clustering to iteratively compute a lookup table or \emph{code book} for the
weights. This codebook is later further compressed using Huffman coding
\cite{huffman1952method}. Note that this method is mostly useful for storage,
but for training or inference, the weights need to be decompressed and their
original value fetched in the code book before being used.\\

\noindent\textbf{Logarithmic quantisation.} Logarithmic quantisation provides
compelling alternatives to uniform quantisation. On the one hand, logarithmic
quantisation enables quantising weights with a larger dynamic range compared to
uniform or linear quantisation. On the other hand, multiplication can be
conveniently represented as an inexpensive bit shift operation if operands are
properly represented in the logarithmic base. This is particularly beneficial
for \ac{FPGA} implementations \cite{alemdar2017ternary}. To leverage this
potential speedup, \cite{DBLP:journals/corr/LinCMB15} forced the weight
representation to be a power of two and \cite{DBLP:conf/iclr/ZhouYGXC17}
improved this technique by applying it iteratively.\\

\begin{figure}[htbp]
  \centering
  \includegraphics[width=0.40\textwidth,trim=5cm 9cm 5cm 9cm, clip]{chapter_sota/assets/binarised_kernels.pdf}
  \caption{Example of binarised kernels and activations in a convolutional
    layer. The kernels are taken from the first layer of a \ac{CNN} trained on CIFAR-10.
    Image taken from \cite{DBLP:conf/nips/HubaraCSEB16}.}
  \label{fig:sota:binarised_kernels}
\end{figure}

\noindent\textbf{Binarisation.} A more extreme version of the quantisation has
been proposed in \cite{courbariaux2015binaryconnect}, where the weight values
are either $-1$ or $+1$. The concept of minimising the bit-width of weights to a
bare minimum is called \emph{binarisation}. This allows for a dramatic
simplification of the computation in the neural network at the expense of a drop
in performance. Binarisation has been further developed in
\cite{DBLP:conf/nips/HubaraCSEB16}, where the authors proposed a method to
binarise both weights and activations (see examples of binarised kernels in
\cref{fig:sota:binarised_kernels}). DoReFa-Net \cite{zhou2016dorefa} built upon
the success of binarised neural networks and introduced the stochastic 8-bit
quantisation of the gradients during the backward pass to accelerate both
training and inference.\\

\noindent\textbf{When to quantise.} Quantisation methods that quantise weights
or activations after the training are called \ac{PTQ} methods. Quantising an
already existing network is a widely used technique in the most famous deep
learning framework \cite{ncnn,qnnpack,snapdragon,tensorrt}. Because they
quantise the weights after the training, \ac{PTQ} methods are fast and easy to
apply. However, they often introduce an irreversible information loss and a
performance drop that needs to be compensated for
\cite{DBLP:journals/ijon/LiangGWSZ21}. In order to solve this issue, several
works proposed to take into account the effect of quantisation on the weights
and feature maps during the training. These methods are called \ac{QAT} methods.
Such methods include BinaryConnect \cite{courbariaux2015binaryconnect}, which
use a variant of Bayesian inference called Expectation Back Propagation
\cite{DBLP:journals/corr/ChengSML15,DBLP:conf/nips/SoudryHM14}. Another
binarisation method uses \ac{STE} \cite{DBLP:journals/corr/BengioLC13} to bypass
the binarisation function in the backward pass
\cite{DBLP:conf/nips/HubaraCSEB16}. \ac{STE} is also employed for quantisation
in \cite{DBLP:conf/cvpr/JacobKCZTHAK18} which uses it together with \emph{fake
quantisation} nodes for 8-bit quantisation (see \cref{fig:sota:fake_quant}). The
fake quantisation nodes are injected inside the computation graph and simulate
the effect of quantisation in the forward pass.\\

Quantisation and binarisation are solutions to compress and accelerate neural
networks. The potential of these techniques is vast, as they offer significant
reductions in memory usage and enhanced computational speed when implemented
correctly and paired with suitable hardware. However, these benefits are not
without their drawbacks. On the downside, such techniques introduce a certain
degree of error which can result in a performance drop, especially if not
properly managed during the training process. This information loss is
particularly notable in the case of \acl{PTQ} methods, which necessitate
additional efforts to mitigate these performance impacts. To address this,
\ac{QAT} methods have been developed, which incorporate the effects of
quantisation during the training phase itself. The more extreme approach,
binarisation, further accentuates the advantages and disadvantages observed in
quantisation. While it offers extreme compression of neural networks, this often
comes at the cost of significant accuracy loss.\\


\begin{figure}[htbp]
  \centering
  \subfloat[Fake quantisation inference\label{fig:sota:fake_quant_inference}]{%
    \includegraphics[height=6cm]{chapter_sota/assets/fake_quant_inference.pdf}}
  \subfloat[fake quantisation training\label{fig:sota:fake_quant_training}]{%
    \includegraphics[height=6cm]{chapter_sota/assets/fake_quant_training.pdf}}
  \caption{Fake quantisation nodes (\emph{fake quant.}) are included in the
    computation graph of \cref{fig:sota:fake_quant_training}, whereas
    \cref{fig:sota:fake_quant_inference} represent the computaion graph used during
    inference. During the inference, weights are stored in \texttt{uint8} format,
    whereas the bias are not, because their computational overhead is
    negligible.\cite{DBLP:conf/cvpr/JacobKCZTHAK18}. Both illustrations are adapted
    from \cite{DBLP:conf/cvpr/JacobKCZTHAK18}.}
  \label{fig:sota:fake_quant}
\end{figure}

\subsection{Removing Weights and Connections}\label{sec:sota:pruning}

Lightweight neural networks can be obtained from a larger network through
pruning. Pruning is the process of removing weights or connections, identified
as redundant or unnecessary, while limiting to a minimum the impact on the
performance of the network. The identification of the latter, often referred to
as determining the \emph{saliency} of weights, has been a hot spot in the
pruning literature \cite{li2023model,cheng2017survey,liang2021pruning}. Pruning
a neural network removes weights and consequently reduces the theoretical
computational complexity of the network as well as its memory footprint.  The
fraction of weight removed during pruning is often denoted as the \emph{pruning
rate}, which is commonly defined as the fraction of the number of weights
removed from the original network over the number of initial weights in that
network. Arguably, the first pruning method, introduced in 1989, was based on
biased weight decay \cite{hanson1988comparing}. In the following years,
\citeauthor{DBLP:conf/nips/CunDS89} proposed a pruning method entitled Optimal
Brain Damage \cite{DBLP:conf/nips/CunDS89} that used the Taylor expansion of the
loss hessian matrix to identify the weights whose removal would have the least
impact on the loss. This method, and in particular the computation of the
hessian matrix were refined in Optimal Brain Surgeon
\cite{DBLP:conf/nips/HassibiS92,DBLP:conf/nips/HassibiSW93,DBLP:conf/icnn/HassibiSW93}.
As neural networks have become larger and more computationally intensive (see
\cref{sec:dlo:architectures,fig:dlo:net_sizes}), pruning has been receiving
increased attention as a method to compress the latter. Pruning methods can be
classified into two categories: \emph{structured} and \emph{unstructured}.\\


\begin{figure}
  \centering
  \subfloat[Structured pruning\label{fig:sota:structured_pruning}]{%
    \includegraphics[width=0.6\textwidth]{chapter_sota/assets/structured_pruning.pdf}}
  \hspace{0.09\textwidth}
  \subfloat[Unstructured pruning\label{fig:sota:unstructured_pruning}]{%
    \includegraphics[width=0.3\textwidth]{chapter_sota/assets/unstructured_pruning.pdf}}
  \caption{Conceptual illustrations of structured and unstructured pruning.}
  \label{fig:sota:pruning}
\end{figure}

\noindent\textbf{Structured pruning.} Structured pruning involves the removal of
entire structural components of the network, such as rows, columns, channels,
filters, layers or even whole subnetworks (note that pruning a channel in layer
$\ell+1$ implies pruning a filter in layer $\ell$, and vice-versa). This type of
pruning results in regular\footnote{regular in this context means that all
weight tensors are dense and that the acceleration of computations does not rely
on sparse computing enhancement.} network structures that are easier to exploit
on typical hardware and do not necessitate a specific sparse computing library
or hardware, making it an attractive approach for practical deployment. To some
extent, structured pruning can be seen as a subcategory of \acl{NAS}
(\cref{sec:sota:nas}), where the search space would be the structure of the
network to be pruned. Structured pruning leads to reduced computation complexity
as well as significant memory footprint reduction, however, it also presents
unique challenges. The impact of removing structural components can be much
greater than eliminating individual weights, hence, structured pruning often
requires more careful consideration of the trade-off between the model
performance and complexity reduction. Since structured pruning operates on a
coarser scale, it typically results in lower pruning rates compared to
unstructured pruning.\\


\noindent\textbf{Weight-dependant structured pruning.} One of the main
categories of structured pruning for \ac{CNN} is weight-dependent pruning. This
strategy assesses the importance of filters based on their respective weights.
The Pruning Filter for Efficient ConvNets method
\cite{DBLP:conf/iclr/0022KDSG17} focuses on the filter norms as their saliency
indicator. Filters with smaller $\ell_1$ norms, which result in weak
activations, are assumed to contribute less to the final classification
decision, hence they become the prime candidates for pruning.  The Filter
Pruning via Geometric Median method \cite{DBLP:conf/cvpr/HeLWHY19} calculates
the geometric median of a set of filters and prunes those filters that are
nearest to this geometric median, rather than the ones considered less important
by \cite{DBLP:conf/iclr/0022KDSG17}. The filters close to the geometric median
are considered by \citeauthor{DBLP:conf/cvpr/HeLWHY19} to be redundant with
other filters in the same layers. \citeauthor{DBLP:conf/cvpr/WangLW21} used
another approach to determine redundancy in \cite{DBLP:conf/cvpr/WangLW21}: The
filters are organized into a graph based on their proximity in the space in
which they are defined. A redundancy metric is computed for each graph and the
least important filters are pruned in the graph with the highest redundancy (as
per the authors, any other method for filter importance evaluation can be used
\cite{DBLP:conf/iclr/0022KDSG17,DBLP:journals/access/PolyakW15,DBLP:conf/iclr/MolchanovTKAK17}).
This process is iteratively applied until the targeted pruning rate is reached.
These weight-dependent strategies tend to be straightforward and usually demand
lower computational costs compared to other methods
\cite{DBLP:journals/corr/abs-2303-00566}. They provide an intuitive
understanding of how different filters contribute to the overall network
performance based on their weight characteristics. \\


\begin{figure}[htbp]
  \centering
  \includegraphics[width=0.8\textwidth]{chapter_sota/assets/thinet.pdf}
  \caption{Illustration Scheme of ThiNet. The dotter filters and corresponding
    channels are the ones to be pruned. Once they are removed, the pruned network
    is fine-tuned. Image taken from \cite{DBLP:conf/iccv/LuoWL17}}
  \label{fig:sota:thinet}
\end{figure}

\noindent\textbf{Activation-based structured pruning.} Another prominent
category of structured pruning is activation-based pruning, where
\emph{activation} denotes the result yielded by a layer for given input data.
This method takes advantage of activation maps (also called feature maps) for
filter pruning. Removing a channel in a feature map is equivalent to removing
the filter that computed this channel. \citeauthor{DBLP:conf/iccv/HeZS17}
proposed a method to prune filters based on a LASSO regression selection while
minimising the least square reconstruction error or the feature map
\cite{DBLP:conf/iccv/HeZS17}. \citeauthor{DBLP:journals/corr/HuPTT16}
capitalised on the abundance of zeros in feature maps that follow the \ac{ReLU}
activation function. They introduced the method called Average Percentage of
Zeros (APoZ) that identifies channels in the feature map with a high count of
null outputs. These channels, which contribute minimally to the final outcome,
can hence be pruned. While the aforementioned methods consider only the feature
map of the layer to be pruned, techniques like ThiNet
\cite{DBLP:conf/iccv/LuoWL17} and Approximated Oracle Filter Pruning
\cite{DBLP:conf/icml/DingDGHY19} exploit the relationships between layers to
guide pruning. They take into consideration the effect a filter removal in one
layer has on the next, allowing for more contextual pruning decisions. More
globally, approaches such as Neuron Importance Score Propagation
\cite{DBLP:conf/cvpr/Yu00LMHGLD18} and Discrimination-aware Channel Pruning
\cite{DBLP:conf/nips/ZhuangTZLGWHZ18} consider the holistic effect of removing a
filter. They aim to understand and quantify the total impact on network
performance when a specific filter is removed, accounting for cascading effects
across all layers.\\

\noindent \textbf{Regularisation-based structured pruning.} Other methods learn
structured sparse networks by introducing various sparsity regularizers. Some
methods focus on \acl{BN} parameters, employing methods like Gated Batch
Normalisation \cite{DBLP:conf/nips/YouYYM019} and Network Slimming
\cite{DBLP:conf/iccv/LiuLSHYZ17}. These methods aim to push certain BN
parameters towards zero, effectively disabling corresponding channels and
inducing sparsity. Network Slimming applies a $\ell_1$ regularisation on the
scaling factors $\gamma$ of the \ac{BN} \cite{DBLP:conf/iccv/LiuLSHYZ17},
whereas \cite{DBLP:conf/nips/YouYYM019} adds the $\ell_1$ regularisation on
scalar factors associated with feature map channels.
\citeauthor{DBLP:conf/icml/KangH20} use \ac{BN} parameters to craft a mask that
prunes channels whose output is likely to be null once evaluated by the
\ac{ReLU} \cite{DBLP:conf/icml/KangH20}. \\

\begin{figure}[htbp]
  \centering
  \includegraphics[width=0.8\textwidth,trim=0 0.4cm 0 0, clip]{chapter_sota/assets/operation_aware.pdf}
  \caption{Comparison of the method described in \cite{DBLP:conf/icml/KangH20}
    (right) and standard channel pruning (left). The differentiable mask allows
    for a soft pruning that can be reverted during the training. Image taken
    from \cite{DBLP:conf/icml/KangH20}}
  \label{fig:sota:operation_aware}
\end{figure}


\noindent \textbf{Taylor Expansion-based structured pruning.} The Taylor
Expansion is a tool that can be used to approximate the change in the loss
function due to pruning. Early unstructured pruning methods, Optimal Brain
Damage \cite{DBLP:conf/nips/CunDS89} and Optimal Brain Surgeon
\cite{DBLP:conf/icnn/HassibiSW93} used Taylor expansion of the hessian matrix to
remove weights with the least impact on the loss function. More recently, in
\cite{DBLP:conf/iclr/MolchanovTKAK17},
\citeauthor{DBLP:conf/iclr/MolchanovTKAK17} used first-order Taylor expansion of
the loss function to compute the importance score and prune the feature maps.
This was later refined in \cite{DBLP:conf/cvpr/MolchanovMTFK19} where the
authors computed the importance score on the weight, rather than the feature
maps, to lower memory consumption.\\

\noindent\textbf{Variational structured pruning.} Variational Bayesian methods
provide a way to tackle the computationally intensive process of inferring
posterior probability distributions in large data sets, by approximating the
posterior distribution with a variational distribution \cite{fox2012tutorial}.
Specific methods like Variational Pruning \cite{DBLP:conf/cvpr/ZhaoNZZZT19} and
Recursive Bayesian Pruning \cite{DBLP:conf/iccv/Zhou0W019} use this approach to
create more effective and stable pruning mechanisms for neural networks.
Variational Pruning, models channel importance as random variables, utilizing
the centrality property of the Gaussian distribution to induce sparsity
\cite{DBLP:conf/cvpr/ZhaoNZZZT19}, while \cite{DBLP:conf/iccv/Zhou0W019} targets
the posterior of redundancy, assuming inter-layer dependency among channels.\\


\noindent\textbf{Dynamic structured pruning.} Dynamic pruning represents a
different approach in which neural networks are pruned during both training and
inference, which facilitates maintaining the model representation capability and
offers superior resource consumption-accuracy trade-offs. In the training phase,
Dynamic Network Surgery \cite{DBLP:conf/nips/GuoYC16} introduces the concept of
dynamic pruning through an unstructured approach. This method applies a binary
mask indicating the importance of weights and updates it alternately with all
the weights, allowing incorrectly pruned parameters an opportunity to revive.
Soft Filter Pruning \cite{DBLP:conf/ijcai/HeKDFY18} implements a structured
version of dynamic pruning. Instead of employing a fixed mask throughout the
training, which could limit the optimization space, Soft Filter Pruning
dynamically generates masks based on the $\ell_2$ norm of filters at every
epoch. The dynamic nature of Soft Filter Pruning allows soft-pruned filters to
be updated in the next epoch, with new masks being formed based on the updated
weights. Focusing on the inference stage, \citeauthor{DBLP:conf/nips/LinRLZ17}
introduced Runtime Neural Pruning \cite{DBLP:conf/nips/LinRLZ17} which employs a
unique framework consisting of a \ac{CNN} backbone and a recurrent neural
network as a decision network. This approach enables the model to adapt to the
properties of different input images by dynamically adjusting its complexity.
For easier tasks or simpler images, the network can become sparser, saving
computational resources. Deep Reinforcement Learning pruning
\cite{DBLP:conf/nips/ChenCP20} learns both the static and dynamic importance of
channels. The static importance refers to the channel's relevance for the entire
dataset, while the dynamic importance is tied to a specific input. Deep
Reinforcement Learning pruning \cite{DBLP:conf/nips/ChenCP20} applies
reinforcement learning to generate a unified pruning decision based on these two
aspects of channel importance. More recently,
\citeauthor{DBLP:conf/cvpr/ElkerdawyE0R22} recently introduced Fire Together
Wire Together \cite{DBLP:conf/cvpr/ElkerdawyE0R22}, another dynamic pruning
method that treats pruning as a self-supervised binary classification problem.
It employs a prediction head to train learnable binary masks and uses a crafted
ground truth mask to guide the learning after each convolutional layer. This
head takes advantage of the \ac{ReLU} activation function, which zeros out
negative values, to identify the filters that will yield zero activations based
on the input and that will be subsequently pruned.\\

As previously discussed, structured pruning removes indivisible groups of
weights and therefore yields regular network architectures that can be
implemented in standard deep learning frameworks in a straightforward way.
Despite its practicality, structured pruning enforces a strong topological prior
by pruning entire groups of weights from the original network, which
consequently leads to a lower sparsity rate compared to its counterpart,
unstructured pruning. Unstructured pruning provides a more flexible approach by
removing individual weights from the original network structure. This process
not only offers greater adaptability compared to structured pruning, but also
results in higher pruning rates.\\

\noindent\textbf{Unstructured pruning.} In the early stages, unstructured
pruning methodologies identified weights that could be eliminated based on their
influence on the Hessian of the loss function
\cite{DBLP:conf/nips/CunDS89,DBLP:conf/icnn/HassibiSW93,DBLP:conf/nips/HassibiSW93}.
A simpler and more tractable strategy for larger networks was later introduced
by \citeauthor{DBLP:conf/nips/HanPTD15} in \cite{DBLP:conf/nips/HanPTD15},
suggesting a straightforward heuristic: the pruning of weights with the smallest
magnitude (\emph{i.e.} absolute value), also referred to as \emph{magnitude
pruning}. This technique presented in \cite{DBLP:conf/nips/HanPTD15} devises an
iterative method wherein a portion of the smallest magnitude weights are
removed, followed by the retraining of the network to compensate for the drop of
the accuracy. This cycle is reiterated until the preferred level of sparsity is
attained. Furthermore, magnitude pruning has been used together with
quantisation and compression techniques to minimise the storage footprint of a
network \cite{DBLP:journals/corr/HanMD15}. Magnitude pruning has also been used
in energy-efficient \ac{CNN} design, as detailed in
\cite{DBLP:conf/cvpr/YangCS17}. In this research, the energy consumption of each
layer is evaluated, and the layers with the highest energy expenditure are
pruned using unstructured magnitude pruning. This layer is then fine-tuned to
retain the network accuracy. This process is repeated iteratively until a
noticeable drop in accuracy is observed. Dynamic Network Surgery
\cite{DBLP:conf/nips/GuoYC16} puts forward a derivative of magnitude pruning. A
mask, whose value is updated during training, is computed for each weight. This
mask is used to stochastically prune a weight or splice\footnote{to splice is
the verb used by the authors (\citeauthor{DBLP:conf/nips/GuoYC16}) to denote the
reactivation of a weight that was previously pruned} it. The saliency of the
weights is ascertained based on the magnitude of the associated mask.\\

\noindent\textbf{Effective subnetworks.} More recently, unstructured pruning
researches focus on the discovery of small sub-networks inside the original
network. In other words, to identify a subset of weight that can perform, under
certain conditions or assumptions, as well as the original network. Most
notably, the Lottery Ticket Hypothesis \cite{DBLP:conf/iclr/FrankleC19} states
that within a large, randomly initialized neural network, there exist
sub-networks or \emph{Lottery Tickets} that are capable of training effectively
when isolated from the rest of the weights. These smaller networks, found by
pruning the smallest magnitude weights from the trained original network, are
observed to train faster and achieve comparable or even superior performance
with respect to the original network. The \emph{lottery ticket} is found by
training the original network up to convergence, then pruned with magnitude
pruning and finally, the remaining weights are reinitialized to their original
values, it is to say the value they had before the training even started. The
resulting sub-network is the \emph{lottery ticket}. This research sparked
significant interest and various works: \cite{DBLP:conf/nips/ZhouLLY19} proposed
an analysis of the results presented in \cite{DBLP:conf/iclr/FrankleC19}. These
results \cite{DBLP:conf/iclr/FrankleC19} have been extended to larger networks
in \cite{DBLP:journals/corr/abs-1903-01611}, the necessity of training the
original network to convergence to find the Lottery Tickets has been challenged
in \cite{DBLP:conf/iclr/LiuSZHD19}. The existence of the \emph{lottery ticket}
(or in other words the subnetwork) has been theoretically proven in
\cite{DBLP:conf/icml/MalachYSS20} and the requirements on the theoretical size
of the original network have been later refined in
\cite{DBLP:conf/nips/PensiaRNVP20,DBLP:conf/nips/OrseauHR20}.\\

\section{Positioning}

Within the various deep neural network compression methods presented in this
chapter, we choose to focus specifically on pruning in the context of supervised
image classification. Among various types of pruning, our interest goes towards
structured pruning due to the flexibility it offers, in particular, its
potential for achieving high pruning rates compared to unstructured pruning.\\

\noindent Our decision to work on pruning is rooted in the following
considerations:
\begin{itemize}
  \item First, pruning allows the creation of lightweight networks while
  preserving or sometimes improving the performance of the original network.

  \item Then, pruning integrates well with other compression techniques and can
  be applied in conjunction with them, on any kind of architecture.

  \item  Finally, pruning does not necessitate the creation and development of
  an architecture from scratch. It can be applied to an already existing
  architecture to compress it, which makes it possible to develop small,
  lightweight networks without the need for extensive research into the creation
  of the base architecture. This approach allows freeing exploration and
  research and development efforts that can be allocated to other topics.

\end{itemize}

Despite its numerous advantages, pruning is not without challenges. One such
challenge is the identification of the weights to be preserved. This is a topic
that is the subject of many works detailed in this chapter. Additionally, the
preserved weights typically require fine-tuning, which can impose a substantial
computational cost. This process of fine-tuning can be both time-consuming and
resource-intensive.\\

\Cref{chap:chapter1} delves into the fine-tuning issue and presents a new
pruning method that circumvents the need for the expensive fine-tuning step. Its
budget loss together with the weight reparametrisation allows for joint
optimisation of the topology and the weights of the network without the need for
auxiliary variables. As a result, the obtained lightweight networks preserve
accuracy after effective pruning and do not require fine-tuning. Furthermore,
\cref{chap:chapter2} tackles the challenge of identifying relevant weights
without even having to train them. It proposes a method of topology selection
given a set of untrained weights that achieves compelling performances, thereby
also sidestepping the fine-tuning. In contrast to other methods, the optimal
pruning rate is discovered in one shot by our pruning strategy that circumvents
the costly gird search for its value. This innovative approach opens up new
possibilities for further reducing the computational costs associated with
pruning and provides a new direction for future research in this area, as well
as neural network training without weight tuning in general.\\


\section{Conclusion}


The evolution of neural networks, along with the growing demand for their
deployment in resource-constrained environments, has underlined the need for
neural network compression techniques. This chapter has examined the
state-of-the-art methodologies for reducing the computational demands and memory
footprints of deep neural networks, thereby facilitating their usage in a
variety of application domains.\\

First, we explored \cref{chap:dlo} the historical progression and the major
architectures of deep neural networks, illustrating the connection between their
complexity and performance. Then, in this chapter, we first investigated the
techniques for accelerating computations within neural networks, emphasising the
role of fast convolution techniques in reducing both runtime and computational
resources. Our focus then shifted to \acl{KD}, a paradigm that allows the
transfer of knowledge from a large, complex network to a smaller, more efficient
one. The core idea is to teach a lightweight student network to mimic the
behaviour of a teacher network, thus achieving comparable performance with a
reduced footprint. Next, we delved into efficient architecture design methods,
including bespoke architectures designed to minimise size while maintaining
performance, and \acl{NAS} strategies for automating the discovery of optimal
architectures. Lastly, we addressed the strategies for compressing and
optimising existing neural networks, considering both quantisation and
binarisation techniques that lower the numerical precision of weights and
activations, as well as pruning techniques that remove redundant or
insignificant weights and connections, resulting in sparser and more
computationally efficient models. In conclusion, these techniques provide a
multi-faceted approach to neural network compression and acceleration, with each
offering unique advantages and trade-offs.\\

The next chapters will present approaches to neural network compression based on
pruning. The \cref{chap:chapter1} details our first contributions that consist
in a method to simultaneously train and prune neural networks while matching a
budget. This method allows bypassing the need for the finetuning step present in
most methods based on magnitude pruning by jointly optimising the topology and
the weights without the need for additional auxiliary parameters. 
\chapter{Weight Reparametrization}

\begin{abstract}
    abstract of this part
\end{abstract}

\section{Introduction And Related Work}

The introduction of neural networks has revolutionised the field of machine
learning, leading to breakthroughs in various areas such as image and speech
recognition, natural language processing, and game playing. However, the size of
neural networks has steadily increased in recent years, largely thanks to the
availability of powerful \ac{GPUs}. They have made it possible to train larger
and more complex models. However, as the size of neural networks has grown, so
have the computational and memory requirements to train and deploy them. The
evolution of neural network architectures can be traced back to the Rosenblatt
Perceptron \cite{rosenblatt1958perceptron}, a single-layer feedforward network
with only one neuron. Over time, neural network architectures became more
complex and began to include multiple layers, known as multilayer perceptron, or
fully connected networks. Then, with the introduction of \ac{CNNs}, neural
network architectures for image recognition have grown even larger.
Convolutional neural networks use convolutional layers to automatically and
adaptively learn spatial hierarchies of features from input images. This allows
\ac{CNNs} to effectively learn and classify images with high accuracy. Some of
the most notable \ac{CNN} architectures include AlexNet
\cite{DBLP:conf/nips/KrizhevskySH12}, which was developed in 2012 and has 60
million parameters. The VGG networks \cite{DBLP:journals/corr/SimonyanZ14a},
developed in 2014, are ranging from 132 to 143 million parameters. Inception
\cite{DBLP:conf/cvpr/SzegedyLJSRAEVR15}, developed in 2014, had 27 million
parameters in its third version. And the ResNet networks
\cite{DBLP:conf/cvpr/HeZRS16}, developed in 2015, are ranging from 11 to 60
million parameters. \\


The need for neural networks in embedded applications has grown in recent years,
with examples such as object detection in self-driving cars, image and speech
recognition in mobile devices, and natural language processing in smart
speakers. These applications require real-time processing and low power
consumption, which are impossible with large neural networks. Pruning techniques
can reduce the size of neural networks, making them more suitable for deployment
on embedded devices while still maintaining or even improving their performance.
Methods such as structured and unstructured weight pruning can reduce the number
of parameters and FLOPS, consequently reducing the network's size, memory and
power consumption.\\


Pruning is an excellent way to obtain lightweight neural networks because it
reduces the number of parameters in a pre-trained network without the need to
design a new architecture from the ground up. Instead of starting from scratch,
pruning techniques can be applied to existing architectures, which have been
trained and tested on large-scale datasets. It aims at reducing the number of
parameters in a network by removing redundant or unnecessary weights. Pruning
methods can be split into two major categories: unstructured weight pruning,
where individual weights of the network are removed based on their importance.
And structured pruning, where entire columns, rows, channels, filters or even
subnetworks are removed. \\


The first methods to prune shallow networks were proposed in the late 1980s.
Techniques from that area include removing the smallest connection
\cite{janowsky1989pruning}, introducing a weight scaling factor and the study of
its impact on the loss function \cite{DBLP:conf/nips/MozerS88} or the study of
the sensitivity of the weights based on the gradients
\cite{DBLP:journals/tnn/Karnin90}. Most influential papers of the early pruning
days are Optimal Brain Damage \cite{DBLP:conf/nips/CunDS89} and Optimal Brain
Surgeon
\cite{DBLP:conf/nips/HassibiS92,DBLP:conf/nips/HassibiSW93,DBLP:conf/icnn/HassibiSW93}.
The former work focuses on pruning weights based on their impact on the loss,
approximated by its Taylor series, which requires the computation of the hessian
matrix of the loss. However, computing the hessian matrix is Intractable in
practice due to the large number of parameters in the neural networks.
Therefore, the authors introduced a few simplifying assumptions, most notably
the diagonal assumption for the hessian matrix: loss perturbations following
weight pruning are assumed to be weight independent. \\


More recently, pruning regained traction with the work of Han et al.
\cite{DBLP:conf/nips/HanPTD15}.  The authors introduced a three-step pruning
method where first, the weights are tuned. Then all the weights whose absolute
value is below a certain threshold are removed. Finally, the remaining weights
are finetuned. \\


Following this work, research efforts stirred toward structured pruning.
Structured pruning removes groups of weights. The substructure of this group can
be a simple row or column in a filter, a channel of a filter, the filter itself
or even entire subnetworks. Structured pruning is not sparsifying weight
tensors, but rather reshaping the network to remove unnecessary parts of it that
are costly to evaluate and do not bring much performance improvement regarding
the considered task. Since the remaining weight tensors are not sparse, speedups
can be achieved with conventional libraries and hardware. In this context, Anwar
et al. \cite{anwar2017structured} proposed a pruning technique on various levels
(channels, kernels and intra-kernel levels) while Li et al.
\cite{DBLP:conf/iclr/0022KDSG17} proposed a pruning method at a larger (filter)
level. Network slimming \cite{DBLP:conf/iccv/LiuLSHYZ17} is a streamlined
approach that aims at pruning the most useless channels in the layers preceding
\ac{batch norm} \cite{DBLP:conf/icml/IoffeS15}. It induces sparsity with
$\ell_1$ penalization of the \ac{batch norm} scaling factors, each one
associated with a channel. Then channels are removed based on the relative
importance of their associated scaling factor, up to a predefined sparsity
ratio. More recent work proposed an automatic policy for pruning, such as
\ac{amc} \cite{DBLP:conf/eccv/HeLLWLH18} which relies on reinforcement learning
with two interacting agents; the first one iterates over the layers of the
architecture and defines a targeted sparsity, and the second agent implements
the targeted sparsity using channel pruning. The \ac{amc} algorithm is either
constrained by accuracy or efficiency, depending on the reward assigned to the
agents.  Going further with the concept of automatic pruning and architecture
search, Ramakrishnan et al. \cite{DBLP:conf/crv/RamakrishnanSN20} adapt $\ell_1$
penalization from \cite{DBLP:conf/iccv/LiuLSHYZ17} to model the relative
importance of layers, groups of layers or network parts that can be removed.
Following the same line, \cite{DBLP:conf/icml/KangH20} proposed a channel
pruning method based on batch normalization parameters. The authors introduce
masks which model the likelihood of feature maps being inhibited by the ReLU
activation function and thereby not contributing to the evaluation of the
underlying network. These masks are obtained by binarizing the cumulative
density function of the gaussian distribution parameterised by the scaling and
the shift of the BN layer. Masks and BN parameters are updated “end-to-end” with
a gradient estimated using the Gumble Softmax trick
\cite{DBLP:conf/iclr/JangGP17}. Authors claim that a high accuracy is maintained
after pruning and without fine-tuning. \\



Although convenient to implement in practice, structured pruning imposes a
strong topological prior by removing whole chunks in the primary network and
achieves a lower sparsity rate compared to unstructured pruning. On the other
hand, unstructured weight pruning focuses on removing independent weights from
the global structure. As a result, this method is much more flexible and leads
to high sparsity rates and compression ratios. Han et al.
\cite{DBLP:conf/nips/HanPTD15} introduced a simple yet effective three-step
algorithm for unstructured weight pruning: a first standard training step to
identify the most important connections, a magnitude pruning step to remove the
smallest weight and a final finetuning step to compensate for the loss of
accuracy. \cite{DBLP:journals/corr/HanMD15} used the same technique in
combination with quantization and Huffman coding, achieving a compression ratio
of up to 49x for a VGG16 network. Other methods do not rely on weight magnitude
such as \cite{DBLP:conf/iclr/LouizosWK18}, which uses non-negative stochastic
gates as a surrogate L0 norm and penalise non-zero weights during training.
Variational Dropout \cite{DBLP:conf/icml/MolchanovAV17} introduces a
multiplicative gaussian noise as an alternative to binary dropout
\cite{DBLP:journals/corr/abs-1207-0580,DBLP:journals/jmlr/SrivastavaHKSS14} with
an unbound dropout rate. Magnitude pruning regains significant attention after
the publication of the Lottery Ticket Hypothesis
\cite{DBLP:conf/iclr/FrankleC19}ttery Tickets, whose training with initial
weights taken from the large networks yields comparably accurate classifiers. To
extract the lottery ticket, it is necessary to train the large network up to
convergence, apply magnitude pruning and restore the original values of the
unpruned weights. This Lottery Ticket can then be trained to match the level of
performances of the large network, with at most the same number of epochs
needed. Although remarkable, this result is hardly applicable in practice since
it requires multiple computationally intensive training steps.\\


These structured or unstructured methods propose different saliency indicators
and pruning criteria that aim at identifying and removing redundant or
unnecessary weights or groups of weights in order to remove them. Removing
weights introduces a loss of functional performance - depending on the task
considered - that needs to be compensated for (with the exception of
\cite{DBLP:conf/icml/KangH20}). This is achieved through finetuning the sparse
or lightened networks obtained after applying the pruning criterion. Finetuning
is a computationally intensive task and requires additional training time.
Moreover, the amount of weights pruned is enforced after the initial training,
meaning that the final target size or weight budget is never considered in the
optimization procedure. Hence the need for a finetuning step. \\


In order to address the aforementioned issues, we introduce a novel
reparametrization that learns not only the weights of a surrogate lightweight
network but also its topology. This reparametrization acts as a regulariser that
models the tensor of the parameters of the surrogate network as the Hadamard
product of a weight tensor and an implicit mask. The latter makes it possible to
implement unstructured pruning constrained with a budget loss that precisely
controls the number of nonzero connections in the resulting network. Experiments
conducted on the CIFAR10 and the TinyImageNet classification tasks, using
standard primary architectures (namely Conv4, VGG19 and ResNet18), show the
ability of our method to train effective surrogate pruned networks without any
fine-tuning.

\section{Pruning With Weight Reparametrization And Budget Loss}

Consider the general case of a a multi-layer neural network, denoted as a
function $f$ of two variables: $\theta$ and $X$. $f$ can be seen as the topology
of the network: a computation graph whose edge values are given by $\theta$.
Indeed, $\theta$ is the set of weights of the network, so that $\theta =
\{\mathbf{w}_1, \mathbf{w}_2, \ldots, \mathbf{w}_L\}$, where $L$ is the number
of layer of the network. $X$ is the input taken by the network. The input $X$ is
an element of a dataset $\mathcal{D}=\{ \mathcal{X}, \mathcal{Y} \}$, where
$\mathcal{X}$ is the set of the input data, and $\mathcal{Y}$ is the set of the
corresponding labels. The elements of $\mathcal{X}$ and $\mathcal{Y}$ are real
valued tensors.

\begin{equation}
    % \centering
    \begingroup
  \setlength\arraycolsep{0pt}
  f \colon\begin{array}[t]{c >{{}}c<{{}} c}
             \mathbb{R}^{\dim (X)} & \to & \mathbb{R}^{\dim (y)} \\ 
             X & \mapsto & f(X, \theta) = \hat{y} 
          \end{array}
  \endgroup
\end{equation}

Evaluating the neural network $f(X_i, \theta)$ yields the output $\hat{y_i}$
which is the prediction of the network for the input $X_i$. The discrepancy
between the output of the neural network $\hat{y_i}$ and the ground truth $y_i
\in \mathcal{Y}$ is computed with a loss function $\mathcal{L}$.  This loss is
then minimized by updating the parameters $\theta$ of the network, thanks to the
backpropagation \cite{rumelhart1985learning,rumelhart1986learning} and gradient
descent methods.\\

The $L_0$ norm is perfectly suited for introducing sparsity in a network by, on
the one hand, acting as a sparsity-inducing regulariser for the weights, and on
the other hand, by indicating the number of non-zero weights in the network,
which is useful for computing the weight budget. \\

Our aim is to propose an end-to-end method that fits into the backpropagation
framework. Therefore, adding a $L_0$ regulariser and a $L_0$ based weight budget
is not possible since $L_0$ norm is not differentiable. Thus we propose our
differentiable reparametrization, which seeks to define a novel weight
expression related to magnitude pruning
\cite{DBLP:conf/nips/CunDS89,DBLP:conf/nips/HanPTD15}. This expression
corresponds to the Hadamard product involving a weight tensor and a function
applied entry-wise to the same tensor (as shown in
\cref{fig:chap1:comparison_reparam_vs_mag_pruning}). This function acts as a
mask that i) multiplies weights by soft-pruning factors which capture their
importance and ii) pushes less important weights to zero through a particular
budget added to the loss function $\mathcal{L}$. \\

\begin{figure}[h]
    \centerline{\includegraphics[width=12.5cm]{chapter_1/assets/comparison_reparam_vs_mag_pruning.pdf}}
  \caption{Comparison of our method and magnitude pruning. Magnitude pruning
  does not include any prior on the weights during the initial training phase
  and needs an additional fine-tuning procedure. Our method embeds a saliency
  heuristic based on the weight magnitude in the weight reparametrization and
  does not requires fine-tuning.}
  \label{fig:chap1:comparison_reparam_vs_mag_pruning}
\end{figure}


Our proposed framework allows for a joint optimization of the network weights
and topology. On the one hand, it prevents disconnections which may lead to
degenerate networks with an irrecoverable performance drop. On the other hand,
it allows reaching a targeted pruning budget in a more convenient way than $L_1$
regularization. Our reparametrization also helps minimizing the discrepancy
between the primary and the surrogate networks by maintaining competitive
performances without fine-tuning. Learning the surrogate network requires only
one step that achieves pruning as a part of network design. This step zeroes out
the targeted number of connections by constraining their reparametrized weights
to vanish.

\subsection{Weight Reparametrization}
\label{sec:chap1:weight_reparam}

We consider the priramy network $f$ as a combination of $L$ layers. The global
expression of $f$ can be recursively defined the application of the layer $\ell$
to the output of the layer $\ell-1$. Without a loss of generality, we omit the
bias for the sake of clarity. This expression is shown on
\cref{eqn:chap1:layer_eq_f}.
\begin{equation}
\label{eqn:chap1:layer_eq_f}
f(\mathbf{x}) = g_L \big(\mathbf{w}_L \cdot g_{L-1}(\mathbf{w}_{L-1} \cdot g_{L-2} \dots
\mathbf{w}_2 \cdot g_1(\mathbf{w}_1 \cdot \mathbf{x}))\big),
\end{equation}
\noindent with $g_\ell$ being a nonlinear activation associated to $\ell \in
\left\{ 1,\dots, L \right\}$ and $\left\{ \mathbf{w}_\ell \right\}_\ell$ a
weight tensor. Keeping the same topology but changing the values of the weight,
we now consider the surrogate network $\hat{f}$ with weights
$\{\hat{w}_\ell\}_\ell$. \Cref{eqn:chap1:layer_eq_f} now becomes
\cref{eqn:chap1:layer_eq_f_hat}. The activation function and the topology of $f$
and $\hat{f}$ are the same. Only the weights are changing.

\begin{equation}
\label{eqn:chap1:layer_eq_f_hat}
\hat{f}(\mathbf{x}) = g_L \big(\mathbf{\hat w}_L \cdot g_{L-1}(\mathbf{\hat w}_{L-1} \cdot g_{L-2}
\dots\mathbf{\hat w}_2 \cdot g_1(\mathbf{\hat w}_1 \cdot \mathbf{x}))\big).
\end{equation}

\noindent In the above \cref{eqn:chap1:layer_eq_f_hat}, $\mathbf{\hat w}_\ell$
is referred to as apparent weight. The apparent weight is a reparametrization of
$\mathbf{w}_\ell$, that includes a prior on its saliency. An apparent weight
$\mathbf{\hat w}_\ell$ of $\hat{f}$ is derived from the standard weight
$\mathbf{w}_\ell$ of $f$ by applying the following reparametrization: 
\begin{equation}
  \label{eqn:reparam}
  \mathbf{\hat w}_\ell = \mathbf{w}_\ell  \odot h_t(\mathbf{w}_\ell),
\end{equation}
\noindent with $h_t$ being the reparametrization function and $t$ its
temperature parameter. Here, $\odot$ represents the Hadamar product. It means
that the reparametrization is element-wise, and every single weight has its own
reparametrization. This reparametrization function enforces the prior that
smallest weights should be removed from the network and act as a surrogate $L_0$
norm for the budget loss (see \cref{sec:chap1:budget_loss}). In order to achieve
this objective, $h_t$ should exhibit four properties: \\

\begin{enumerate}
  \item $\forall x \in \mathds{R},~~ 0 \leq h_t(x) \leq 1 $
  \item $h_t(x) \in C^1 \text{ on } \mathds{R}$
  \item $h_t(x) = h_t(-x)$
  \item $\forall a,\varepsilon \in\mathds{R}^{+\ast},~ \exists ~t
  \in\mathds{R}^{+\ast} ~ | ~ h_t(x) \leq \varepsilon, x \in [-a,a]$
\end{enumerate}

\noindent\textbf{First Property - Constrained Image} \\
\begin{equation}
    \centering
    \forall x \in \mathds{R},~~ 0 \leq h_t(x) \leq 1
    \label{eqn:chap1:reparam_prop1}
\end{equation}
\\
There should not be any co-adaptation between the weights and its
reparametrization. Indeed, the reparametrization function should not act as a
scaling factor for the weight and scale it so that the apparent weight is larger
than the original weight. Finally, the apparent weight should have the same sign
as the original weight. That's why the image of $\mathbb{R}$ by $h_t$ should be
the segment $[0,1]$.\\

\noindent\textbf{Second Property - Differentiability} \\
\begin{equation}
    \centering
    h_t(x) \in C^1 \text{ on } \mathds{R}
    \label{eqn:chap1:reparam_prop2}
\end{equation}
\\
Our method should fit in the backpropagation method. Since the optimization will
be achieved by gradient descent, the reparametrization function should be
derivable to ensure that it has a computable gradient.\\

\noindent\textbf{Third Property - Symmetry} \\

\begin{equation}
    \centering
    h_t(x) = h_t(-x)
    \label{eqn:chap1:reparam_prop3}
\end{equation}
\\
The reparametrization function should not induce any bias toward the positive or
negative weights, so that only their magnitudes matter. It implies that the
reparametrization function should be symmetric with respect to the origin.\\


\noindent\textbf{Fourth Property - Upper Bounded Segment} \\

\begin{equation}
    \centering
    \forall a,\varepsilon \in\mathds{R}^{+\ast},~ \exists ~t
    \in\mathds{R}^{+\ast} ~ | ~ h_t(x) \leq \varepsilon, x \in [-a,a]
    \label{eqn:chap1:reparam_prop4}
\end{equation}
\\
The last property ensures the existence of a temperature parameter $t$, which
allows upper-bounding the response of $h_t$ on any interval for any arbitrary
$\varepsilon$. More formally, for any arbitrarily large $a$ and arbitrarily
small $\varepsilon$, it exists a temperature $t$ which guarantees that the
reparametrization of any $x$ is smaller than $\varepsilon$, provided that $x$ is
in the segment $[-a, a]$. Hence, $h_t$ acts as a stopband filter which
eliminates the smallest weights where the parameter $t$ controls the width of
that filter. \\



\begin{figure}
    \centering
    \subfloat[$h_{t}$ with $t=1$ and varying $n$]{
        \includegraphics[height=0.36\linewidth]{chapter_1/assets/reparam_funct_varying_n.pdf}
        \label{fig:chap1:reparam_funct_varying_n}} \subfloat[$h_{t}$ with $n=2$
    and varying $t$]{
        \includegraphics[height=0.36\linewidth]{chapter_1/assets/reparam_funct_varying_t.pdf}
        \label{fig:chap1:reparam_funct_varying_t}} \caption{\centering
    Reparametrization function $h_t$ with varying temperature parameter $t$ and
    power $n$. $t$ controls the width of the pit and $n$ controls the steepness
    of the slope.}
    \label{fig:stopband}
\end{figure}

Weight distribution varies greatly from one layer to another. In order to match
a specific budget (see \cref{sec:chap1:budget_loss}), the width of the stopband,
controlled by $t$, is tuned according to the weight distribution of each layer.
The manual setting of this parameter is difficult and cumbersome, so in
practice, $t$ is learned as a part of gradient descent on a layer-by-layer
basis.\\

% TODO: Ajouter amorce pour discussions sur l'initialisation de la température.
%the initial setting  $t_\text{init}$ of this temperature is shown in
%\cref{tbl:pruningperformances}.\\


Considering the aforementioned four properties of $h_t$, a simple choice of that
function is 
\begin{equation}
  \label{eqn:chap1:h_star_expression}
  h_t^*(x) = \exp\bigg\{{-\displaystyle\frac{1}{(tx)^n}}\bigg\}, ~ n\in 2\mathds{N},
\end{equation}
\noindent where $n$ controls the crispness of $h_t^*$. $n$ is not considered as
a parameter of $h_t$ (or $h_t^*$) since we use a fixed value for our
experiements, whereas $t$ is a learnt parameter and varies from one layer to
another. Although the function described in \cref{eqn:chap1:h_star_expression}
satisfies the four above properties, $h_t^*$ suffers from numerical instability
as it generates \ac{nan} outputs in most of the widely used deep learning
frameworks. Due to the way Backpropagation works, a single \ac{nan} in a weight
tensor makes the whole optimization process for the entire network no longer
possible. We consider instead a stabilized variant with a similar behavior, as
\cref{eqn:chap1:h_star_expression},  that still satisfies the four above
properties (see also \cref{fig:chap1:h_stable_vs_unstable}). This numerically
stable variant is  defined as 
\begin{equation}
  \label{eqn:chap1:stable_h_expression}
  h_t(x) = C_1 \biggl( \text{exp} \bigg\{-\displaystyle\frac{1}{(tx)^n +1}\bigg\} - C_2 \biggr),
\end{equation}
\noindent with $C_1=\frac{1}{1-e^{-1}}$ and $C_2 = e^{-1}$.\\

The addition of the scalar value 1 at the denominator in
\cref{eqn:chap1:stable_h_expression} is a mean to achieve numerical stability.
In equation \cref{eqn:chap1:h_star_expression}, the denominator $(tx)^n$ has the
potential to approach very small values that result in numerical instabilities,
leading to \ac{nan} outputs. The addition of 1 to the denominator makes the
function numerically stable and avoids producing \ac{nan} outputs. This solution
is favored over adding a small value such as an arbitrarily small $\varepsilon$,
as the latter requires careful consideration of its magnitude and may result in
either dramatic alterations to the shape of the function or continued numerical
instability if not carefuly chosen. The addition of the value 1 to the
denominator provides a straightforward and sufficient mean to stabilize the
function. Constants $C_1$ and $C_2$ are introduced to compensate for the slight
alterations to the shape of the function caused by the addition of 1 to the
denominator and thus to ensure that the first property
(\cref{eqn:chap1:reparam_prop1}) is respected. Although both $h_t^*$ and $h_t$
satisfy the four properties, they do not possess the exact same shapes, as
demonstrated in figure (\ref{fig:chap1:h_stable_vs_unstable}).\\

\begin{figure}
  \centering
  \centerline{\includegraphics[width=0.5\linewidth]{chapter_1/assets/h_stable_vs_unstable.pdf}}
  \caption{\centering The unstable reparametrization function $h_t^*$ and its
  stable alternative $h_t$, with $t=1$ and $n=4$ for both functions.} 
  \label{fig:chap1:h_stable_vs_unstable}
\end{figure}

\subsection{Budget Loss}
\label{sec:chap1:budget_loss}

Most traditional pruning methods in deep learning do not explicitly incorporate
the target weight budget during the optimization procedure. The amount of
weights pruned is typically enforced post-training, which can lead to suboptimal
results compared to methods that consider the weight budget during optimization.
Our method introduces a budget loss term, in addition to the main task loss
term, that drives the network to match and respect a given weight budget during
the training process. Consequently, the trained network can be pruned to the
desired pruning rate with a marginal loss in accuracy and does not need
fine-tuning.\\


The considered budget is weight-based and should quantify the target fraction of
active connection in the network. To build the budget loss, we first introduce a
cost function that quantify the number of active connection in the network. Let
$C(\{\mathbf{w}_1,\dots, \mathbf{w}_L\})$ be the {\em current} cost associated
to a neural network and its set of {\em current} weights and $C_\text{target}$
the {\em targeted} one. $C_\text{target}$ is the number of connections that
should be active at the end of the training procedure. The budget loss is
defined as \\

\begin{equation}
  \label{eqn:chap1:simple_budget}
  {\cal L}_\text{budget} = \bigl( C(\{\mathbf{w}_1,\dots, \mathbf{w}_L\}) - C_\text{target} \bigr)^2.
\end{equation} \\


\noindent This budget loss is combined with the main task loss (that is a
classification loss in our experiments - see \cref{sec:chap1:experiments}). The
budget loss $ {\cal L}_\text{budget}$ is in quadratic form to ensure the
minimisation if this loss will in turn minimise the difference between the {\em
current} cost and the {\em targeted} one. For a better  conditioning of this
combination, we normalize the budget loss by $C_\text{initial}$. The latter
corresponds to the cost of the primary unpruned network and it is set in
practice to the number of its parameters (see also
\cref{sec:chap1:experiments}). Hence, \cref{eqn:chap1:simple_budget} is updated
as  \\

\begin{equation}
  \label{eqn:realbudget}
  {\cal L}_\text{budget} = \biggl( \displaystyle\frac{C(\{\mathbf{w}_1,\dots, \mathbf{w}_L\}) - C_\text{target}}{C_\text{initial}} \biggr)^2.
\end{equation}\\

Finally, the two losses are combined together via a strictly positive mixing
parameter $\lambda$ that  controls the relative importance of  the budget loss
${\cal L}_\text{budget}$ compared to the main task loss ${\cal L}_\text{task}$,
leading to\\

\begin{equation}
  \label{eqn:globalloss}
   {\cal L} =  {\cal L}_\text{task} + \lambda \cdot {\cal L}_\text{budget}.
\end{equation} \\

Ideally, the budget of a neural network could be evaluated as the number of
multiply-add operations, often referred as \ac{FLOPs}, needed for a forward pass
or through the $\ell_0$ norm of its weights. However, neither are known to be
differentiable and therefore cannot be used in a gradient-based optimization. In
order to circumvent this limitation, we use our weight reparametrizaion as a
surrogae measure of $\ell_0$ and we define the cost function as \\

\begin{equation}
  \label{eqn:chap1:cost_function}
  C(\{\mathbf{w}_1,\dots, \mathbf{w}_L\}) = \displaystyle \sum_{i=1}^{L} h_t(\mathbf{w}_i). 
\end{equation} \\

One could argue that the cost should be normalized layer-wise and therefore that
the right hand term of \cref{eqn:chap1:cost_function} should be written as
$$\displaystyle\sum_{i=1}^{L}\frac{h_t(\mathbf{w}_i)}{\text{Card}(\mathbf{w}_i)}$$
where $\text{Card}(.)$ denotes the cardinal function, it is to say in this case
the number of scalar elements in a weight tensor. However, the number of
elements in a layer greatly varies from one layer to another (as demonstrated in
\cref{fig:chap1:vgg16_per_layer_param_and_norm_factor}). As a result,  the
budget loss relative importance would vary from one layer to another. More
importantly, the optimization process would have less incentive to introduce
sparsity in larger layers since their normalization factor would make the budget
loss negligable compared to other layers or the main task loss. This is critical
since the large layers are generaly the one where the highest pruning rates can
be achieved \cite{DBLP:journals/corr/abs-2202-12002}. Regarding the
aforementioned reasons, a better alternative is to normalize by the initial cost
$C_\text{initial}$, as done in \cref{eqn:realbudget}.\\



\begin{figure}
  \centering
  \subfloat[Number of parameters]{
      \includegraphics[height=0.36\linewidth]{chapter_1/assets/vgg16_num_params_per_layer.pdf}
      \label{fig:chap1:num_parap_vgg16}} \subfloat[Normalization factor]{
      \includegraphics[height=0.36\linewidth]{chapter_1/assets/vgg16_normalization_factor_per_layer.pdf}
      \label{fig:chap1:norm_factor_vgg16}} \caption{\centering Log-scale plot of
      number of parameters and normalization factor per layer for a VGG16
      network. The significant differences in term of the number of parameters
      yields dramaticaly different normalization factors. Some of them are 4
      orders of magnitude appart, and all of them are vanishingly small compared
      to a standard main task loss value.} 
  \label{fig:chap1:vgg16_per_layer_param_and_norm_factor}
\end{figure}

\section{Experiments And Results}
\label{sec:chap1:experiments}

\begin{figure}
  \centering
  \subfloat[Conv4 - CIFAR10]{
      \includegraphics[height=0.36\linewidth]{chapter_1/assets/reparam_vs_mpft_Conv4_cifar10.pdf}
      \label{fig:chap1:reparam_vs_mpft_conv4_cifar10}}
  \subfloat[Conv4 - CIFAR100]{
      \includegraphics[height=0.36\linewidth]{chapter_1/assets/reparam_vs_mpft_Conv4_cifar100.pdf}
      \label{fig:chap1:reparam_vs_mpft_conv4_cifar100}}
      \\
  \subfloat[Conv4 - CIFAR10 (Number of Epochs)\label{fig:chap1:reparam_vs_mpft_conv4_cifar10_epochs}]{
      \includegraphics[height=0.36\linewidth]{chapter_1/assets/reparam_vs_mpft_training_time_Conv4_cifar10.pdf}}
  \subfloat[Conv4 - CIFAR100 (Number of Epochs)\label{fig:chap1:reparam_vs_mpft_conv4_cifar100_epochs}]{
      \includegraphics[height=0.36\linewidth]{chapter_1/assets/reparam_vs_mpft_training_time_Conv4_cifar100.pdf}}
  \caption{\centering Performances comparison of our method {\em(Ours)} against
  magnitude pruning without {\em(MP w/o FT)} and with fine-tuning {\em(MP w/ FT)} with a Conv4 network on
  CIFAR10 and CIFAR100 datasets, for different pruning rates.
  \Cref{fig:chap1:reparam_vs_mpft_conv4_cifar10} and
  \cref{fig:chap1:reparam_vs_mpft_conv4_cifar100} show the testing accuracy of
  the model and \cref{fig:chap1:reparam_vs_mpft_conv4_cifar10_epochs} and
  \cref{fig:chap1:reparam_vs_mpft_conv4_cifar100_epochs} the number of epochs
  needed to obtain the best model.}
  \label{fig:chap1:reparam_vs_mpft_conv4}
\end{figure}

\begin{figure}
  \centering
  \subfloat[VGG16 - CIFAR10]{
      \includegraphics[height=0.36\linewidth]{chapter_1/assets/reparam_vs_mpft_PrunableVGG16_cifar10.pdf}
      \label{fig:chap1:reparam_vs_mpft_vgg16_cifar10}} 
  \subfloat[VGG16 - CIFAR100]{
      \includegraphics[height=0.36\linewidth]{chapter_1/assets/reparam_vs_mpft_PrunableVGG16_cifar100.pdf}
      \label{fig:chap1:reparam_vs_mpft_vgg16_cifar100}} 
  \\
  \subfloat[VGG16 - CIFAR10 (Number of Epochs)\label{fig:chap1:reparam_vs_mpft_vgg16_cifar10_epochs}]{
    \includegraphics[height=0.36\linewidth]{chapter_1/assets/reparam_vs_mpft_training_time_PrunableVGG16_cifar10.pdf}}
  \subfloat[VGG16 - CIFAR100 (Number of Epochs)\label{fig:chap1:reparam_vs_mpft_vgg16_cifar100_epochs}]{
      \includegraphics[height=0.36\linewidth]{chapter_1/assets/reparam_vs_mpft_training_time_PrunableVGG16_cifar100.pdf}}

    
  \caption{\centering Performances comparison of our method \em{(Ours)} against
  magnitude pruning with fine-tuning \em{(MP+FT)} with a VGG16 network on
  CIFAR10 and CIFAR100 datasets, for different pruning rates.
  \Cref{fig:chap1:reparam_vs_mpft_vgg16_cifar10} and
  \cref{fig:chap1:reparam_vs_mpft_vgg16_cifar100} show the testing accuracy of
  the model and \cref{fig:chap1:reparam_vs_mpft_vgg16_cifar10_epochs} and
  \cref{fig:chap1:reparam_vs_mpft_vgg16_cifar100_epochs} the
  number of epochs needed to obtain the best model.}
  \label{fig:chap1:reparam_vs_mpft_vgg16}
\end{figure}

\begin{figure}
  \centering
  \subfloat[ResNet20 - CIFAR10\label{fig:chap1:reparam_vs_mpft_resnet20_cifar10}]{
      \includegraphics[height=0.36\linewidth]{chapter_1/assets/reparam_vs_mpft_PrunableResNet20_cifar10.pdf}}
  \subfloat[ResNet20 - CIFAR100\label{fig:chap1:reparam_vs_mpft_resnet20_cifar100}]{
      \includegraphics[height=0.36\linewidth]{chapter_1/assets/reparam_vs_mpft_PrunableResNet20_cifar100.pdf}} 
  \\
  \subfloat[ResNet20 - CIFAR10 (Number of Epochs)\label{fig:chap1:reparam_vs_mpft_resnet20_cifar10_epochs}  ]{
      \includegraphics[height=0.36\linewidth]{chapter_1/assets/reparam_vs_mpft_training_time_PrunableResNet20_cifar10.pdf}}
  \subfloat[ResNet20 - CIFAR100 (Number of Epochs)\label{fig:chap1:reparam_vs_mpft_resnet20_cifar100_epochs}]{
      \includegraphics[height=0.36\linewidth]{chapter_1/assets/reparam_vs_mpft_training_time_PrunableResNet20_cifar100.pdf}}
  \caption{\centering Performances comparison of our method \em{(Ours)} against
  magnitude pruning with fine-tuning \em{(MP+FT)} with a ResNet20 network on
  CIFAR10 and CIFAR100 datasets, for different pruning rates.
  \Cref{fig:chap1:reparam_vs_mpft_resnet20_cifar10} and
  \cref{fig:chap1:reparam_vs_mpft_resnet20_cifar100} show the
  testing accuracy of the model and
  \cref{fig:chap1:reparam_vs_mpft_resnet20_cifar10_epochs} and
  \cref{fig:chap1:reparam_vs_mpft_resnet20_cifar100_epochs}
  the number of epochs needed to obtain the best model.}
  \label{fig:chap1:reparam_vs_mpft_resnet20}
\end{figure}

\chapter{Mask Training}

\begin{abstract}
    abstract of the chapter
\end{abstract}

\section{Introduction and related work}
% region: introduction
Deep neural networks are nowadays becoming mainstream in solving many image
processing tasks including visual category recognition. The success of these
models has been reached at the expense of an increase in their inference time,
memory consumption and energy footprint. With the era of intelligent embedded
systems (provided with limited energy and computational resources), a current
trend is to make these models {\it lightweight and frugal} while maintaining
their high accuracy.   Existing solutions in lightweight network design are
targeted toward creating small and efficient architectures from scratch
\cite{DBLP:conf/cvpr/HuangLMW18, DBLP:conf/cvpr/SandlerHZZC18,
DBLP:journals/corr/HowardZCKWWAA17, DBLP:conf/icml/TanL19} while others derive
highly compact yet effective neural networks from larger ones. These methods
predominantly include knowledge distillation
\cite{DBLP:journals/corr/HintonVD15, DBLP:conf/iclr/ZagoruykoK17,
DBLP:journals/corr/RomeroBKCGB14, DBLP:conf/aaai/MirzadehFLLMG20,
DBLP:conf/cvpr/ZhangXHL18, DBLP:conf/cvpr/AhnHDLD19} and pruning
\cite{DBLP:conf/nips/CunDS89, DBLP:conf/nips/HassibiS92,
DBLP:conf/nips/HanPTD15}. \\  

Pruning methods, either structured or unstructured, are particularly successful,
and seek to remove connections with the least perceptible impact on
classification accuracy. Structured pruning consists in {\it jointly} removing
groups of weights, entire channels or subnetworks
\cite{DBLP:conf/iclr/0022KDSG17, DBLP:conf/iccv/LiuLSHYZ17}, whereas
unstructured pruning aims at removing weights {\it individually}
\cite{DBLP:conf/nips/HanPTD15,DBLP:journals/corr/HanMD15}.  Unstructured pruning
has witnessed a recent surge in interest in the wake of the Lottery Ticket
Hypothesis  \cite{DBLP:conf/iclr/FrankleC19}; an empirical study in
\cite{DBLP:conf/iclr/FrankleC19} shows that large pretrained networks encompass
subnetworks, called \textit{Lottery Tickets}, whose training with initial
weights taken from the large networks yields comparably accurate classifiers.
Another study \cite{DBLP:conf/iclr/LiuSZHD19} pushes that finding further and
concludes that only the topology of these subnetworks is actually important in
order to reach comparable performances.  In general, extracting an efficient
subnetwork is still an open problem and is computationally demanding as this
amounts to full training of large networks (till convergence) prior to their
pruning. Existing alternatives approach this problem using early pruning
\cite{DBLP:conf/iclr/LeeAT19,
DBLP:conf/iclr/WangZG20,DBLP:conf/nips/TanakaKYG20}, but still require to train
the weights. In contrast to these works, our proposed solution in this paper
identifies effective subnetworks by training only their topology and without any
weights tuning.\\ 

A theoretical analysis in
\cite{DBLP:conf/icml/MalachYSS20,DBLP:conf/nips/PensiaRNVP20,DBLP:conf/nips/OrseauHR20}
has established the sufficient conditions about the existence of efficient and
effective subnetworks in over-parameterized large networks, nonetheless, no
constructive proof has been provided in order to identify these subnetworks. In
this context, Zhou et al. \cite{DBLP:conf/nips/ZhouLLY19} proposed the first
attempt to extract efficient subnetworks using stochastic mask training. A
probability of selecting each weight is defined (as the sigmoid of a mask) and
trained using the \ac{STE} \cite{DBLP:journals/corr/BengioLC13}. During
training, weights are frozen and only the masks are allowed to vary. However,
the major drawback of this method resides in the vanishing gradient of the
sigmoid which makes mask training numerically challenging. Ramanujan et al.
\cite{DBLP:conf/cvpr/RamanujanWKFR20} proposed another alternative, based on
binarized saliency indicators learned with \ac{STE}, which selects the most
prominent weights in the resulting subnetworks. Nevertheless, since this method
enfore the pruning rate \textit{a priori}, finding the pruning rate giving the
higest performances has to be made through a cumbersome and time-consuming
binary search or grid-search. \\ 

Considering the limitation of the aforementioned related  work, we introduce in
this paper a new stochastic subnetwork selection method based on Gumbel Softmax.
The latter allows sampling subnetworks whose weights are the most relevant for
classification. The proposed contribution also relies on a new mask
parametrization, dubbed as \ac{ASLP}, that allows a better conditioning of the
gradient and thereby mitigates numerical instability during mask optimization.
Besides, when combining \ac{ASLP} with a learned weight rescaling mechanism,
training is accelerated and the accuracy of the resulting subnetworks improves
as shown later in experiments.
% endregion: introduction


\section{Extracting Effective Subnetworks with Gumbel-Softmax}
% region: method
% endregion: method


\subsection{Mask Training with Gumbel Softmax}
% region: mask training
% endregion: mask training


\subsection{Smart Weight rescaling}
% region: smart rescaling
% endregion: smart rescaling


\section{Experiments}
% region: experiments

\subsection{Performances}

\subsection{Impact of initialisation}

\subsection{Impact of smart rescaling}

% endregion: experiments


\section{Conclusion}
% region : conclusion
%endregion : conclusion


Let $f_\theta$ be a deep neural network whose weights defined as $\theta =
\left\{\bm{w}_1,\mathellipsis, \bm{w}_L \right\}$, with $L$ being its depth,
$\bm{w}_\ell \in \mathbb{R}^{d_{\ell} \times d_{\ell-1}}$ its
$\ell^\textrm{th}$ layer weights, and $d_\ell$ the dimension of $\ell$. The
output of a given layer $\ell$ is defined as 
\begin{equation}
  \label{eq:layer_eq}
  \mathbf{z}_{\ell} = g_\ell(\bm{w}_\ell \otimes \mathbf{z}_{\ell-1}),
\end{equation}
being  $g_\ell$ an activation function and $\otimes$ the usual matrix product.
Without a loss of generality, we omit the bias in the definition of
(\ref{eq:layer_eq}).

% ------------------------------------------------------------------------------
\subsection{Stochastic Weight Sampling}
\indent Given a network $f_\theta$, weight pruning consists in removing
connections in the graph of $f_\theta$. A node in this graph refers to a
neural unit while an edge corresponds  to a cross-layer connection. Pruning is
usually obtained by freezing and zeroing-out  a subset of weights in $\theta$,
and this is achieved  by multiplying $\bm{w}_\ell$ by a binary mask
$\bm{m}_\ell \in \{ 0,1 \}^{\text{dim}(\bm{w}_\ell)}$. The
binary entries of $\bm{m}_\ell$ are set depending on whether the underlying
layer connections are kept or removed, so \Cref{eq:layer_eq} becomes
\begin{equation}
  \label{eq:pruned_layer_eq}
  \mathbf{z}_{\ell} = g_\ell( (\bm{m}_\ell \odot \bm{w}_\ell ) \otimes \mathbf{z}_{\ell-1} ).
\end{equation}
Here $\odot$ stands for the element-wise matrix product. In this definition,
the masks $\{\bm{m}_\ell\}_\ell$ are stochastic and sampled from a Bernoulli
distribution.\\

\noindent\textbf{Straight Through Estimator.} Zhou et
al.~\cite{DBLP:conf/nips/ZhouLLY19} consider a Bernoulli parametrization of
$\{\bm{m}_\ell\}_\ell$ in order to sample masks in \Cref{eq:pruned_layer_eq}.
However, due to sampling which is not a differentiable operation, optimizing directly
$\{{\bm{m}_\ell}\}_\ell$ is not possible. Existing
solutions, including \cite{DBLP:conf/nips/ZhouLLY19}, rely on the Straight
Trough Estimator (STE), already described
in~\cite{DBLP:journals/corr/BengioLC13}.  The definition of
$\{\bm{m}_\ell\}_\ell$ is instead based on another {\it latent}
parametrization $\{\bm{\hat{m}}_\ell\}_\ell$, detailed subsequently, and
obtained by applying a sigmoid function $\sigma(.)$ to $\bm{\hat{m}}_\ell$.
This allows optimizing $\bm{\hat{m}}_\ell$  using gradient descent while
considering the following surrogate of  \Cref{eq:pruned_layer_eq} 
\begin{equation}
  \label{eq:pruned_layer_eq2}
  \mathbf{z}_{\ell} = g_\ell( ( \sigma(\bm{\hat{m}}_\ell) \odot \bm{w}_\ell ) \otimes \mathbf{z}_{\ell-1} ).
\end{equation}

\noindent Authors in  \cite{DBLP:conf/nips/ZhouLLY19} use the STE in
order to back-propagate the gradient and to update the parameters of the
Bernoulli distribution $\bm{\hat{m}}_\ell$ with gradient descent.\\

\noindent\textbf{Gumbel-Softmax.} In what follows, we consider an alternative
STE based on Gumbel Softmax (GS)~\cite{DBLP:conf/iclr/JangGP17}. The proposed
method, dubbed as Straight Through Gumbel Softmax (STGS), is based (i) on a
variant of GS, and also (ii) on the argmax operator which allows sampling from
a categorical distribution, as the limit of GS (i.e., when its softmax temperature
approaches zero).  Let $z$ be a categorical random variable, associated with $n$
class probability distribution $\mathcal{P} = [\pi_1,\dots,\pi_n]$. The Gumbel
Softmax estimator (i) takes a vector of log-probabilities $\log(\mathcal{P})
=[\log(\pi_1),\dots, \log(\pi_n)]$ as an input, (ii) disrupts the latter with
a random additive noise sampled from the Gumbel distribution, and (iii) takes
the argmax, yielding a categorical variable. More formally, following
\cite{DBLP:conf/iclr/JangGP17}, the value $q$ of our categorical variable $z$
is obtained as 
\begin{equation}
  \label{eq:gumbel-softmax-argmax}
  q = \underset{k}{ \text{argmax}} \ [ \log(\pi_k)+g_k ],
\end{equation}
with $g_k$ being i.i.d sampled from  the Gumbel distribution.\\
\noindent In what follows, and unless stated otherwise, we omit $\ell$ from
$\bm{w}_\ell$ and we write it for short as $\bm{w}$. Let $\bm{w}_{ij}$ be the
weight associated to the i-th and j-th neurons  respectively belonging to
layers $\ell-1$ and $\ell$; we define a two-class categorical
distribution $\mathcal{P}_{ij}$ on $\{0,1\}$ as
$\mathcal{P}_{ij}(z=1)=\pi_1^{ij}$, and $\mathcal{P}_{ij}(z=0)=\pi_2^{ij}$
with $\pi_1^{ij}=p_{ij}$, $\pi_2^{ij}=1-p_{ij}$ and $p_{ij}$ being the
probability to keep the underlying connection. In other words, keeping the
weight $\bm{w}_{ij}$ (or not) in the sampled topology is a Bernoulli trial
with a probability $p_{ij}$. Considering
\Cref{eq:gumbel-softmax-argmax}, a binary mask  $\bm{m}_{ij}$ is defined as
$1_{\{q_{ij}=1\}}$, $1_{\{\}}$ being the indicator function and $q_{ij} =
{\text{argmax}_{k \in \{1,2\}}}\big[\log(\pi_k^{ij})+g_k^{ij}\big]$.
\noindent Thanks to STGS, it becomes possible to learn $p_{ij}$ for each
weight through stochastic gradient descent (SGD). However, optimizing $p_{ij}$
(with SGD) raises a major issue as $p_{ij}$ may not be appropriately bounded
and thereby $\log(p_{ij})$ and $\log(1-p_{ij})$ would also be undefined.  On
another hand,  solving constrained SGD, besides being computationally
expensive and challenging, may result into worse local minimum. In order to
overcome all these issues, one may consider an alternative
reparametrization $p_{ij}=\sigma(\bm{\hat{m}}_{ij})$, with $\bm{\hat{m}}_{ij}$
being a latent mask variable and $\sigma$ the sigmoid function which bounds
$p_{ij}$ in $[0,1]$. However, this workaround suffers (in practice) from
numerical instability in gradient estimation (due to the log and the sigmoid)
and is also computationally demanding. \\

  

\noindent\textbf{Arbitrarily Shifted Log Parametrization.}
Another alternative is to consider $\bm{\hat{m}}_{ij} =
\log(p_{ij})$ and $\log(1-p_{ij}) = \log(1-\exp(\bm{\hat{m}}_{ij}))$ and learn
the underlying mask. However, this reparametrization is also flawed in the
same way as the aforementioned sigmoid reparametrization. In what follows, we
propose an equivalent formulation which turns out to be highly effective and
numerically more stable. Considering  

\begin{equation}
  \begin{bmatrix}
    \bm{\hat{m}}_{ij} \\
    0  \\
  \end{bmatrix}
  = \log\big(\mathcal{P}_{ij}(.)\big) + c =
  \begin{bmatrix}
    \log(p_{ij}) + c \\
    \log(1-p_{ij}) + c\\
  \end{bmatrix},
  \label{eq:our-formulation}
\end{equation}

\noindent in the above definition, instead of using $\log(\mathcal{P}_{ij}(.))$, we
consider $\log(\mathcal{P}_{ij}(.)) + c$  as an input of the argmax in
\cref{eq:gumbel-softmax-argmax}. The constant $c \in \mathds{R}$ ensures that
if $\bm{\hat{m}}_{ij} > 0$, then $\log(p_{ij}) \in ]-\infty,0] \Leftrightarrow
p_{ij} \in [0,1]$. This is enforced by setting the second coefficient of
$\mathcal{P}_{ij}$ to 0, rather than computing it explicitly. The formulation
of \cref{eq:our-formulation} is theoretically equivalent to the aforementioned
sigmoid reparametrization. Indeed, solving the system of
\cref{eq:our-formulation} w.r.t. $\bm{\hat{m}}_{ij}$ yields $p_{ij} =
\sigma(\bm{\hat{m}}_{ij})$. \noindent Differently put, the formulation in
\cref{eq:our-formulation} considers a reparametrization $\bm{\hat{m}}_{ij} =
\log(p_{ij})+c$ which is strictly equivalent to the sigmoid one while being
computationally more efficient and also stable. Note that adding any arbitrary constant $c$ to the log-probability
makes the outcome of Gumbel-Softmax sampling and argmax invariant.


  % ------------------------------------------------------------------------------
  \subsection{Weight Rescaling}
  \label{sec:smart-rescale}
Subnetwork selection may disrupt the dynamic of the forward pass
\cite{DBLP:conf/iccv/HeZRS15,DBLP:conf/cvpr/RamanujanWKFR20}, and thereby
requires adapting  weights accordingly.    Dynamic weight rescale (DWR)
\cite{DBLP:conf/nips/ZhouLLY19}, and scaled Kaiming distribution
\cite{DBLP:conf/cvpr/RamanujanWKFR20}  are two  known mechanisms that adapt the
weights of the selected subnetworks.  However, some of these heuristics, besides
being handcrafted,   rely on the strong assumption that rescaling should be
proportional to the pruning rate.  In what follows,  we consider a new weight
adaptation mechanism, referred to as Smart Rescale (SR). Instead of handcrafting
this rescaling factor proportionally to the pruning rate (as achieved for
instance in \cite{DBLP:conf/nips/ZhouLLY19}), SR is learned layerwise and
provides an effective (and also efficient)  way to adapt the dynamic of the
forward pass without retraining the entire weights of the selected subnetwork.
Indeed, this rescaling ends up reducing the amount of epochs needed to reach
convergence and also improving accuracy (at some extent) as shown later in
experiments. \\ With SR, the $\ell$-th layer network output becomes 
  \begin{equation}
   \mathbf{z}_{\ell} = g_\ell(s_\ell \times (\bm{m}_\ell \odot \bm{w}_\ell) \otimes \mathbf{z}_{\ell-1}),
  \end{equation}
  \noindent where  $s_\ell$ refers to the rescaling factor  of  the $\ell$-th
  layer (see also algorithm~\ref{alg:gumbel-forward}).  Smart Rescale increases
  the flexibility of subnetwork selection and adaptation compared to DWR (which
  is bound to the pruning rate).  Moreover,  scaling factors obtained with SR
  vary smoothly  ---  and this makes training more stable with stochastic
  gradient descent (SGD) --- compared to the ones obtained with DWR which are
  again set to the observed {\it pruning rates},  and changes of the latter are
  more abrupt due to stochastic mask sampling.    
  \begin{algorithm}
    \caption{Forward pass for our method}
    \label{alg:gumbel-forward}
    \begin{algorithmic}[1]
    \REQUIRE A network $f_\theta$, with weights $\{\bm{w}_\ell\}_\ell$, ASLP  $\{\bm{\hat{m}}_\ell\}_\ell$, and input training data $\{({\bf x}_k,{\bf y}_k)\}_k$ 
    \STATE
    $q_{i,j} \gets \text{argmax} 
    \begin{bmatrix}
      \bm{\hat{m}}_{i,j} + g_{i,j} \\
      0 + g'_{i,j}\\
    \end{bmatrix}$ \COMMENT{Sampling of a topology} 
    \STATE $m_{ij} \gets 1_{\{q_{ij}=1\}}$
   \COMMENT{
    Giving the masks $\bm{m}_{i,j}$ their values} 
    \STATE {\bf Return} ${\cal L}\big(f_\theta(\{{\bf x}_k\}_k; \{s_\ell (\bm{m}_\ell \odot \bm{w}_\ell)\}_\ell),\{{\bf y}_k\}_k\big)$
    \COMMENT{Computing the loss with masked weights and SR}
    \end{algorithmic}
  \end{algorithm} 
  % ==============================================================================
  \section{Experiments}\label{sec:experiments}
  
  In this section, we show the performance of our method on the standard CIFAR10
  and CIFAR100 datasets.   They consist of  60k  colored images of $32\times 32$
  pixels each.  Training,  validation and test sets  include  45k,  5k and 10k
  images respectively.  \\  In order to demonstrate the effectiveness of our
  method, we chose the widely used SGD optimizer with a momentum of  0.9  and a
  learning rate of 50.    Faster convergence is obtained with higher learning
  rates,   however,  the latter also lead  to worse observed accuracy.   During
  training,  the maximum number of epochs is set to 1000 and early stopping is
  triggered  if the accuracy on the validation set stops improving during 100
  epochs.   In all these experiments, neither weight decay nor $\ell_2$
  regularization are applied. See implementation details and our code on the ASLP GitHub
  \cite{Dupont2022}.

\begin{table*}[htbp]
  \centering
  \resizebox{14.0cm}{!} 
{
  \begin{tabular}{@{}llcccccccc|c@{}}
    \cline{3-11}
                       &                                                        & \multicolumn{8}{c|}{Cifar 10}                                                                                                 & Cifar 100 \\ 
                       \cline{3-11}
                            &                                                       & \multicolumn{4}{c}{w/o data augmentation}                                     & \multicolumn{4}{c|}{with data augmentation (w.d.a)}         &  w.d.a  \\
                            &                                                        & $\varnothing$ & WR             & SC            & WR+SC          & $\varnothing$ & WR             & SC            & WR+SC          &           WR+SC   \\ \midrule
    \multirow{4}{*}{Conv2} & \cite{DBLP:conf/nips/ZhouLLY19} (averaging)         &64.4          & 65.0          & 66.3          & 66.0          & -             & -             & -             & -            &               -        \\
                            & \cite{DBLP:conf/cvpr/RamanujanWKFR20}\footnotemark ($k=50\%$)  &               &   -           &         -     &         -     &       -       &      -        & 71.5   &       71.7       &     40.9         \\ 
                            & Our ASLP (averaging)                                     & 68.2          & 66.9          & 68.3          & 66.5          & \textbf{76.0} & \textbf{76.6} & 76.8          & 77.3          &       -         \\
                            & Our ASLP (thresholding)                                     & \textbf{68.7} & \textbf{67.8} & \textbf{68.4} & \textbf{67.1} & 75.9          & 76.4          & \textbf{77.5} & \textbf{77.5} &    \textbf{43.3}          \\
                            \midrule
  
    \multirow{4}{*}{Conv4} & \cite{DBLP:conf/nips/ZhouLLY19} (averaging)       & 65.4          & 71.1          & 66.2          & 72.5          & -             & -             & -             & -             &             -           \\
                            & \cite{DBLP:conf/cvpr/RamanujanWKFR20}\footnotemark[\value{footnote}] ($k=50\%$) &        -      &  -            &            -  &  -            &  -            &  -            & 81.6  &     80.5    &  51.1  \\ 
                            & Our ASLP (averaging)                                          & 70.6          & 71.8          & 69.5          & 71.8          & 83.4          & 84.4          & 83.7          & 84.1          &        -   \\
                            & Our ASLP (thresholding)                                      & \textbf{71.5}  & \textbf{72.8}& \textbf{70.2} & \textbf{72.7} & \textbf{83.7}& \textbf{85.0} & \textbf{84.5} & \textbf{84.8} &         \textbf{51.7}  \\
                            \midrule
    \multirow{4}{*}{Conv6} & \cite{DBLP:conf/nips/ZhouLLY19} (averaging)    & 63.5          & 76.3          & 65.4          & 76.5          & -             & -             & -             & -      &  -       \\
                            & \cite{DBLP:conf/cvpr/RamanujanWKFR20}\footnotemark[\value{footnote}] ($k=50\%$) &      -        &    -          &        -      &     -         &      -        &           -   &  85.4 &    85.1   &   \textbf{53.8}   \\ 
                            & Our ASLP (averaging)                                      & 72.9          & 76.1          & 71.9          & 75.6          & 85.3          & 86.2          & 85.3          & 86.2          &  - \\
                            & Our ASLP (thresholding)                                       & \textbf{73.7} &\textbf{77.0}  &\textbf{72.6}  &\textbf{76.6}  &\textbf{86.0}  &\textbf{86.9}  & \textbf{86.3} &\textbf{86.9}  &   52.8 \\
                            \bottomrule
    \end{tabular}
}

  \caption{\footnotesize Comparison of our method against \cite{DBLP:conf/nips/ZhouLLY19}
  and \cite{DBLP:conf/cvpr/RamanujanWKFR20} on Conv2, Conv4 and Conv6. These
  results are averaged through five independent runs. "WR" (Weight Rescale)
  refers to ``Dynamic Weight Rescale'' or ``Smart Rescale'' depending on which
  methods is used (respectively \cite{DBLP:conf/nips/ZhouLLY19} or our proposed
  ASLP). Again, "SC" refers to the ``Signed Constant'' distribution. The latest results on CIFAR 100
  were recently obtained with data augmentation and WR+SC.}
  \label{tbl:conv_compare}

\end{table*}

    

% ------------------------------------------------------------------------------
\subsection{Performance and comparison}
The accuracy of our method is evaluated on subnetworks whose topology
corresponds to connections with  (trained) probabilities larger than 0.5;  in
other words,  if {\it the binary event of keeping a connection is more likely
than its removal}.  This setting is referred to as  {\it thresholding}.  As a
matter of comparison,  we also consider the setting in
\cite{DBLP:conf/nips/ZhouLLY19} which consists in sampling ten different
subnetworks and evaluating an average accuracy over  these subnetworks.  This
setting  is referred to as {\it averaging}.  In these experiments,  we use the
same networks as
\cite{DBLP:conf/nips/ZhouLLY19,DBLP:conf/cvpr/RamanujanWKFR20} (originally
introduced by  Frankle and Carbin \cite{DBLP:conf/iclr/FrankleC19}) namely
Conv2,  Conv4 and Conv6 which are  variants of  VGG16. \\
\indent   \cref{tbl:conv_compare}  shows a comparison of our method against
\cite{DBLP:conf/nips/ZhouLLY19,DBLP:conf/cvpr/RamanujanWKFR20}.    These
results show means of five independent runs; each run corresponds either to
``thresholding'' or ``averaging''.  These performances show a consistent gain
(in accuracy) of our subnetwork selection.   We also observe that
``thresholding'' is already effective compared to ``averaging''; indeed, our
method reaches a  high accuracy despite learning a single subnetwork topology,
and this makes it also highly efficient for training compared to  the related
work \cite{DBLP:conf/nips/ZhouLLY19,DBLP:conf/cvpr/RamanujanWKFR20}.\\ 
\indent Furthermore, our method and \cite{DBLP:conf/nips/ZhouLLY19} do not
impose a pruning rate. The optimal pruning rate is found during optimization and
is is arround 51\%, whereas \cite{DBLP:conf/cvpr/RamanujanWKFR20} enforces a 50\%
pruning rate ($k=50\%$). Thus, the networks capacities
are comparable. \\


  % ------------------------------------------------------------------------------
  \subsection{Ablation study}
  In this section, we discuss the impact of all the components of the method
  when taken individually and combined,  namely the use of weight rescaling
  (WR): either DWR or our proposed SR.   We also consider another criterion:
  signed constant  (SC) which consists in replacing weights in a given layer by
  the products of their signs and the standard deviation of their original
  weight distribution.  We show  all these results with and without data
  augmentation, which is composed of the combination of zero-padding,  random crops and
  random  horizontal flips.   Note that  pixel intensities are normalized  from
  their original values in $[0,255]$ to $[0,1]$.
    
  From the results in table~\ref{tbl:conv_compare}, we observe a clear gain of
  our method alone w.r.t.  \cite{DBLP:conf/nips/ZhouLLY19} and the use of SR
  increases further its accuracy (excepting Conv2 w/o data augmentation).  The
  gain in performances increases  significantly with Conv6 and reaches up to 4
  points even when no data augmentation is used.   Note that the use of data
  augmentation attenuates, at some extent, the effect of SR on larger networks
  (conv4 and Conv6). Nonetheless,  as discussed in \cref{sec:SR-impact}, the
  positive impact of SR resides also in training efficiency.   In contrast to
  SR, signed constant improves accuracy by a small margin when combined with
  data augmentation. 
  
  \footnotetext{Performances for \cite{DBLP:conf/cvpr/RamanujanWKFR20} are reported
  with the optimizer described in \cref{sec:experiments}. It is possible to
  improve performances by tuning the learning rate scheduler but this is out of
  the scope of this paper.}
  
  % ------------------------------------------------------------------------------
  \subsection{Computational efficiency}
  \label{sec:SR-impact}
DWR requires rectifying weights layerwise using the inverse of the observed
(computed) pruning rates. These layerwise evaluations introduce a significant
overhead at each training epoch. In contrast, SR consists in simple products
involving one scalar per layer. When training Conv4, we found (on average) that
enabling DWR increases epoch runtime by $0.2s$ while our SR by $0.13s$ only, so
SR speeds up training overhead by 35\% compared to DWR.  When data augmentation
and signed constant are used, SR allows a significant gain in the number of
training epochs. Indeed, enabling SR on Conv4  saves (on average) 19.7\%
training epochs (8.2\%, 14.0\%  on Conv2 and Conv6 respectively) before
converging to its highest accuracy. Finally, our ``thresholding'' setting not
only improves accuracy but makes subnetwork selection (training) and also
inference more efficient compared to the related work
\cite{DBLP:conf/nips/ZhouLLY19,DBLP:conf/cvpr/RamanujanWKFR20},  as this
selection is again achieved  once and thereby only one subnetwork is applied
during inference.
 

% ==============================================================================
\section{Conclusion}

In this paper, we introduce a novel method that extracts effective subnetworks
from larger networks without training its weights. The proposed method optimizes
a probability distribution which measures the relevance of weights, and only
those with the highest relevance define the topology of the selected
subnetworks. An efficient and effective weight rescaling mechanism is also
introduced and allows rectifying the parameters of the selected subnetworks
which improves performances and reduces the number epochs needed to reach
convergence. Experiments conducted on the standard CIFAR10 and  CIFAR100
datasets show the effectiveness of our subnetwork selection method w.r.t. the
related work. Future work includes the study of the scalability of the proposed
method on more complex datasets and other larger networks.\\

\noindent\textbf{Acknowledgement.} 
This work was performed using HPC resources from GENCI-IDRIS (Grant 2021-AD011011427R1). \\
It has been achieved within a partnership between Sorbonne University and Netatmo.

\noindent\textbf{Code.} Our code is available at:\\ \href{https://github.com/N0ciple/ASLP}{\texttt{https://github.com/N0ciple/ASLP}}

\begin{table}
  \centering
  \resizebox{16.5cm}{!} {
    \begin{tabular}{llcccccccc}
      \cmidrule[\heavyrulewidth]{3-10}
        &  & \multicolumn{4}{c}{\textbf{w/ data augmentation}} & \multicolumn{4}{c}{\textbf{w/o augmentation}} \\
       &  &  $\varnothing$ & \textbf{SC} & \textbf{WR} & \textbf{WR+SC} & $\varnothing$ & \textbf{SC} & \textbf{WR} & \textbf{WR+SC} \\
      \toprule

      % Conv 2 results
      \multirow{18}{*}{} \multirow{6}{*}{\textbf{Conv2}} & ASLP (pre-pruning) & 75.55 $\pm$ 0.15 & 75.47 $\pm$ 0.69 & 76.04 $\pm$ 0.44 & 76.29 $\pm$ 0.28 & 68.05 $\pm$ 0.46 & 67.68 $\pm$ 0.68 & 65.98 $\pm$ 0.77 & 65.25 $\pm$ 0.68 \\
        & \cite{DBLP:conf/nips/ZhouLLY19} (pre-pruning) & - & - & - & - & 67.21 $\pm$ 0.34 & 66.35 $\pm$ 0.78 & 56.57 $\pm$ 2.99 & 56.25 $\pm$ 1.77 \\
      \cmidrule(lr){2-10}
      % Conv 2 post pruining
        & ASLP (thresholding) & \textbf{75.70 $\pm$ 0.30} & \textbf{75.81 $\pm$ 0.69} & \textbf{76.48 $\pm$ 0.68} & \textbf{76.92 $\pm$ 0.24} & \textbf{68.24 $\pm$ 0.14} &\textbf{ 68.11 $\pm$ 0.64} & \textbf{66.84 $\pm$ 0.46} & \textbf{66.05 $\pm$ 0.93} \\
        & ASLP (averaging)& 75.42 $\pm$ 0.25 & 75.50 $\pm$ 0.56 & 76.05 $\pm$ 0.44 & 76.44 $\pm$ 0.19 & 68.09 $\pm$ 0.35 & 67.69 $\pm$ 0.52 & 65.79 $\pm$ 0.65 & 65.35 $\pm$ 0.83 \\
        & \cite{DBLP:conf/nips/ZhouLLY19} (averaging)& - & - & - & - & 67.12 $\pm$ 0.25 & 66.34 $\pm$ 0.41 & 56.71 $\pm$ 2.99 & 56.26 $\pm$ 1.64 \\
        & \cite{DBLP:conf/cvpr/RamanujanWKFR20} ($k=50\%$) & 74.18 $\pm$ 0.76 & 75.19 $\pm$ 0.56 & 74.51 $\pm$ 0.31 & 75.45 $\pm$ 0.44 & - & - & - & - \\
      \midrule

      % Conv 4 results
       \multirow{6}{*}{\textbf{Conv4}} & ASLP (pre-pruning) & 82.28 $\pm$ 0.22 & 83.30 $\pm$ 0.56 & 82.80 $\pm$ 0.27 & 83.60 $\pm$ 0.59 & 70.71 $\pm$ 0.67 & 68.64 $\pm$ 1.34 & 71.68 $\pm$ 0.48 & 71.04 $\pm$ 0.79 \\
        & \cite{DBLP:conf/nips/ZhouLLY19} (pre-pruning) & - & - & - & - & 68.14 $\pm$ 0.71 & 67.55 $\pm$ 0.67 & 58.01 $\pm$ 2.27 & 53.89 $\pm$ 4.81 \\
      \cmidrule(lr){2-10}
      % Conv 4 post pruining
        & ASLP (thresholding) & \textbf{83.03 $\pm$ 0.31} & \textbf{83.73 $\pm$ 0.46} & \textbf{83.59 $\pm$ 0.29} & \textbf{84.06 $\pm$ 0.31} & \textbf{71.64 $\pm$ 0.36} & \textbf{69.74 $\pm$ 1.37} & \textbf{72.85 $\pm$ 0.48} & \textbf{72.08 $\pm$ 0.62} \\
        & ASLP (averaging)& 82.29 $\pm$ 0.25 & 83.22 $\pm$ 0.56 & 82.79 $\pm$ 0.30 & 83.46 $\pm$ 0.49 & 70.88 $\pm$ 0.47 & 68.77 $\pm$ 1.42 & 71.82 $\pm$ 0.53 & 71.09 $\pm$ 0.69 \\
        & \cite{DBLP:conf/nips/ZhouLLY19} (averaging)& - & - & - & - & 68.09 $\pm$ 0.84 & 67.48 $\pm$ 0.52 & 58.13 $\pm$ 2.39 & 53.84 $\pm$ 5.00 \\
        & \cite{DBLP:conf/cvpr/RamanujanWKFR20} ($k=50\%$) & 82.38 $\pm$ 0.29 & 83.61 $\pm$ 0.38 & 81.71 $\pm$ 0.59 & 83.55 $\pm$ 0.32 & - & - & - & - \\
      \midrule

      % Conv 6 results
       \multirow{6}{*}{\textbf{Conv6}} & ASLP (pre-pruning) & 84.40 $\pm$ 0.39 & 85.58 $\pm$ 0.30 & 84.59 $\pm$ 0.35 & 85.62 $\pm$ 0.29 & 72.73 $\pm$ 0.70 & 69.52 $\pm$ 1.79 & 75.20 $\pm$ 1.05 & 74.55 $\pm$ 0.99 \\
        & \cite{DBLP:conf/nips/ZhouLLY19} (pre-pruning) & - & - & - & - & 70.74 $\pm$ 1.09 & 69.06 $\pm$ 1.93 & 45.01 $\pm$ 17.09 & 36.65 $\pm$ 15.43 \\
      \cmidrule(lr){2-10}
      % Conv 6 post pruining
        & ASLP (thresholding) & \textbf{84.98 $\pm$ 0.33} & \textbf{86.49 $\pm$ 0.36} & \textbf{85.32 $\pm$ 0.27} & \textbf{86.21 $\pm$ 0.34} & \textbf{73.32 $\pm$ 0.42} & \textbf{69.83 $\pm$ 1.46} & \textbf{76.20 $\pm$ 0.91} & \textbf{75.30 $\pm$ 0.89} \\
        & ASLP (averaging)& 84.24 $\pm$ 0.28 & 85.67 $\pm$ 0.34 & 84.51 $\pm$ 0.35 & 85.49 $\pm$ 0.38 & 72.62 $\pm$ 0.57 & 69.53 $\pm$ 1.68 & 75.24 $\pm$ 0.69 & 74.50 $\pm$ 0.96 \\
        & \cite{DBLP:conf/nips/ZhouLLY19} (averaging) & - & - & - & - & 70.71 $\pm$ 0.98 & 69.16 $\pm$ 1.92 & 44.77 $\pm$ 17.02 & 36.59 $\pm$ 15.32 \\
        & \cite{DBLP:conf/cvpr/RamanujanWKFR20} ($k=50\%$) & 84.67 $\pm$ 0.35 & 85.87 $\pm$ 0.13 & 84.37 $\pm$ 0.58 & 85.84 $\pm$ 0.51 & - & - & - & - \\
      \bottomrule
    \end{tabular}
  }
  \caption{Comparison of ASLP's performance with Edge-Popup and Supermask
  \cite{DBLP:conf/cvpr/RamanujanWKFR20,DBLP:conf/nips/ZhouLLY19} on CIFAR10
  using various configurations. We use the configurations tested by the author
  of the aforementioned articles. Performance measures reported are accuracy and
  are presented with and without data augmentation, weight rescaling (WR), and
  the signed constant weight distribution (SC). We report performances for both
  "thresholding" and "averaging" setups for our method. For edge-popup, we use
  the best $k$ value for Conv\{2,4,6\} repported in
  \cite{DBLP:conf/cvpr/RamanujanWKFR20}.  Across all setups, our method ASLP
  outperforms Edge-Popup and Supermask}
  \label{tab:chap2:con_performances_comparison_cifar10}
  
\end{table}



% pre prun
\begin{table}
  \centering
  \resizebox{16.5cm}{!}{
    \begin{tabular}{lllllllllll}
      \cmidrule[\heavyrulewidth]{3-10}
      &  & \multicolumn{4}{c}{\textbf{w/ data augmentation}} & \multicolumn{4}{c}{\textbf{w/o augmentation}} \\
      &  &  $\varnothing$ & \textbf{SC} & \textbf{WR} & \textbf{WR+SC} & $\varnothing$ & \textbf{SC} & \textbf{WR} & \textbf{WR+SC} \\
      \toprule
      \multirow{18}{*}{}
      \multirow{6}{*}{Conv2} & ASLP (pre-pruning) & 38.45 $\pm$ 0.59 & 38.20 $\pm$ 0.96 & 41.23 $\pm$ 0.78 & 41.33 $\pm$ 0.77 & 38.46 $\pm$ 0.87 & 38.58 $\pm$ 1.05 & 41.53 $\pm$ 0.38 & 41.53 $\pm$ 0.46 \\
      & \cite{DBLP:conf/nips/ZhouLLY19} (pre-pruning) & - & - & - & - & 38.01 $\pm$ 1.18 & 37.40 $\pm$ 0.70 & 26.25 $\pm$ 2.41 & 23.63 $\pm$ 1.38 \\ 
      \cmidrule(lr){2-10}
      & ASLP (thresholding) & \textbf{38.64 $\pm$ 0.92} & 38.31 $\pm$ 0.75 & \textbf{41.81 $\pm$ 0.84} & \textbf{42.06 $\pm$ 0.76} & \textbf{38.72 $\pm$ 0.59} & 38.64 $\pm$ 1.23 & \textbf{42.42 $\pm$ 0.30} & \textbf{41.95 $\pm$ 0.68} \\
      & ASLP (averaging) & 38.49 $\pm$ 0.61 & 38.18 $\pm$ 0.81 & 41.12 $\pm$ 0.66 & 41.17 $\pm$ 0.54 & 38.40 $\pm$ 0.81 & \textbf{38.71 $\pm$ 1.05} & 41.66 $\pm$ 0.39 & 41.42 $\pm$ 0.55 \\
      & \cite{DBLP:conf/nips/ZhouLLY19} (averaging) & - & - & - & - & 38.09 $\pm$ 1.03 & 37.28 $\pm$ 0.47 & 26.03 $\pm$ 2.23 & 23.49 $\pm$ 1.36 \\
      & \cite{DBLP:conf/cvpr/RamanujanWKFR20} ($k=50\%$) & 38.47 $\pm$ 0.46 & \textbf{39.83 $\pm$ 0.46} & 38.57 $\pm$ 0.59 & 39.87 $\pm$ 0.78 & - & - & - & - \\
      \midrule
      \multirow{6}{*}{Conv4} & ASLP (pre-pruning) & 47.20 $\pm$ 1.08 & 48.82 $\pm$ 0.73 & 49.22 $\pm$ 0.24 & 50.15 $\pm$ 0.89 & 46.93 $\pm$ 0.39 & 48.52 $\pm$ 0.57 & 49.75 $\pm$ 0.54 & 50.14 $\pm$ 0.86 \\
      & \cite{DBLP:conf/nips/ZhouLLY19} (pre-pruning) & - & - & - & - & 45.73 $\pm$ 1.00 & 47.88 $\pm$ 1.04 & 27.84 $\pm$ 2.28 & 27.77 $\pm$ 5.17 \\
      \cmidrule(lr){2-10}
      & ASLP (thresholding) & \textbf{47.78 $\pm$ 1.18} & 49.33 $\pm$ 0.77 & \textbf{50.33 $\pm$ 0.39} & \textbf{51.49 $\pm$ 0.43} & \textbf{47.56 $\pm$ 0.36} & \textbf{49.30 $\pm$ 0.54} & \textbf{50.39 $\pm$ 0.58} & \textbf{51.16 $\pm$ 0.94} \\
      & ASLP (averaging) & 47.18 $\pm$ 1.17 & 48.78 $\pm$ 0.79 & 49.39 $\pm$ 0.30 & 50.17 $\pm$ 0.50 & 46.89 $\pm$ 0.52 & 48.74 $\pm$ 0.47 & 49.55 $\pm$ 0.57 & 50.23 $\pm$ 0.87 \\
      & \cite{DBLP:conf/nips/ZhouLLY19} (averaging) & - & - & - & - & 45.84 $\pm$ 1.01 & 47.72 $\pm$ 0.75 & 27.70 $\pm$ 2.41 & 27.53 $\pm$ 5.20 \\
      & \cite{DBLP:conf/cvpr/RamanujanWKFR20} ($k=50\%$) & 47.75 $\pm$ 0.63 & \textbf{50.16 $\pm$ 0.47} & 48.20 $\pm$ 0.72 & 50.02 $\pm$ 0.65 & - & - & - & - \\
      \midrule
      \multirow{6}{*}{Conv6} & ASLP (pre-pruning) & 50.25 $\pm$ 1.08 & 51.65 $\pm$ 0.76 & 50.61 $\pm$ 0.46 & 51.67 $\pm$ 0.34 & 50.50 $\pm$ 0.39 & 51.96 $\pm$ 0.39 & 50.41 $\pm$ 0.38 & 51.87 $\pm$ 0.54 \\
      & \cite{DBLP:conf/nips/ZhouLLY19} (pre-pruning) & - & - & - & 1.80 $\pm$ 1.50 & 49.21 $\pm$ 0.84 & 50.63 $\pm$ 0.73 & 2.56 $\pm$ 1.66 & 9.16 $\pm$ 5.51 \\
      \cmidrule(lr){2-10}
      & ASLP (thresholding) & 51.09 $\pm$ 0.92 & 53.00 $\pm$ 0.52 & \textbf{51.70 $\pm$ 0.48} & 52.85 $\pm$ 0.50 & \textbf{51.43 $\pm$ 0.41} & \textbf{53.10 $\pm$ 0.27} & \textbf{51.52 $\pm$ 0.35} & \textbf{53.22 $\pm$ 0.54} \\
      & ASLP (averaging) & 50.22 $\pm$ 1.09 & 51.72 $\pm$ 0.73 & 50.56 $\pm$ 0.33 & 51.59 $\pm$ 0.24 & 50.47 $\pm$ 0.42 & 52.00 $\pm$ 0.27 & 50.38 $\pm$ 0.33 & 51.82 $\pm$ 0.34 \\
      & \cite{DBLP:conf/nips/ZhouLLY19} (averaging) & - & - & - & - & 49.19 $\pm$ 0.75 & 50.66 $\pm$ 0.47 & 2.54 $\pm$ 1.63 & 9.21 $\pm$ 5.50 \\
      & \cite{DBLP:conf/cvpr/RamanujanWKFR20} ($k=50\%$) & \textbf{51.13 $\pm$ 0.39} & \textbf{53.48 $\pm$ 0.51} & 51.06 $\pm$ 1.11 & \textbf{54.01 $\pm$ 0.35} & - & - & - & - \\
      \bottomrule
    \end{tabular}
  }
  \caption{Comparison of ASLP's performance with Edge-Popup and Supermask
  \cite{DBLP:conf/cvpr/RamanujanWKFR20,DBLP:conf/nips/ZhouLLY19} on CIFAR100
  using various configurations. We use the configurations tested by the author
  of the aforementioned articles. Performance measures reported are accuracy and
  are presented with and without data augmentation, weight rescaling (WR), and
  the signed constant weight distribution (SC). Pre-pruning results are reported
  for ASLP and Supermask. Since pruning is baked in the Edge-Popup method, we do
  not report a specific pre-pruning performance for it. We report performances for both
  "thresholding" and "averaging" setups for our method. For edge-popup, we use
  the best $k$ value for Conv\{2,4,6\} repported in
  \cite{DBLP:conf/cvpr/RamanujanWKFR20}. For smaller networks, ASLP outperforms
  the other methods, with the exception of the SC setup for Conv2 and Conv4.
  However, for Conv6, ASLP's performance is superior when data augmentation is
  disabled, while edge-popup achieves better results with data augmentation
  enabled (except for the WR setup).}
  \label{tab:chap2:con_performances_comparison_cifar100}
\end{table}

\chapter{Conclusion and Perspectives}

\section{Summary of contributions}

In this thesis, we addressed the issue of \aclp{DNN} compression, specifically
from the perspective of pruning, and in particular, we focused on the problem of
performance drop after pruning. We proposed several solutions to address this
issue and ultimately questioned the very necessity of training the weights. We
summarised our contributions in the following paragraphs.\\

\noindent \textbf{Budget-aware pruning with weight reparametrisation.} Pruning a
network post-training introduces a performance drop that needs to be compensated
for with fine-tuning. In \cref{chap:chapter1} we propose a budget-aware pruning
method based on a weight reparametrisation. Respecting a budget throughout
training allows for joint optimisation of the weights and the topology.
Moreover, by controlling the number of parameters that will remain, it
encourages the network not to use more capacity and therefore weights than what
will be allowed once pruning is enforced.  To reach this goal, we introduce in
\cref{chap:chapter1} two main components that work together. On the one hand, a
budget regularisation loss that computes the current weight budget at each
training step, guiding the optimisation process to adhere to it. On the other
hand, a weight reparametrisation that embeds the saliency of the weights in
their expression and thereby soft-prune them during training. Both components
are based on our reparametrisation function that acts as a surrogate $\ell_0$
norm and have been carefully designed to be differentiable and numerically
stable.\\

We validated our approach by comparing our method against magnitude pruning with
and without fine-tuning on various datasets and network architectures. Our
method performs consistently better than magnitude pruning without fine-tuning
and, for almost all tested pruning rates, better than magnitude pruning with
fine-tuning. We also validated the relevance of each component of our method
individually in a set of comparative experiments. Finally, we provided
experimental results to discuss and support the choice of the mixing coefficient
and tested our method on trained and pruned initialisation to show the
importance of budget enforcement and weight reparametrisation, even on already
pruned networks when they undergo fine-tuning.\\

\noindent \textbf{Pruning without weight training with stochastic sampling.}
When it comes to estimating the saliency of weights, the general approach is to
derive an indicator based on their value, such as magnitude pruning which
considers the absolute value of the weight as its saliency. However, these
approaches, by design, cannot treat differently two connections with the same
weight value. In \cref{chap:chapter2}, we proposed a new stochastic approach to
extract lightweight subnetworks from a large untrained network. This approach
estimates the importance of a weight based on trained masks which are auxiliary
variables that represent their associated weight saliency and are consequently
not bound to the value of the weights. Furthermore, to also tackle the
aforementioned issue with the necessity to fine-tune pruned networks, the method
detailed in \cref{chap:chapter2} does not require any weight training and relies
purely on topology selection through the optimisation of the auxiliary masks.
This method works by stochastically sampling topologies from a large untrained
network, based on the value of the masks, interpreted as probabilities of
selection of the corresponding weight. These sampled topologies are evaluated to
eventually identify a subnetwork with compelling performances. The subnetwork is
extracted by pruning the weights of the large network identified as redundant
from the larger network, with no performance drop. To achieve this, we
introduced two components called \acf{ASLP} and \acf{SR}. The former is a
computationally efficient and numerical stable technique that relies on \acl{GS}
to train the masks in a stochastic context. The former is an efficient
learnt-based weight rescaling mechanism that allows the network to rescale the
weight distributions in order to mitigate the disruption of the weight
distribution statistics caused by the pruning. We also introduce a thresholding
strategy responsible for pruning the weights, that allows to effectively
\emph{freeze} the topology.\\

We validated our approach by comparing our method against other state-of-the-art
methods on various datasets and network architectures. Our method performs
better than those other methods in most tested scenarios, offering higher
accuracy. We also provided experimental results to validate the relevance of our
\ac{SR} mechanism and thresholding strategy, support our choice of learning rate
and finally, show that our method is robust to modification of the pruning rate
post-training. Finally, our code has been made publicly available \footnote{Code
available at: \url{https://github.com/N0ciple/ASLP}} and contains the
instructions to reproduce our results as well as a reimplementation of the
state-of-the-art method we benchmark against in PyTorch.\\ 

\section{Perspectives}

In this section, we discuss the perspectives and future works that could be
undertaken to improve the methods we proposed in this thesis as well as push
forward the findings we made.\\

\noindent \textbf{Experimental validation on larger datasets and architectures.}
In our experiments, we chose to focus on results reliability and therefore we
chose to run every configuration for every experiment at least 5 times to
average the results and provide their standard deviation. This choice was made
to avoid drawing conclusions based on a single run that could be an outlier.
However, this choice comes at the cost of computational time and resources,
thereby limiting the scale of datasets and architectures we could evaluate.\\

Futur works and development efforts could target the evaluation of our method on
larger networks and datasets, namely the ResNet-50 architecture
\cite{DBLP:conf/cvpr/HeZRS16}, Vision Transformers
\cite{DBLP:conf/iclr/DosovitskiyB0WZ21}, both in combination with the ImageNet
dataset \cite{DBLP:journals/ijcv/RussakovskyDSKS15}. A larger dataset like
ImageNet would allow to sample more topologies and therefore explore the
topology space more thoroughly.\\

\noindent \textbf{Structured Pruning.} The methods introduced in
\cref{chap:chapter1,chap:chapter2} are unstructured pruning methods, meaning
that they prune weights individually which is a flexible approach that allows to
reach high pruning rates. However, the speedup obtained by unstructured pruning
is not straightforward and could necessitate additional optimisations. On the
other hand, structured pruning methods, which prune weights in groups, yield
networks with lower pruning rates but with a regular structure. This regularity
can be exploited to obtain a more straightforward speedup in the most popular
Deep Learning frameworks
\cite{DBLP:conf/nips/PaszkeGMLBCKLGA19,DBLP:journals/corr/AbadiABBCCCDDDG16}.\\

Our method, \acf{ASLP}, could benefit from a structured pruning approach. In
addition to the aforementioned network regularity, using a structured approach
could allow to reduce the number of masks to train. Instead of training a mask
per weight, it is possible to train a mask per group of weights. This could lead
to significant memory savings and speedups during training since the sampling
operation takes a heavy toll on the \ac{GPU}. Our preliminary works on a
semi-structured approach, where we start by pruning the network with a
structured approach and then perform an unstructured pruning step afterwards,
limits the sampling: we only sample the weights that are not pruned by the
structured part. This approach is promising since it can reduce on average the
number of masks to sample. However, the theoretical sampling speedup is not
observed in practice due to memory latency caused by partial access to the
masks. A careful reimplementation of the mask partial selection and sampling
logic could resolve this issue and allow for faster sampling.\\


\noindent \textbf{Controlling mask magnitude.} In \cref{chap:chapter2}, we used
a learning rate value of 50 that is several orders of magnitude higher than
standard learning rates used in baselines training \cite{nvidia-baselines}. This
choice is motivated and explained in \cref{sec:chap2:impact_learning_rate}.
However, this high learning rate together with vanishingly small gradients as
masks move away from the origin (as explained in
\cref{sec:chap2:stochastic-sampling}) can lead to masks being stuck at their
high or low value and therefore being effectively frozen. \\

Adding a regularisation term to the loss function that penalises masks with
extreme values, or any other mechanism that can limit the magnitude of the masks
could help to mitigate this issue and prevent a mask from being frozen. Our
preliminary experiments with naive regularisation loss show improved results in
the aforementioned semi-supervised setup.\\


\noindent \textbf{Better initialisation scheme.} The \ac{ASLP} method introduced
in \cref{chap:chapter2} extracts a lightweight and effective neural network from
a large untrained one. The weights of the large network are initialised with
state-of-the-art methods such as Kaiming initialisation
\cite{DBLP:conf/iccv/HeZRS15} and are not modified. However, these
initialisations are designed with weight training in mind and might not be
optimal for the \ac{ASLP} method which does not train the weights.\\

A better initialisation scheme could be designed to improve the performance of
the \ac{ASLP} method. This initialisation scheme could be inspired by trained
weights distributions and could be designed to be more robust to the pruning and
sampling operation.\\

\noindent \textbf{Training through pruning.} \ac{ASLP} and experiments conducted
in \cref{chap:chapter2} showed that it is possible to achieve compelling
performances without training the weights. This raises the question of the very
necessity of training the weights and opens the way for new research directions
that investigate the possibility of training a network through pruning.
Furthermore, in this context, the word \emph{training} is to be understood
\emph{lato sensu} and could include any strategy that selects a topology, not
necessarily strategies that rely on gradient-based mask training as we proposed.
\appendix

\chapter{Appendix}

\section{Relationship between \acl{MAC} Operations and the Number of Parameters}\label{sec:appendix:macs}

For a convolution operation, the number of parameters of a layer is not
representative of its computational complexity. Each kernel has to be spatially
convolved with the entire input. The resulting convolutional complexity is, for
one part, highly dependent on the input size, and for the other part, higher
than the number of parameters.\\

Without loss of generality, consider a 2D square matrix $M$ of size $m \times
m$, and a 2D convolution kernel $K$ of size $k \times k$, with $k<m$. The output
of the spatial convolution of $M$ by $K$ is denoted $O$. The matrix $O$ is of
size $(m-k+1) \times (m-k+1)$. Each one of the $(m-k+1)^2$ elements of $O$
necessitates $k^2$ multiplications and $k^2-1$ additions. For the sake of
simplicity, we will consider $k^2$ \acfp{MAC} operations per element of $O$. The
total number of \acp{MAC} needed to compute $O$, denoted $\mu$, is therefore:\\
$$
\mu = (m-k+1)^2 \times k^2
$$\\

Considering that there are $k^2$ elements in $K$, the ratio between the number
of \acp{MAC} and the number of parameters is:

$$
\frac{\mu}{k^2} = (m-k+1)^2
$$\\


Since $k<m$, the ratio $\frac{\mu}{k^2}$ is always greater than $1$, and grows
quadratically with $m$. Therefore, for a 2D convolution, the computational
complexity can roughly be estimated as $(m-k+1)^2$ times the number of
parameters in the convolution kernel.\\

\section{Annihiliation of the Mixing Coefficient \texorpdfstring{$\lambda$}{lambda}}
\label{sec:appendix:annihilation}

\begin{figure}[!h]
\centering
\subfloat[increasing\label{fig:appendix:annihilation_increasing}]{
    \includegraphics[width=0.49\textwidth]{appendix/assets/annihiling_lambda_max.pdf}}
\subfloat[decreasing\label{fig:appendix:annihilation_decreasing}]{
    \includegraphics[width=0.49\textwidth]{appendix/assets/annihiling_lambda_min.pdf}}
\end{figure}
\newpage

%récupérer les citation avec "/footnotemark"
\nocite{*}

%choix du style de la biblio
\bibliographystyle{plainnat}
%ajout de la bibliographie dans la table des matières
\addcontentsline{toc}{chapter}{Bibliography}
%inclusion de la biblio
\bibliography{bibliography.bib}
%voir wiki pour plus d'information sur la syntaxe des entrées d'une bibliographie


\end{document}