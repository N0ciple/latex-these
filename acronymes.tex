% acronym style

\definecolor{mycolor}{HTML}{363687} % define your color using a hex code


\acsetup{
    make-links = true,
}

\RenewAcroTemplate{long-short}{%
  \acroiffirstTF{%
    \acrowrite{long}%
    \acspace(%
    \acroifT{foreign}{\acrowrite{foreign}, }%
    \textcolor{mycolor}{\acrowrite{short}}% <-- Change is here
    \acroifT{alt}{ \acrotranslate{or} \acrowrite{alt}}%
    \acrogroupcite
    )%
  }%
  {\acrowrite{short}}%
}

\RenewAcroTemplate{short}{%
\textcolor{mycolor}{\acrowrite{short}}%
}


% acronyms

\DeclareAcronym{BN}{
    short=BN,
    long=Batch Normalisation,
}

\DeclareAcronym{CNN}{
    short=CNN,
    long=Convolutional Neural Network,
    long-plural=s,
}

\DeclareAcronym{GPU}{
    short=GPU,
    long=Graphics Processing Unit,
    long-plural=s,
}

\DeclareAcronym{TPU}{
    short=TPU,
    long=Tensor Processing Unit,
    long-plural=s,
}

\DeclareAcronym{CPU}{
    short=CPU,
    long=Central Processing Unit,
    long-plural=s,
}

\DeclareAcronym{amc}{
    short=AMC,
    long=AutoML for Model Compression,
}

\DeclareAcronym{nan}{
    short=\texttt{NaN},
    long=Not a Number,
}

\DeclareAcronym{FLOP}{
    short=FLOP,
    long=Floating Point Operation,
    long-plural=s,
}

\DeclareAcronym{ReLU}{
    short=ReLU,
    long=Rectified Linear Unit,
    long-plural=s,
}

\DeclareAcronym{STE}{
    short=STE,
    long=Straight Through Estimator,
}

\DeclareAcronym{LTH}{
    short=LTH,
    long=Lottery Ticket Hypothesis,
}

\DeclareAcronym{ASLP}{
    short=ASLP,
    long=Arbitrarily Shifted Log Parametrisation,
}

\DeclareAcronym{SGD}{
    short=SGD,
    long=Stochastic Gradient Descent,
}
\DeclareAcronym{LT}{
    short=LT,
    long=Lottery Ticket,
    long-plural=s,
}

\DeclareAcronym{GS}{
    short=GS,
    long=Gumbel-Softmax,
}

\DeclareAcronym{STGS}{
    short=STGS,
    long=Straight Through Gumbel-Softmax,
}

\DeclareAcronym{DWR}{
    short=DWR,
    long=Dynamic Weight Rescaling,
}

\DeclareAcronym{SR}{
    short=SR,
    long=Smart Rescale,
}

\DeclareAcronym{WR}{
    short=WR,
    long=Weight Rescaling,
}

\DeclareAcronym{SC}{
    short=SC,
    long=Signed Constant,
}

\DeclareAcronym{FS}{
    short=FS,
    long=Fan Scaling,
}

\DeclareAcronym{NAS}{
    short=NAS,
    long=Neural Architecture Search,
}

\DeclareAcronym{KD}{
    short=KD,
    long=Knowledge Distillation,
}

\DeclareAcronym{MAC}{
    short=MAC,
    long=Multiply-Accumulate,
}

\DeclareAcronym{SE}{
    short=SE,
    long=Squeeze-and-Excitation,
}

\DeclareAcronym{FFT}{
    short=FFT,
    long=Fast Fourier Transform,
}

\DeclareAcronym{FPGA}{
    short=FPGA,
    long=Field Programmable Gate Array,
}

\DeclareAcronym{ANN}{
    short=ANN,
    long=Artificial Neural Network,
    long-plural=s,
}

\DeclareAcronym{DNN}{
    short=DNN,
    long=Deep Neural Network,
    long-plural=s,
}

\DeclareAcronym{MLP}{
    short=MLP,
    long=Multilayer Perceptron,
    long-plural=s,
}

\DeclareAcronym{FP32}{
    short=FP32,
    long=single-precision floating-point format,
}

\DeclareAcronym{PTQ}{
    short=PTQ,
    long=Post-Training Quantisation,
}

\DeclareAcronym{QAT}{
    short=QAT,
    long=Quantisation-Aware Training,
}

\DeclareAcronym{TA}{
    short=TA,
    long=Teacher Assistant,
    long-plural=s,
}

\DeclareAcronym{SKD}{
    short=SKD,
    long=Scaled Kaiming distribution
}

\DeclareAcronym{DAG}{
    short=DAG,
    long=Directed Acyclic Graph,
    long-plural=s,
}

\DeclareAcronym{FC}{
    short=FC,
    long=Fully Connected,
}