

% abstract

\begin{abstract}

    Thanks to the miniaturisation of electronics, embedded devices have become
    more and more ubiquitous, since the 2010s, realising various tasks all
    around us. As their usage is developing, there is a growing demand for these
    devices to process data and make complex decisions efficiently. Deep neural
    networks are powerful tools to achieve this goal, however, these networks
    are often too heavy and complex to fit on embedded devices. Thus, there is a
    compelling need to devise methods to compress these large networks without
    significantly compromising their efficacy. This PhD thesis introduces two
    innovative methods, centred around the concept of pruning, aiming to
    compress neural networks while ensuring minimal impact on their accuracy.

    This PhD thesis first introduces a budget-aware method for compressing large
    neural networks with weight reparametrisation and budget loss that does not
    require fine-tuning. Traditional pruning methods often rely on post-training
    saliency indicators to remove weights, disregarding the targeted pruning
    rate. Our approach integrates a budget loss, driving the pruning process
    towards a specific value during training, thereby achieving a joint
    optimisation of topology and weights. By soft-pruning the smallest weights
    using weight reparametrisation, our method significantly mitigates accuracy
    degradation in comparison to traditional pruning techniques. We show the
    effectiveness of our approach across various datasets and architectures.

    This PhD thesis later focuses on the extraction of effective subnetworks
    without weight training. Our goal is to identify the best subnetwork
    topology in a large network without optimising its weights while still
    delivering compelling performance. This is achieved using our novel
    Arbitrarily Shifted Log Parametrisation, which serves as a differentiable
    relaxation of discrete topology sampling, enabling the training of masks
    that represent the probability of selection of the weights. Alongside, a
    weight rescaling mechanism (referred to as Smart Rescale) is also
    introduced, which allows enhancing the performance of the extracted
    subnetworks as well as speeding up their training. Our proposed approach
    also finds the optimal pruning rate after one training pass, thereby
    circumventing computationally expensive gird-search and training across
    various pruning rates. As shown through comprehensive experiments, our
    method consistently outperforms closely related state-of-the-art techniques
    and allows designing lightweight networks which can reach high sparsity
    levels without significant loss in accuracy.

\end{abstract}

