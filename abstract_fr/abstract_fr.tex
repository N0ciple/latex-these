% abstract
\chapter*{Résumé}

Grâce à la miniaturisation de l'électronique, les dispositifs embarqués sont
devenus de plus en plus omniprésents depuis les années 2010, réalisant diverses
tâches tout autour de nous. À mesure que leur utilisation se développe, la
demande pour des dispositifs traitant les données et prennant des décisions
complexes de manière efficace augmente. Les réseaux de neurones profonds sont
des outils puissants pour atteindre cet objectif, cependant, ces réseaux sont
souvent trop lourds et complexes pour être intégrés dans des appareils
embarqués. C'est pourquoi il est impératif de concevoir des méthodes pour
compresser ces grands réseaux de neurones sans compromettre significativement
leur performance. Cette thèse de doctorat introduit deux méthodes innovantes,
centrées autour du concept d'élagage, visant à compresser les réseaux de
neurones tout en assurant un impact minimal sur leur précision.

Cette thèse de doctorat introduit d'abord une méthode prenant en compte le
budget pour compresser de grands réseaux de neurones à l'aide de
reparamétrisation des poids et d'une fonction de coût budgétaire, le tout ne
nécessitant pas de fine-tuning. Les méthodes d'élagage traditionnelles
s'appuient souvent sur des indicateurs de saillance post-entraînement pour
supprimer les poids, négligeant le taux d'élagage ciblé. Notre approche intègre
une fonction de coût budgétaire, guidant le processus d'élagage vers une valeur
spécifique de parcimonie pendant l'entraînement, réalisant ainsi une
optimisation conjointe de la topologie et des poids. En simulant l'élaguage des
poids les plus petits en cours d'entraînement grâce à la reparamétrisation des
poids, notre méthode atténue significativement la perte de la précision par
rapport aux techniques d'élagage traditionnelles. Nous démontrons l'efficacité
de notre approche à travers divers ensembles de données et architectures.

Cette thèse de doctorat se concentre ensuite sur l'extraction de sous-réseaux
efficaces, sans entraînement des poids. Notre objectif est d'identifier la
meilleure topologie d'un sous-réseau dans un grand réseau sans en optimiser les
poids tout en offrant des performances convaincantes. Ceci est réalisé grâce à
notre méthode appelée Arbitrarily Shifted Log-Parametrisation, qui sert à
échantillonner des topologies discrètes de manière différentiable, permettant
l'entraînement de masques représentant la probabilité de sélection des poids.
Parallèlement, un mécanisme de recalibrage des poids (appelé Smart Rescale) est
également introduit, permettant d'améliorer les performances des sous-réseaux
extraits ainsi que d'accélérer leur entraînement. Notre approche proposée trouve
également le taux d'élagage optimal après un unique entraînement, évitant ainsi
la recherche exhaustive d'hyperparamètres et un entraînement pour chaque taux
d'élagage. Nous montrons à travers un ensemble expériences que notre méthode
surpasse constamment les techniques de l'état de l'art étroitement liées et
permet de concevoir des réseaux légers pouvant atteindre des niveaux élevés de
parcimonie sans perte significative de précision.

