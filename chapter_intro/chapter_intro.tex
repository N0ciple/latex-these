\chapter{Introduction}\label{chap:intro}

% - révolutions industrielles de la première à la 4ème avec pour objectifs la mecanisation du travail dans la première (détailler les 3 autres)

% - composante de la 4ème révolution industrielle : 
%     - l'intelligence artificielle permise par le développement des capacités de calcul
%     - naissance d'objects intelligents et doté de capacités de calcul embarqués (smartphones, voitures autonomes, satellites, etc...)

% - la taille des modèles est un frein au déploiement de l'intelligence artificielle sur des objets embarqués.
% - CEPENDANT: la 4ème révolution industrielle est aussi celle des objets communiquants donc on pourrait déporter les calculs sur des serveurs distants.
% - MAIS: il existe des raisons de vouloir faire du calcul embarqué:
%     - confidentialité des données
%     - distribuer la charge de calcul et ne pas assumer le coût, ni de la communication, ni du calcul
%     - plus grand réactivité (temps de latence)
%     - coût de la communication
%     - pas necessairement d'accès à internet (autonomie: ex: rover sur Mars)

From the spinning jenny, blast furnace and steam engine that sparked the first
industrial revolution to the \ac{IOT} devices that drives the fourth, the
objective of mechanising labour and optimising productivity has been a
persistent theme throughout the past centuries. The first industrial revolution,
which dates back to 1760, introduced mechanisation through the use of water
wheels and steam engines. The second industrial revolution, starting towards the
end of the \textsc{XIX}th century, is linked to the development of automobiles, crude oil
extraction and assembly lines powered by electric energy. The third industrial
revolution, also called the digital revolution took place in the second half of
the \textsc{XX}th century and brought electronics, information and communication
technology, and automated production. The Fourth Industrial Revolution, often
known as Industry 4.0, inaugurates the digital integration of production chains
as well as smart and connected devices that lead to more efficient manufacturing
systems. The fourth industrial revolution focuses on the interconnectivity of
devices and the development of their computational capabilities. This track
leads to the emergence of \ac{IOT} devices with embedded computing facilities,
such as smartphones, autonomous vehicles or satellites.\\

In parallel with these industrial revolutions, the field of \ac{AI} has seen
substantial growth and development. The term \emph{\acl{AI}} was first used at
the Dartmouth workshop in 1956 which is considered to be the founding event of
\ac{AI} as a research field \cite{dartmouth1956}. It launched decades of
research into machine learning natural language processing among others
\cite{nilsson1998artificial}. In the subsequent decades, \ac{AI} saw significant
strides, including the development of rule-based systems, called expert systems
\cite{giarratano1994expert}, in the 1970s and the early exploration of machine
learning in the 1980s \cite{rumelhart1986learning}. These developments
paralleled the third industrial revolution, setting the stage for further
advancements in \ac{AI}. In the late XXth and early XXIst centuries, coinciding
with the premises of the fourth industrial revolution, that \ac{AI} started to
draw tremendous attention from both researchers and industrials with the advent
of Deep Learning. Deep Learning is a subset of machine learning which uses
multi-layer \ac{ANN} to learn and model complex patterns in datasets in an
end-to-end fashion, bringing significant improvement over manually engineered
features. The fast development of Deep learning has been driving advancements in
various domains such as natural language processing
\cite{DBLP:conf/emnlp/BudzianowskiV19,DBLP:conf/naacl/DevlinCLT19,DBLP:conf/nips/VaswaniSPUJGKP17},
image and speech recognition
\cite{DBLP:conf/nips/KrizhevskySH12,DBLP:journals/corr/SimonyanZ14a,DBLP:conf/cvpr/HeZRS16,DBLP:journals/corr/HannunCCCDEPSSCN14,DBLP:conf/icassp/ChanJLV16,DBLP:conf/icml/AmodeiABCCCCCCD16},
text and image generation
\cite{goodfellow2020generative,karras2019style,DBLP:conf/emnlp/BudzianowskiV19},
video game playing \cite{silver2016mastering,silver2018general} and molecule
folding \cite{jumper2021highly} to name a few.\\


The conquest of new fields and the quest for performance improvement of Deep
Learning models have led to a significant increase in their computational
complexity and size, particularly regarding their number of parameters. The
sheer size of modern \ac{ANN} presents a significant barrier to their deployment
on embedded devices whose memory and computational resources are inherently
limited. To circumnavigate this hurdle, the prevalent approach is to offload
computations onto remote servers, leveraging the ever-interconnected nature of
modern \ac{IOT} devices and appliances.\\

% TODO: Travailler le pragraphe suivant en donnant des raisons détaillées de
% vouloir travailler en local. Il faut reprendre, entre autre, le point du fixme
% dans le chapitre SOTA

Nonetheless, several compelling reasons exist for conducting embedded
computation. These include ensuring data privacy, distributing the computational
load to avoid bearing the cost of communication and computation, enhancing
responsiveness by reducing latency, minimising communication costs, and not
necessarily relying on internet access (thus ensuring autonomy; e.g., a Mars
rover).\\

\section{Industrial Context}

% - Thèse CIFRE Netamo 
% - Netatmo : entreprise française spécialisée dans les objets connectés
% - en particulier les cameras de sécurité pour les particuliers qui font de la reconnaissance de visages et de la detection d'objets
% - le but est de faire du calcul embarqué sur les cameras pour éviter de devoir envoyer les données sur des serveurs distants
% - C'est utile pour les raisons évoquées à la section précédente, et en particulier pour la confidentialité des données etle fait de ne pas avoir à faire payer d'abonnement aux utilisateurs

This research work forms part of a CIFRE thesis with Netatmo, a French company
specialising in smart devices that is now part of the Legrand Group. Notably,
Netatmo has established a niche in security cameras for individual use that
perform tasks such as face recognition and object detection.\\

The objective is to conduct embedded computation on these cameras, negating the
need to send data to distant servers. This approach aligns well with the reasons
outlined in the previous section, particularly in ensuring data privacy and
averting the necessity of charging users with subscription fees.\\

As such, this research aims to delve into the compression of neural networks to
help overcome the constraints of model size, facilitating the deployment of \ac{AI}
in embedded devices. This would be a significant stride in realising the full
potential of Industry 4.0, enabling greater efficiency, productivity, and
autonomy while ensuring user privacy.\\


\section{Why Deep learning}

Deep learning, a subset of machine learning, has been at the forefront of the
most exciting capabilities in Artificial Intelligence. It employs artificial
neural networks with multiple abstraction layers to model and understand complex
patterns in datasets. Deep learning models have proven their mettle in numerous
domains and have been particularly revolutionary in the realm of computer vision
tasks.\\

Computer vision, which lies at the heart of Netatmo's smart camera
functionalities, is a field of artificial intelligence that trains computers to
interpret and understand the visual world. By utilising digital images from
cameras and videos, as well as deep learning models, machines can accurately
identify and classify objects, and then react to what they 'see.'\\

\begin{figure}[htbp]
    \centering
    \includegraphics[width=0.8\textwidth]{chapter_intro/assets/models_vs_human.pdf}
    \caption{Models top-5 accuracy on ImageNet \cite{deng2009imagenet} compared
    to human performance.}
    \label{fig:intro:models_vs_humans}
\end{figure}

Neural networks are the backbone of most advanced computer vision applications,
including Netatmo's facial recognition and object detection features. Neural
networks are computing systems with interconnected nodes that work similarly to
neurons in the human brain. Using algorithms, they can recognise patterns, and
by processing data through these networks, accurate decisions or predictions can
be made.\\

More specifically, deep learning's Convolutional Neural Networks (CNNs) have set
new benchmarks in the field of computer vision. They can process images
directly, reducing the need for manual feature extraction, and their capacity
for hierarchical feature learning makes them particularly effective for tasks
such as object recognition and classification. Their architecture is such that
they excel at recognising patterns in the spatial and hierarchical structure of
the data, enabling them to outperform other machine learning models in computer
vision tasks.\\

In essence, given the nature of tasks the Netatmo cameras are designed to
perform, deep learning and neural networks are not just a choice but a
necessity. They represent the state of the art in computer vision tasks and
continue to push the boundaries of what is possible in this domain.\\

\section{Challenges}

While deep learning, particularly through the use of Convolutional Neural
Networks (CNNs), is the technology of choice for computer vision applications,
it comes with its set of challenges that need to be addressed, especially in the
context of deploying these models on embedded devices. These challenges include,
but are not limited to, model complexity, computational requirements, power
consumption, data privacy, and real-time processing.\\

One of the most significant challenges in deploying deep learning models,
especially CNNs, on embedded devices, is the large model size. These models
often have millions of parameters, making them computationally heavy and
challenging to fit into the limited memory of embedded devices.\\

Secondly, these complex models require substantial computational resources to
operate. This translates into high energy consumption, a critical issue for
battery-powered devices. Balancing the demand for high-performing models against
the constraints of power consumption becomes a significant challenge.\\

Data privacy presents another challenge. While it is possible to process data on
remote servers, doing so can raise serious privacy concerns. In the context of
Netatmo's security cameras, the data being processed can be sensitive, as it can
involve recognising faces or detecting objects in private spaces.\\

Another consideration is the need for real-time processing. For security
cameras, it is crucial to detect and alert anomalies in real-time. High latency
can lead to delayed alerts, reducing the effectiveness of the security system.\\

Lastly, the deployment of deep learning models on embedded devices must be
robust to environmental changes and adaptable to new situations. It is crucial
to ensure that the performance of these models remains consistent in different
conditions.\\

To conclude, while deep learning and CNNs represent an exciting frontier in
computer vision applications, several challenges need to be addressed for
efficient and effective deployment on embedded devices. Addressing these
challenges forms the crux of this research, with a particular focus on model
compression techniques to reduce the size and complexity of neural networks
without significant loss in performance.\\


\section{Contributions}
\section{Outline}