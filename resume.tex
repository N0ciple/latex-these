\documentclass{report}

%====================== PACKAGES ======================

% region: packages

%\usepackage{extsize}
\usepackage[english]{babel}
\usepackage[utf8x]{inputenc}
%pour gérer les positionnement d'images
\usepackage{float}
\usepackage{amsmath}
\usepackage{graphicx}
\usepackage[colorinlistoftodos]{todonotes}
\usepackage{url}
%pour les informations sur un document compilé en PDF et les liens externes / internes
% \usepackage[hidelinks,colorlinks,linkcolor=black,]{hyperref}
\usepackage[hidelinks]{hyperref}
%pour la mise en page des tableaux
\usepackage{array}
\usepackage{tabularx}
%pour utiliser \floatbarrier
%\usepackage{placeins}
%\usepackage{floatrow}
%espacement entre les lignes
\usepackage{setspace}
%modifier la mise en page de l'abstract
\usepackage{abstract}
%police et mise en page (marges) du document
\usepackage[T1]{fontenc}
\usepackage[top=2.5cm, bottom=2.5cm, left=2.5cm, right=2.5cm]{geometry}
%Pour les galerie d'images
% \usepackage{subfig}

% endregion: packages

%====================== PACKAGES PERSO ======================

% region: packages perso

\usepackage{lmodern} % autre police d'écriture qui supporte le scaling
\usepackage[noabbrev]{cleveref} % références croisées [noabbrev] pour éviter les abréviations
\usepackage{acro} % acronymes (voir ici : https://tex.stackexchange.com/questions/492175/how-to-generate-list-of-abbreviations-in-latex)
\usepackage{mathrsfs}  % pour les polices mathématiques ex: \mathscr{X}
\usepackage{dsfont,amsfonts} % pour les polices mathématiques ex: \mathds{X}
% \usepackage{subfig} % pour les sous figures
\usepackage{lipsum} % pour générer du texte aléatoire
\usepackage{booktabs} % pour les tableaux
\usepackage{multirow} % pour les tableaux
\usepackage{etoc} % pour les tables des matières par chapitres 
\usepackage{algorithm} % pour les algorithmes
\usepackage{algorithmic} % pour les algorithmes
\usepackage[justification=justified,format=plain,labelfont=bf,width=0.9\textwidth]{caption}
\usepackage{bibentry} % Pour ajouter des refs qui ne sont pas citée directement
\usepackage{bm} % pour les polices mathématiques ex: \mathbf{X}
\usepackage{amssymb} % symboles mathématiques suplémentaires (comme \varnothing)
\usepackage{amsthm} % pour les théorèmes et surtout les preuves (voir ici: https://stackoverflow.com/questions/1449370/latex-error-environment-proof-undefined)
\usepackage[square,numbers]{natbib} % pour citer les auteurs directement [DEUX REFS] (1):https://tex.stackexchange.com/questions/69379/how-do-i-cite-author-in-latex (2) https://www.overleaf.com/learn/latex/Bibliography_management_with_natbib
\usepackage[left,modulo]{lineno} % pour numéroter les lignes
\usepackage{etoolbox} % for the \patchcmd command
\usepackage{subcaption}% pour avoir des sous tables et les sous figures (remplace subfig)
\usepackage{xcolor} % to use colors, specially for acronyms 
\usepackage{sectsty} % pour le style des sections (voir ici : https://tex.stackexchange.com/questions/311487/how-to-change-the-title-abstract-and-headings-font-to-sans-serif)
\usepackage{bbold} % pour les 1 avec 2 barres
\usepackage{stmaryrd} % pour les crochets d'intervalles

% endregion: packages perso


%====================== COMMANDES PERSO ======================
\newcommand\myfunc[5]{%
  \begingroup
  \setlength\arraycolsep{0pt}
  #1\colon\begin{array}[t]{c >{{}}c<{{}} c}
             #2 & \to & #3 \\ #4 & \mapsto & #5 
          \end{array}%
  \endgroup}

% Pour pouvoir créer des propotision (voir ici : https://www.overleaf.com/learn/latex/Theorems_and_proofs)
\newtheorem{proposition}{Proposition}[section]
\newtheorem*{remark}{Remark} % spécifique pour les remarques


% Patch abstract pour ne pas utiliser de "titlepage" et ainsi avoir une
% numérotation de page qui ne se remet pas à 1 à chaque chapitre.
% Voir ici : https://tex.stackexchange.com/a/483381
\patchcmd{\abstract}{\titlepage}{\cleardoublepage}{}{}
\patchcmd{\endabstract}{\endtitlepage}{\clearpage}{}{}
%====================== INFORMATION ET REGLES ======================

%rajouter les numérotation pour les \paragraphe et \subparagraphe
\setcounter{secnumdepth}{4}
\setcounter{tocdepth}{4}

% Réglage propres au document PDF

\hypersetup{							% Information sur le document
pdfauthor = {Robin Dupont},			% Auteurs
pdftitle = {Résumé du Manuscrit de Thèse de Robin Dupont},			% Titre du document
% pdfsubject = {Mémoire de Projet},		% Sujet
% pdfkeywords = {Tag1, Tag2, Tag3, ...},	% Mots-clefs
pdfstartview={FitH}}					% ajuste la page à la largueur de l'écran
%pdfcreator = {MikTeX},% Logiciel qui a crée le document
%pdfproducer = {}} % Société avec produit le logiciel


%====================== SECTION PERSO ====================== 
% acronym style

\definecolor{mycolor}{HTML}{363687} % define your color using a hex code


\acsetup{
    make-links = true,
}

\RenewAcroTemplate{long-short}{%
  \acroiffirstTF{%
    \acrowrite{long}%
    \acspace(%
    \acroifT{foreign}{\acrowrite{foreign}, }%
    \textcolor{mycolor}{\acrowrite{short}}% <-- Change is here
    \acroifT{alt}{ \acrotranslate{or} \acrowrite{alt}}%
    \acrogroupcite
    )%
  }%
  {\acrowrite{short}}%
}

\RenewAcroTemplate{short}{%
\textcolor{mycolor}{\acrowrite{short}}%
}


% acronyms

\DeclareAcronym{BN}{
    short=BN,
    long=Batch Normalisation,
}

\DeclareAcronym{CNN}{
    short=CNN,
    long=Convolutional Neural Network,
    long-plural=s,
}

\DeclareAcronym{GPU}{
    short=GPU,
    long=Graphics Processing Unit,
    long-plural=s,
}

\DeclareAcronym{TPU}{
    short=TPU,
    long=Tensor Processing Unit,
    long-plural=s,
}

\DeclareAcronym{CPU}{
    short=CPU,
    long=Central Processing Unit,
    long-plural=s,
}

\DeclareAcronym{amc}{
    short=AMC,
    long=AutoML for Model Compression,
}

\DeclareAcronym{nan}{
    short=\texttt{NaN},
    long=Not a Number,
}

\DeclareAcronym{FLOP}{
    short=FLOP,
    long=Floating Point Operation,
    long-plural=s,
}

\DeclareAcronym{ReLU}{
    short=ReLU,
    long=Rectified Linear Unit,
    long-plural=s,
}

\DeclareAcronym{STE}{
    short=STE,
    long=Straight Through Estimator,
}

\DeclareAcronym{LTH}{
    short=LTH,
    long=Lottery Ticket Hypothesis,
}

\DeclareAcronym{ASLP}{
    short=ASLP,
    long=Arbitrarily Shifted Log Parametrisation,
}

\DeclareAcronym{SGD}{
    short=SGD,
    long=Stochastic Gradient Descent,
}
\DeclareAcronym{LT}{
    short=LT,
    long=Lottery Ticket,
    long-plural=s,
}

\DeclareAcronym{GS}{
    short=GS,
    long=Gumbel-Softmax,
}

\DeclareAcronym{STGS}{
    short=STGS,
    long=Straight Through Gumbel-Softmax,
}

\DeclareAcronym{DWR}{
    short=DWR,
    long=Dynamic Weight Rescaling,
}

\DeclareAcronym{SR}{
    short=SR,
    long=Smart Rescale,
}

\DeclareAcronym{WR}{
    short=WR,
    long=Weight Rescaling,
}

\DeclareAcronym{SC}{
    short=SC,
    long=Signed Constant,
}

\DeclareAcronym{FS}{
    short=FS,
    long=Fan Scaling,
}

\DeclareAcronym{NAS}{
    short=NAS,
    long=Neural Architecture Search,
}

\DeclareAcronym{KD}{
    short=KD,
    long=Knowledge Distillation,
}

\DeclareAcronym{MAC}{
    short=MAC,
    long=Multiply-Accumulate,
}

\DeclareAcronym{SE}{
    short=SE,
    long=Squeeze-and-Excitation,
}

\DeclareAcronym{FFT}{
    short=FFT,
    long=Fast Fourier Transform,
}

\DeclareAcronym{FPGA}{
    short=FPGA,
    long=Field Programmable Gate Array,
}

\DeclareAcronym{ANN}{
    short=ANN,
    long=Artificial Neural Network,
    long-plural=s,
}

\DeclareAcronym{DNN}{
    short=DNN,
    long=Deep Neural Network,
    long-plural=s,
}

\DeclareAcronym{MLP}{
    short=MLP,
    long=Multilayer Perceptron,
    long-plural=s,
}

\DeclareAcronym{FP32}{
    short=FP32,
    long=single-precision floating-point format,
}

\DeclareAcronym{PTQ}{
    short=PTQ,
    long=Post-Training Quantisation,
}

\DeclareAcronym{QAT}{
    short=QAT,
    long=Quantisation-Aware Training,
}

\DeclareAcronym{TA}{
    short=TA,
    long=Teacher Assistant,
    long-plural=s,
}

\DeclareAcronym{SKD}{
    short=SKD,
    long=Scaled Kaiming distribution
}

\DeclareAcronym{DAG}{
    short=DAG,
    long=Directed Acyclic Graph,
    long-plural=s,
}

\DeclareAcronym{FC}{
    short=FC,
    long=Fully Connected,
}

\DeclareAcronym{CL}{
    short=Conv,
    long=Convolutional,
    long-plural=s,
}

\DeclareAcronym{OBS}{
    short=OBS,
    long=Optimal Brain Surgeon,
}

\DeclareAcronym{OBD}{
    short=OBD,
    long=Optimal Brain Damage,
}
% ----- Hyphenation
% \hyphenation{re-para-me-tri-za-tion}
% \hyphenation{re-para-me-tri-sa-tion}
% ----- Style des liens
% \hypersetup{
%   citecolor=blue,
% }

% ################  A CHANGER POUR REVUE ?
\linespread{1} % régler l'espacement entre les lignes

\renewcommand\linenumberfont{\normalfont\large} % régler la taille des numéros de ligne

%======================== DEBUT DU DOCUMENT ========================

\begin{document}

% TODO: adapter pour le confort de lecture
\fontsize{14}{16}\selectfont

% style des sections : fonte sans serifs
\allsectionsfont{\sffamily}

%page de garde
\title{\vspace{-3.0cm}Manuscript}
\author{Robin Dupont}
\date{}

%page blanche
%\newpage

%ne pas numéroter cette page
%\thispagestyle{empty}
%\newpage

%

% abstract

\chapter*{Abstract}
Thanks to the miniaturisation of electronics, embedded devices have become
more and more ubiquitous, since the 2010s, realising various tasks all
around us. As their usage is developing, there is a growing demand for these
devices to process data and make complex decisions efficiently. Deep neural
networks are powerful tools to achieve this goal, however, these networks
are often too heavy and complex to fit on embedded devices. Thus, there is a
compelling need to devise methods to compress these large networks without
significantly compromising their efficacy. This PhD thesis introduces two
innovative methods, centred around the concept of pruning, aiming to
compress neural networks while ensuring minimal impact on their accuracy.

This PhD thesis first introduces a budget-aware method for compressing large
neural networks with weight reparametrisation and budget loss that does not
require fine-tuning. Traditional pruning methods often rely on post-training
saliency indicators to remove weights, disregarding the targeted pruning
rate. Our approach integrates a budget loss, driving the pruning process
towards a specific value during training, thereby achieving a joint
optimisation of topology and weights. By soft-pruning the smallest weights
using weight reparametrisation, our method significantly mitigates accuracy
degradation in comparison to traditional pruning techniques. We show the
effectiveness of our approach across various datasets and architectures.

This PhD thesis later focuses on the extraction of effective subnetworks
without weight training. Our goal is to identify the best subnetwork
topology in a large network without optimising its weights while still
delivering compelling performance. This is achieved using our novel
Arbitrarily Shifted Log Parametrisation, which serves as a differentiable
relaxation of discrete topology sampling, enabling the training of masks
that represent the probability of selection of the weights. Alongside, a
weight rescaling mechanism (referred to as Smart Rescale) is also
introduced, which allows enhancing the performance of the extracted
subnetworks as well as speeding up their training. Our proposed approach
also finds the optimal pruning rate after one training pass, thereby
circumventing computationally expensive gird-search and training across
various pruning rates. As shown through comprehensive experiments, our
method consistently outperforms closely related state-of-the-art techniques
and allows designing lightweight networks which can reach high sparsity
levels without significant loss in accuracy.


\maketitle
% ----- Tables des matières, figures et tableaux  + acronymes ------
% \tableofcontents
% \listoffigures
% \listoftables
\newpage
% ---------------------------------------------------------------

\thispagestyle{empty}
\setcounter{page}{0}
%ne pas numéroter le sommaire

\newpage
% \printacronyms

%espacement entre les lignes d'un tableau
\renewcommand{\arraystretch}{1.5}

%====================== INCLUSION DES PARTIES ======================

~
\thispagestyle{empty}
%recommencer la numérotation des pages à "1"
\setcounter{page}{0}
\newpage

% ################  A CHANGER POUR REVUE ?
\linenumbers
\section{Introduction}

La quatrième révolution industrielle, aussi appellée _Industrie 4.0_, inaugure
l'intégration numérique des chaînes de production et des appareils intelligents
et connectés pour des systèmes de fabrication plus efficaces. En parallèle, la
recherche en intelligence artificielle (IA) a connu une croissance
substantielle, avec des avancées significatives en matière de systèmes à base de
règles, d'apprentissage automatique et de traitement du langage naturel. 

Les réseaux de neurones profonds (DNN) modernes offrent un potentiel
significatif pour renforcer les capacités des appareils de l'IoT. Cependant,
leur déploiement sur ces appareils présente un défi en raison des contraintes de
calcul et de mémoire. Au lieu de déplacer les calculs vers le cloud, cette thèse
plaide en faveur de la réalisation de calculs embarqués. Cela garantit une
meilleure confidentialité des données, limite les communications, augmente la
réactivité en réduisant la latence et permet une autonomie.

Le travail de cette thèse est réalisé dans le contexte industriel avec Netatmo,
une entreprise française spécialisée dans les appareils intelligents. Le but est
de faire fonctionner les DNN directement sur leurs caméras de sécurité, évitant
ainsi l'envoi de données à des serveurs distants. Cela présente un cas
convaincant pour le développement de réseaux de neurones allégés, adaptés aux
appareils de l'IoT, qui maintiennent la puissance de leurs homologues plus
grands tout en étant nettement moins gourmands en taille et en exigences de
calcul.

En raison de la nature des tâches que les caméras Netatmo sont conçues pour
effectuer, l'apprentissage profond et les DNN ne sont pas seulement un choix,
mais une nécessité. Ils représentent l'état de l'art dans les tâches de vision
par ordinateur qui surpassent les autres algorithmes et permettent une détection
et une reconnaissance d'objets précises et fiables.

Cette thèse aborde le défi de la compression des DNN à travers l'élagage, une
technique qui vise à réduire la taille d'un réseau de neurones en supprimant les
paramètres redondants ou inutiles. Elle présente également des méthodes de
rééchantillonnage stochastique des poids qui ne nécessitent pas la formation du
réseau pour déterminer la pertinence des poids. Cela permet de rechercher une
topologie à la fois légère et performante à l'intérieur du réseau original sans
formation du réseau.


\newpage

%récupérer les citation avec "/footnotemark"
\nocite{*}

%choix du style de la biblio
\bibliographystyle{plainnat}
%ajout de la bibliographie dans la table des matières
\addcontentsline{toc}{chapter}{Bibliography}
%inclusion de la biblio
\bibliography{bibliography.bib}
%voir wiki pour plus d'information sur la syntaxe des entrées d'une bibliographie


\end{document}