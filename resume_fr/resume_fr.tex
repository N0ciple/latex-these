\section{Introduction}

La quatrième révolution industrielle, aussi appellée _Industrie 4.0_, inaugure
l'intégration numérique des chaînes de production et des appareils intelligents
et connectés pour des systèmes de fabrication plus efficaces. En parallèle, la
recherche en intelligence artificielle (IA) a connu une croissance
substantielle, avec des avancées significatives en matière de systèmes à base de
règles, d'apprentissage automatique et de traitement du langage naturel. 

Les réseaux de neurones profonds (DNN) modernes offrent un potentiel
significatif pour renforcer les capacités des appareils de l'IoT. Cependant,
leur déploiement sur ces appareils présente un défi en raison des contraintes de
calcul et de mémoire. Au lieu de déplacer les calculs vers le cloud, cette thèse
plaide en faveur de la réalisation de calculs embarqués. Cela garantit une
meilleure confidentialité des données, limite les communications, augmente la
réactivité en réduisant la latence et permet une autonomie.

Le travail de cette thèse est réalisé dans le contexte industriel avec Netatmo,
une entreprise française spécialisée dans les appareils intelligents. Le but est
de faire fonctionner les DNN directement sur leurs caméras de sécurité, évitant
ainsi l'envoi de données à des serveurs distants. Cela présente un cas
convaincant pour le développement de réseaux de neurones allégés, adaptés aux
appareils de l'IoT, qui maintiennent la puissance de leurs homologues plus
grands tout en étant nettement moins gourmands en taille et en exigences de
calcul.

En raison de la nature des tâches que les caméras Netatmo sont conçues pour
effectuer, l'apprentissage profond et les DNN ne sont pas seulement un choix,
mais une nécessité. Ils représentent l'état de l'art dans les tâches de vision
par ordinateur qui surpassent les autres algorithmes et permettent une détection
et une reconnaissance d'objets précises et fiables.

Cette thèse aborde le défi de la compression des DNN à travers l'élagage, une
technique qui vise à réduire la taille d'un réseau de neurones en supprimant les
paramètres redondants ou inutiles. Elle présente également des méthodes de
rééchantillonnage stochastique des poids qui ne nécessitent pas la formation du
réseau pour déterminer la pertinence des poids. Cela permet de rechercher une
topologie à la fois légère et performante à l'intérieur du réseau original sans
formation du réseau.
