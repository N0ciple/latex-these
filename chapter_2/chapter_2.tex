\chapter{Mask Training}
\label{chap:chapter2}


\localtableofcontents


\begin{abstract}
    abstract of the chapter
\end{abstract}

\section{Introduction and related work}
% region: introduction
Deep neural networks are increasingly utilized for solving various image
processing tasks, particularly in the domain of visual category recognition.
While these models have shown significant success, their extensive computational
requirements, memory consumption, and energy consumption are notable
limitations. In light of the emergence of intelligent embedded systems with
restricted computational resources and limited energy availability, there is a
growing need to develop these models with lightweight architectures that
prioritise efficiency while maintaining high accuracy. The aim is to make these
models both "lightweight and frugal," which presents a significant challenge for
researchers in the field.\\

Creating small and efficient neural network architectures from scratch is a
viable approach to designing lightweight networks. Despite being smaller than
typical state-of-the-art networks, these networks are still able to achieve good
performance levels. SqueezeNet \cite{DBLP:journals/corr/IandolaMAHDK16} is an
example of such a network, which achieves comparable performance to AlexNet on
ImageNet but with only fifty times fewer parameters.  MobileNet
\cite{howard2017mobilenets} and MobileNetV2 \cite{DongMobileNetV2} are also
popular examples of lightweights networks that use depthwise separable
convolution. ShuffleNet \cite{ZhangShuffleNet} and ShuffleNetV2
\cite{MaShuffleNetV2} are other approaches that rely on the grouped convolution
concept and channel shuffling. CondenseNet \cite{huang2018condensenet} levrage
DenseNet-like \cite{huang2017densely} connections together with grouped
convolution to pick the most useful feature from previous layer maps.\\

Designing efficient lightweight neural networks requires significant efforts
from human experts. These networks are designed from scratch, with a focus on
reducing the number of parameters and computational complexity while maintaining
high accuracy. The design process involves careful consideration of network
architecture, layer types, hyperparameters, and optimisation techniques. The
development of such networks, therefore, requires a significant amount of time
and expertise.\\

Another way of obtaining lightweight neural network is to reduce the size of an
already existing large neural network, thourgh pruning. As discussed in
\cref{chap:chapter1}, pruning methods, either structured or unstructured, are
particularly successful at lightening large neural networks, and seek to remove
connections with the least perceptible impact on classification accuracy.
Structured pruning consists in {\it jointly} removing groups of weights, entire
channels or subnetworks \cite{DBLP:conf/iclr/0022KDSG17,
DBLP:conf/iccv/LiuLSHYZ17}, whereas unstructured pruning aims at removing
weights {\it individually}
\cite{DBLP:conf/nips/HanPTD15,DBLP:journals/corr/HanMD15}.  Unstructured pruning
has witnessed a recent surge in interest in the wake of the \ac{LTH}
\cite{DBLP:conf/iclr/FrankleC19}; an empirical study in
\cite{DBLP:conf/iclr/FrankleC19} demonstrates that large pretrained networks
encompass lightweight subnetworks, referred to as \textit{\acp{LT}}, which can
achieve comparable performance to the original large network in a similar number
of epochs when trained in isolation with initial weights taken from the large
network. To identify these \acp{LT}, the large network is trained until
convergence, followed by pruning the smallest weights based on their magnitude.
The remaining weights are then rewound to their original value, it is to say the
value they had before the training of the large network begins. This resulting
subnetwork is known as a \textit{Lottery Ticket}.
Another study \cite{DBLP:conf/iclr/LiuSZHD19} pushes that finding further and concludes that
only the topology of these subnetworks is actually important in order to reach
comparable performances. Authors in \cite{DBLP:conf/iclr/LiuSZHD19} point out
that the weights of the \acp{LT} are not important and can be randomly
initialized, provided that the optimisation procedure is carefully designed.\\

In general, extracting an efficient subnetwork is
still an open problem and is computationally demanding as this amounts to full
training of large networks (untill convergence) prior to their pruning. Existing
alternatives approach this problem using early pruning
\cite{DBLP:conf/iclr/LeeAT19,
DBLP:conf/iclr/WangZG20,DBLP:conf/nips/TanakaKYG20}, but still require to train
the weights. In contrast to these works, our proposed solution in this chapter
identifies effective subnetworks by training only their topology and without any
weights tuning.\\ 

At first, it seems counter intuitive that there exists a subnetwork in a large
network, and that this subnetwork can achieve compelling performances without
any weight training. This as been first conjectured in
\cite{DBLP:conf/cvpr/RamanujanWKFR20} as the Strong \ac{LTH}. A theoretical
analysis brought proof that such subnetworks do exist.
\cite{DBLP:conf/icml/MalachYSS20} demonstrate that a neural network of width $d$
and depth $l$ can be approximated by pruning a randomly initialized one that is
a factor $O(d^4l^2)$ wider and twice as deep. The  upper bound on the network
width has latter been improved by \cite{DBLP:conf/nips/OrseauHR20} to
$O(d^2\log(dl))$ under the assumption of a hyperbolic weight distribution. This
upper bound was eventually rafined to $O(d\log(dl))$ by
\cite{DBLP:conf/nips/PensiaRNVP20} for a broad class of distributions, including
the uniform distribution whihc is widely used for weight initialisation \cite{DBLP:conf/iccv/HeZRS15}.\\

Although their existence has been proven, no constructive proof has been
provided in order to identify the lightweight and efficient from scratch
subnetworks. In this context, Zhou et al. \cite{DBLP:conf/nips/ZhouLLY19}
proposed the first attempt to extract efficient subnetworks using stochastic
mask training. A probability of selecting each weight is defined (as the sigmoid
of a mask) and trained using the \ac{STE} \cite{DBLP:journals/corr/BengioLC13}.
During training, weights are frozen and only the masks are allowed to vary.
However, the major drawback of this method resides in the vanishing gradient of
the sigmoid which makes mask training numerically challenging. Ramanujan et al.
\cite{DBLP:conf/cvpr/RamanujanWKFR20} proposed another alternative, based on
binarized saliency indicators learned with \ac{STE}, which selects the most
prominent weights in the resulting subnetworks. Nevertheless, since this method
enfore the pruning rate \textit{a priori}, finding the pruning rate giving the
higest performances has to be made through a cumbersome and time-consuming
binary search or grid-search. \\ 

Considering the limitation of the aforementioned related  work, we introduce in
this chapter a new stochastic subnetwork selection method based on \ac{GS}.
The latter allows sampling subnetworks whose weights are the most relevant for
classification. The proposed contribution also relies on a new mask
parametrisation, dubbed as \ac{ASLP}, that allows a better conditioning of the
gradient and thereby mitigates numerical instability during mask optimisation.
Besides, when combining \ac{ASLP} with a learned weight rescaling mechanism,
training is accelerated and the accuracy of the resulting subnetworks improves
as shown later in experiments.

The rest of this chapter is organized as follows.
% // TODO: Completer "chapter organisation"


% endregion: introduction


\section{Extracting Effective Subnetworks with Gumbel-Softmax}
% region: method
% endregion: method

Considering the same formalism as in \cref{chap:chapter1}, let $f_\theta$ be a
deep neural network whose weights defined as $\theta =
\left\{\bm{w}_1,\mathellipsis, \bm{w}_L \right\}$, with $L$ being its depth,
$\bm{w}_\ell \in \mathbb{R}^{d_{\ell} \times d_{\ell-1}}$ its $\ell^\textrm{th}$
layer weights, and $d_\ell$ the dimension of $\ell$. The output of a given layer
$\ell$ is defined as \\

\begin{equation}
  \label{eqn:chap2:layer_eq}
  \mathbf{z}_{\ell} = g_\ell(\bm{w}_\ell \otimes \mathbf{z}_{\ell-1}),
\end{equation}\\

being  $g_\ell$ an activation function and $\otimes$ the usual matrix product.
Without a loss of generality, we omit the bias in the definition of
(\ref{eqn:chap2:layer_eq}).

% ------------------------------------------------------------------------------
\subsection{Stochastic Weight Sampling}
% region: stochastic-sampling
\indent Given a network $f_\theta$, weight pruning consists in removing
connections in the graph of $f_\theta$. A node in this graph refers to a
neural unit while an edge corresponds to a cross-layer connection. Pruning is
usually obtained by freezing and zeroing out  a subset of weights in $\theta$,
and this is achieved  by multiplying $\bm{w}_\ell$ by a binary mask
$\bm{m}_\ell \in \{ 0,1 \}^{\text{dim}(\bm{w}_\ell)}$. The
binary entries of $\bm{m}_\ell$ are set depending on whether the underlying
layer connections are kept or removed, so \Cref{eqn:chap2:layer_eq} becomes\\

\begin{equation}
  \label{eqn:chap2:pruned_layer_eq}
  \mathbf{z}_{\ell} = g_\ell( (\bm{m}_\ell \odot \bm{w}_\ell ) \otimes \mathbf{z}_{\ell-1} ).
\end{equation}\\

Here $\odot$ stands for the element-wise matrix product. In this definition, the
masks $\{\bm{m}_\ell\}_\ell$ are stochastic and sampled from a Bernoulli
distribution. However, sampling is not a differentiable operation, therefore,
optimizing directly $\{{\bm{m}_\ell}\}_\ell$ is not possible. To overcome this
issue, while staying in the \ac{SGD} framework the Straight Through Estimator
(STE) technique is applied together with a reparametrisation of the mask.\\

\noindent\textbf{Straight Through Estimator.} Zhou et
al. \cite{DBLP:conf/nips/ZhouLLY19} consider a Bernoulli parametrisation of
$\{\bm{m}_\ell\}_\ell$ in order to sample masks in
\Cref{eqn:chap2:pruned_layer_eq}. Since sampling is not a differentiable
operation, they rely on the \ac{STE}. It is a technique developed in
\cite{DBLP:journals/corr/BengioLC13} that enables the training of neural
networks with discrete activations, such as binary or quantized activations. The
technique involves using a continuous approximation to the non-differentiable
activation function during forward propagation, and simply passing the gradient
of the loss function through the non-differentiable activation during
backpropagation. This allows for the use of \ac{SGD} to optimize the network,
which was previously not possible with discrete activations. \\

In order to apply \ac{STE} to the problem of Bernoulli stochastic mask
sampling, the definition of $\{\bm{m}_\ell\}_\ell$ is based on another
{\it latent} parametrisation $\{\bm{\hat{m}}_\ell\}_\ell$, detailed
subsequently, and obtained by applying a sigmoid function $\sigma(.)$ to
$\bm{\hat{m}}_\ell$. This allows optimizing $\bm{\hat{m}}_\ell$  using gradient
descent by considering the following surrogate of
\cref{eqn:chap2:pruned_layer_eq} in the backward pass of the backpropagation
algorithm:\\

\begin{equation}
  \label{eqn:chap2:pruned_layer_eq2}
  \mathbf{z}_{\ell} = g_\ell( ( \sigma(\bm{\hat{m}}_\ell) \odot \bm{w}_\ell ) \otimes \mathbf{z}_{\ell-1} ).
\end{equation} \\

\noindent As a result, although masks $\bm{m}_\ell$ are sampled and thus
disconnected from the computation graph (sampling being not differentiable),
their reparametrisation $\bm{\hat{m}}_\ell$ can be updated as if they were used in
the computational graph as in \cref{eqn:chap2:pruned_layer_eq}.\\


\noindent\textbf{Gumbel-Softmax.} In what follows, we consider an alternative to
\ac{STE} based on Gumbel-Softmax \cite{DBLP:conf/iclr/JangGP17} that demonstrate
better performances for differentiable categorical sampling. Gumbel-Softmax is a
technique that can be used to approximate a discrete categorical distribution
with a continuous relaxation. Gumbel-Softmax works by using the Gumbel
distribution to add noise to a categorical distribution and then applying the
Softmax function to obtain a continuous relaxation of the discrete distribution.
The proposed method, dubbed as \ac{STGS}, is based on a variant of \ac{GS}
combined with \ac{STE}. In the forward pass, the softmax of \ac{GS} is replaced
by an argmax operator. Since this operator is not differentiable, the standard
softmax if considered in the backward pass. The argmax operator allows sampling
from a categorical distribution, as the limit of \ac{GS} (\emph{i.e.}, when its
softmax temperature approaches zero). \\

Let $z$ be a categorical random variable, associated with $n$ class probability
distribution $\mathcal{P} = [\pi_1,\dots,\pi_n]$. to sample in a
differentiable manner, the Gumbel-Softmax estimator takes as an input a vector
of log-probabilities \\

\begin{equation}
  \label{eqn:chap2:gumbel-softmax-input}
  \log(\mathcal{P}) =[\log(\pi_1),\dots, \log(\pi_n)] 
\end{equation}\\

then it disrupts the latter with a random additive noise sampled from the Gumbel
distribution, and finally takes the argmax of it, yielding a categorical
variable. More formally, following \cite{DBLP:conf/iclr/JangGP17}, the value $q$
of our categorical variable $z$ is obtained as \\

\begin{equation}
  \label{eqn:chap2:gumbel-softmax-argmax}
  q = \underset{k}{ \text{argmax}} \ [ \log(\pi_k)+g_k ],
\end{equation}\\

with $g_k$ being independent and identically distributed sampled from  the
Gumbel distribution noted $\mathcal{G}(0,1)$.\\

For mask sampling, only two possible outcomes are considered. Either the
corresponding weight is selected and its mask is set to 1, or it is pruned
from the sampled topology and its mask is set to 0. In what follows, and
unless stated otherwise, we omit $\ell$ from $\bm{w}_\ell$ and we write it for
short as $\bm{w}$. Let $\bm{w}_{ij}$ be the weight associated to the i-th and
j-th neurons respectively belonging to layers $\ell-1$ and $\ell$. Since there
are two possible outcomes for the masks define a two-class categorical distribution
$\mathcal{P}_{ij}$ on $\{0,1\}$ as\\

\begin{equation}
    \left\{ \begin{array}{c}
    \mathcal{P}_{ij}(z=1)=\pi_1^{ij} \\ 
    \mathcal{P}_{ij}(z=0)=\pi_2^{ij}
    \end{array} \right.
\end{equation}\\

with $\pi_1^{ij}=p_{ij}$ and $p_{ij}$ being the probability to keep the
underlying connection. Since there are only to mutualy exclusive outcomes,
$\pi_2^{ij}=1-p_{ij}$. In other words, keeping the weight $\bm{w}_{ij}$ (or
not) in the sampled topology is a Bernoulli trial with a probability $p_{ij}$.
Considering \Cref{eqn:chap2:gumbel-softmax-argmax}, a binary mask  $\bm{m}_{ij}$
is defined as\\

\begin{equation}
  \label{eqn:chap2:mask_value}
  \bm{m}_{ij} = 1_{\{q_{ij}=1\}}
\end{equation}\\
 
$1_{\{\}}$ being the indicator function and following
\cref{eqn:chap2:gumbel-softmax-argmax}, $q_{ij}$ is\\

\begin{equation}
  \label{eqn:chap2:q_ij_expression}
  q_{ij} = {\text{argmax}_{k \in \{1,2\}}}\big[\log(\pi_k^{ij})+g_k^{ij}\big]
\end{equation}\\

\noindent the proposed \ac{STGS} algorithm enables the learning of probabilities
$p_{ij}$ for each weight $\bm{w}_{ij}$ through \ac{SGD}. However, optimizing
$p_{ij}$ (with \ac{SGD}) raises a major issue. Since the optimisation is not
coinstrained, $p_{ij}$ can take values  larger than 1 or smaller than 0. As a
consequence, it could no longer be interpreted as a probability, moreover,
$\log(p_{ij})$ and $\log(1-p_{ij})$ would also be undefined.  On another hand,
solving constrained SGD, besides being computationally expensive and
challenging, may result into worse local minimum. In order to overcome all these
issues, one may consider an alternative reparametrisation
$p_{ij}=\sigma(\bm{\hat{m}}_{ij})$, similar to the reparametrisation in
\cite{DBLP:conf/nips/ZhouLLY19}, with $\bm{\hat{m}}_{ij}$ being a latent mask
variable and $\sigma$ the sigmoid function which bounds $p_{ij}$ in $[0,1]$.
However, this workaround suffers in practice from numerical instability in
gradient estimation and is also computationally demanding. Both limitations are
mostly due to the combination of log and sigmoid functions. \\


\noindent\textbf{Arbitrarily Shifted Log Parametrisation.} In order to solve the
issues related to \ac{STGS} in the context of this chapter, another alternative
is to consider the following expressions for $p_{ij}$:\\

\begin{equation}
  \left\{ \begin{array}{c}
      \log(p_{ij}) = \bm{\hat{m}}_{ij} \\
      \log(1-p_{ij}) = \log(1-\exp(\bm{\hat{m}}_{ij}))
    \end{array} 
  \right.
\end{equation}\\

\noindent and learn the underlying mask $\bm{\hat{m}}_{ij}$. However, this
reparametrisation is also flawed in the same way as the aforementioned sigmoid
reparametrisation. Indeed, the combination of the log and the exponential leads
to severe numerical instabilities, that necessitate a cumbersome stabilization
by adding $\varepsilon$ to prevent $p_{ij}$ and $\log(1-p_{ij})$ from being to
close to 0. Furthermore, it is important to note that the above formulation is
tremendously computationally intensive, since it necessitate the evaluation of
log and exponential for evely mask in the network.\\

In what follows, we propose an equivalent formulation which
turns out to be highly effective and numerically more stable. Considering \\

% // TODO: explliquer qu'au lieu de considerer le vecteuyr de proba, on fixe le
% deuxième terme à 0 et on considère qu'il y a une variable C qui est ajouté.

\begin{equation}
  \begin{bmatrix}
    \bm{\hat{m}}_{ij} \\
    0  \\
  \end{bmatrix}
  = \log\big(\mathcal{P}_{ij}(.)\big) + c =
  \begin{bmatrix}
    \log(p_{ij}) + c \\
    \log(1-p_{ij}) + c\\
  \end{bmatrix},
  \label{eqn:chap2:our-formulation}
\end{equation}\\

\noindent in the above definition, instead of using $\log(\mathcal{P}_{ij}(.))$
as an input for \ac{STGS} (\emph{c.f.} \cref{eqn:chap2:gumbel-softmax-input}), we
consider $\log(\mathcal{P}_{ij}(.)) + c$  as an input of the argmax in
\cref{eqn:chap2:gumbel-softmax-argmax}.\\

The constant $c \in \mathds{R}$ ensures that if $\bm{\hat{m}}_{ij} > 0$, then
$\log(p_{ij}) \in ]-\infty,0] \Leftrightarrow p_{ij} \in [0,1]$. This is
enforced by setting the second coefficient of $\mathcal{P}_{ij}$ to 0, rather
than computing it explicitly. The formulation of
\cref{eqn:chap2:our-formulation} is theoretically equivalent to the
aforementioned sigmoid reparametrisation. Indeed, solving the system of
\cref{eqn:chap2:our-formulation} w.r.t. $\bm{\hat{m}}_{ij}$ yields $p_{ij} =
\sigma(\bm{\hat{m}}_{ij})$. \noindent Differently put, the formulation in
\cref{eqn:chap2:our-formulation} considers a reparametrisation
$\bm{\hat{m}}_{ij} = \log(p_{ij})+c$ which is strictly equivalent to the sigmoid
one while being computationally more efficient and also stable. Note that adding
any arbitrary constant $c$ to the log-probability makes the outcome of
Gumbel-Softmax sampling and argmax invariant.
% endregion: stochastic-sampling


\subsection{Smart Weight rescaling}
% region: smart rescaling
% endregion: smart rescaling


\section{Experiments}
% region: experiments

\subsection{Performances}

% region: perf-tables
\begin{table}
  \centering
  \resizebox{16.5cm}{!} {
    \begin{tabular}{llcccccccc}
      \cmidrule[\heavyrulewidth]{3-10}
        &  & \multicolumn{4}{c}{\textbf{w/ data augmentation}} & \multicolumn{4}{c}{\textbf{w/o augmentation}} \\
       &  &  $\varnothing$ & \textbf{SC} & \textbf{WR} & \textbf{WR+SC} & $\varnothing$ & \textbf{SC} & \textbf{WR} & \textbf{WR+SC} \\
      \toprule

      % Conv 2 results
      \multirow{12}{*}{} \multirow{4}{*}{\textbf{Conv2}} & ASLP (thresholding) & \textbf{75.70 $\pm$ 0.30} & \textbf{75.81 $\pm$ 0.69} & \textbf{76.48 $\pm$ 0.68} & \textbf{76.92 $\pm$ 0.24} & \textbf{68.24 $\pm$ 0.14} &\textbf{ 68.11 $\pm$ 0.64} & \textbf{66.84 $\pm$ 0.46} & \textbf{66.05 $\pm$ 0.93} \\
        & ASLP (averaging)& 75.42 $\pm$ 0.25 & 75.50 $\pm$ 0.56 & 76.05 $\pm$ 0.44 & 76.44 $\pm$ 0.19 & 68.09 $\pm$ 0.35 & 67.69 $\pm$ 0.52 & 65.79 $\pm$ 0.65 & 65.35 $\pm$ 0.83 \\
        & \cite{DBLP:conf/nips/ZhouLLY19} (averaging)& - & - & - & - & 67.12 $\pm$ 0.25 & 66.34 $\pm$ 0.41 & 56.71 $\pm$ 2.99 & 56.26 $\pm$ 1.64 \\
        & \cite{DBLP:conf/cvpr/RamanujanWKFR20} ($k=50\%$) & 74.18 $\pm$ 0.76 & 75.19 $\pm$ 0.56 & 74.51 $\pm$ 0.31 & 75.45 $\pm$ 0.44 & - & - & - & - \\
      \midrule

      % Conv 4 results
       \multirow{4}{*}{\textbf{Conv4}} & ASLP (thresholding) & \textbf{83.03 $\pm$ 0.31} & \textbf{83.73 $\pm$ 0.46} & \textbf{83.59 $\pm$ 0.29} & \textbf{84.06 $\pm$ 0.31} & \textbf{71.64 $\pm$ 0.36} & \textbf{69.74 $\pm$ 1.37} & \textbf{72.85 $\pm$ 0.48} & \textbf{72.08 $\pm$ 0.62} \\
        & ASLP (averaging)& 82.29 $\pm$ 0.25 & 83.22 $\pm$ 0.56 & 82.79 $\pm$ 0.30 & 83.46 $\pm$ 0.49 & 70.88 $\pm$ 0.47 & 68.77 $\pm$ 1.42 & 71.82 $\pm$ 0.53 & 71.09 $\pm$ 0.69 \\
        & \cite{DBLP:conf/nips/ZhouLLY19} (averaging)& - & - & - & - & 68.09 $\pm$ 0.84 & 67.48 $\pm$ 0.52 & 58.13 $\pm$ 2.39 & 53.84 $\pm$ 5.00 \\
        & \cite{DBLP:conf/cvpr/RamanujanWKFR20} ($k=50\%$) & 82.38 $\pm$ 0.29 & 83.61 $\pm$ 0.38 & 81.71 $\pm$ 0.59 & 83.55 $\pm$ 0.32 & - & - & - & - \\
      \midrule

      % Conv 6 results
       \multirow{4}{*}{\textbf{Conv6}} & ASLP (thresholding) & \textbf{84.98 $\pm$ 0.33} & \textbf{86.49 $\pm$ 0.36} & \textbf{85.32 $\pm$ 0.27} & \textbf{86.21 $\pm$ 0.34} & \textbf{73.32 $\pm$ 0.42} & \textbf{69.83 $\pm$ 1.46} & \textbf{76.20 $\pm$ 0.91} & \textbf{75.30 $\pm$ 0.89} \\
        & ASLP (averaging)& 84.24 $\pm$ 0.28 & 85.67 $\pm$ 0.34 & 84.51 $\pm$ 0.35 & 85.49 $\pm$ 0.38 & 72.62 $\pm$ 0.57 & 69.53 $\pm$ 1.68 & 75.24 $\pm$ 0.69 & 74.50 $\pm$ 0.96 \\
        & \cite{DBLP:conf/nips/ZhouLLY19} (averaging) & - & - & - & - & 70.71 $\pm$ 0.98 & 69.16 $\pm$ 1.92 & 44.77 $\pm$ 17.02 & 36.59 $\pm$ 15.32 \\
        & \cite{DBLP:conf/cvpr/RamanujanWKFR20} ($k=50\%$) & 84.67 $\pm$ 0.35 & 85.87 $\pm$ 0.13 & 84.37 $\pm$ 0.58 & 85.84 $\pm$ 0.51 & - & - & - & - \\
      \bottomrule
    \end{tabular}
  }
  \caption{Comparison of ASLP's performance with Edge-Popup and Supermask
  \cite{DBLP:conf/cvpr/RamanujanWKFR20,DBLP:conf/nips/ZhouLLY19} on CIFAR10
  using various configurations. We use the configurations tested by the author
  of the aforementioned articles. Performance measures reported are accuracy and
  are presented with and without data augmentation, weight rescaling (WR), and
  the signed constant weight distribution (SC). We report performances for both
  "thresholding" and "averaging" setups for our method. For edge-popup, we use
  the best $k$ value for Conv\{2,4,6\} repported in
  \cite{DBLP:conf/cvpr/RamanujanWKFR20}.  Across all setups, our method ASLP
  outperforms Edge-Popup and Supermask}
  \label{tab:chap2:con_performances_comparison_cifar10}
  
\end{table}



\begin{table}
  \centering
  \resizebox{16.5cm}{!}{
    \begin{tabular}{lllllllllll}
      \cmidrule[\heavyrulewidth]{3-10}
      &  & \multicolumn{4}{c}{\textbf{w/ data augmentation}} & \multicolumn{4}{c}{\textbf{w/o augmentation}} \\
      &  &  $\varnothing$ & \textbf{SC} & \textbf{WR} & \textbf{WR+SC} & $\varnothing$ & \textbf{SC} & \textbf{WR} & \textbf{WR+SC} \\
      \toprule
      \multirow{12}{*}{}
      \multirow{4}{*}{Conv2} & ASLP (thresholding) & \textbf{38.64 $\pm$ 0.92} & 38.31 $\pm$ 0.75 & \textbf{41.81 $\pm$ 0.84} & \textbf{42.06 $\pm$ 0.76} & \textbf{38.72 $\pm$ 0.59} & 38.64 $\pm$ 1.23 & \textbf{42.42 $\pm$ 0.30} & \textbf{41.95 $\pm$ 0.68} \\
      & ASLP (averaging) & 38.49 $\pm$ 0.61 & 38.18 $\pm$ 0.81 & 41.12 $\pm$ 0.66 & 41.17 $\pm$ 0.54 & 38.40 $\pm$ 0.81 & \textbf{38.71 $\pm$ 1.05} & 41.66 $\pm$ 0.39 & 41.42 $\pm$ 0.55 \\
      & \cite{DBLP:conf/nips/ZhouLLY19} (averaging) & - & - & - & - & 38.09 $\pm$ 1.03 & 37.28 $\pm$ 0.47 & 26.03 $\pm$ 2.23 & 23.49 $\pm$ 1.36 \\
      & \cite{DBLP:conf/cvpr/RamanujanWKFR20} ($k=50\%$) & 38.47 $\pm$ 0.46 & \textbf{39.83 $\pm$ 0.46} & 38.57 $\pm$ 0.59 & 39.87 $\pm$ 0.78 & - & - & - & - \\
      \midrule

      \multirow{4}{*}{Conv4} & ASLP (thresholding) & \textbf{47.78 $\pm$ 1.18} & 49.33 $\pm$ 0.77 & \textbf{50.33 $\pm$ 0.39} & \textbf{51.49 $\pm$ 0.43} & \textbf{47.56 $\pm$ 0.36} & \textbf{49.30 $\pm$ 0.54} & \textbf{50.39 $\pm$ 0.58} & \textbf{51.16 $\pm$ 0.94} \\
      & ASLP (averaging) & 47.18 $\pm$ 1.17 & 48.78 $\pm$ 0.79 & 49.39 $\pm$ 0.30 & 50.17 $\pm$ 0.50 & 46.89 $\pm$ 0.52 & 48.74 $\pm$ 0.47 & 49.55 $\pm$ 0.57 & 50.23 $\pm$ 0.87 \\
      & \cite{DBLP:conf/nips/ZhouLLY19} (averaging) & - & - & - & - & 45.84 $\pm$ 1.01 & 47.72 $\pm$ 0.75 & 27.70 $\pm$ 2.41 & 27.53 $\pm$ 5.20 \\
      & \cite{DBLP:conf/cvpr/RamanujanWKFR20} ($k=50\%$) & 47.75 $\pm$ 0.63 & \textbf{50.16 $\pm$ 0.47} & 48.20 $\pm$ 0.72 & 50.02 $\pm$ 0.65 & - & - & - & - \\
      \midrule

      \multirow{4}{*}{Conv6} & ASLP (thresholding) & 51.09 $\pm$ 0.92 & 53.00 $\pm$ 0.52 & \textbf{51.70 $\pm$ 0.48} & 52.85 $\pm$ 0.50 & \textbf{51.43 $\pm$ 0.41} & \textbf{53.10 $\pm$ 0.27} & \textbf{51.52 $\pm$ 0.35} & \textbf{53.22 $\pm$ 0.54} \\
      & ASLP (averaging) & 50.22 $\pm$ 1.09 & 51.72 $\pm$ 0.73 & 50.56 $\pm$ 0.33 & 51.59 $\pm$ 0.24 & 50.47 $\pm$ 0.42 & 52.00 $\pm$ 0.27 & 50.38 $\pm$ 0.33 & 51.82 $\pm$ 0.34 \\
      & \cite{DBLP:conf/nips/ZhouLLY19} (averaging) & - & - & - & - & 49.19 $\pm$ 0.75 & 50.66 $\pm$ 0.47 & 2.54 $\pm$ 1.63 & 9.21 $\pm$ 5.50 \\
      & \cite{DBLP:conf/cvpr/RamanujanWKFR20} ($k=50\%$) & \textbf{51.13 $\pm$ 0.39} & \textbf{53.48 $\pm$ 0.51} & 51.06 $\pm$ 1.11 & \textbf{54.01 $\pm$ 0.35} & - & - & - & - \\
      \bottomrule
    \end{tabular}
  }
  \caption{Comparison of ASLP's performance with Edge-Popup and Supermask
  \cite{DBLP:conf/cvpr/RamanujanWKFR20,DBLP:conf/nips/ZhouLLY19} on CIFAR100
  using various configurations. We use the configurations tested by the author
  of the aforementioned articles. Performance measures reported are accuracy and
  are presented with and without data augmentation, weight rescaling (WR), and
  the signed constant weight distribution (SC). Pre-pruning results are reported
  for ASLP and Supermask. Since pruning is baked in the Edge-Popup method, we do
  not report a specific pre-pruning performance for it. We report performances for both
  "thresholding" and "averaging" setups for our method. For edge-popup, we use
  the best $k$ value for Conv\{2,4,6\} repported in
  \cite{DBLP:conf/cvpr/RamanujanWKFR20}. For smaller networks, ASLP outperforms
  the other methods, with the exception of the SC setup for Conv2 and Conv4.
  However, for Conv6, ASLP's performance is superior when data augmentation is
  disabled, while edge-popup achieves better results with data augmentation
  enabled (except for the WR setup).}
  \label{tab:chap2:con_performances_comparison_cifar100}
\end{table}
% endregion: perf-tables

\subsection{Impact of initialisation}

\subsection{Impact of smart rescaling}

% endregion: experiments


\section{Conclusion}
% region : conclusion
%endregion : conclusion

  % ------------------------------------------------------------------------------
  \subsection{Weight Rescaling}
  \label{sec:smart-rescale}
Subnetwork selection may disrupt the dynamic of the forward pass
\cite{DBLP:conf/iccv/HeZRS15,DBLP:conf/cvpr/RamanujanWKFR20}, and thereby
requires adapting  weights accordingly.    Dynamic weight rescale (DWR)
\cite{DBLP:conf/nips/ZhouLLY19}, and scaled Kaiming distribution
\cite{DBLP:conf/cvpr/RamanujanWKFR20}  are two  known mechanisms that adapt the
weights of the selected subnetworks.  However, some of these heuristics, besides
being handcrafted,   rely on the strong assumption that rescaling should be
proportional to the pruning rate.  In what follows,  we consider a new weight
adaptation mechanism, referred to as Smart Rescale (SR). Instead of handcrafting
this rescaling factor proportionally to the pruning rate (as achieved for
instance in \cite{DBLP:conf/nips/ZhouLLY19}), SR is learned layerwise and
provides an effective (and also efficient)  way to adapt the dynamic of the
forward pass without retraining the entire weights of the selected subnetwork.
Indeed, this rescaling ends up reducing the amount of epochs needed to reach
convergence and also improving accuracy (at some extent) as shown later in
experiments. \\ With SR, the $\ell$-th layer network output becomes 
  \begin{equation}
   \mathbf{z}_{\ell} = g_\ell(s_\ell \times (\bm{m}_\ell \odot \bm{w}_\ell) \otimes \mathbf{z}_{\ell-1}),
  \end{equation}
  \noindent where  $s_\ell$ refers to the rescaling factor  of  the $\ell$-th
  layer (see also algorithm~\ref{alg:gumbel-forward}).  Smart Rescale increases
  the flexibility of subnetwork selection and adaptation compared to DWR (which
  is bound to the pruning rate).  Moreover,  scaling factors obtained with SR
  vary smoothly  ---  and this makes training more stable with stochastic
  gradient descent (SGD) --- compared to the ones obtained with DWR which are
  again set to the observed {\it pruning rates},  and changes of the latter are
  more abrupt due to stochastic mask sampling.    
  \begin{algorithm}
    \caption{Forward pass for our method}
    \label{alg:gumbel-forward}
    \begin{algorithmic}[1]
    \REQUIRE A network $f_\theta$, with weights $\{\bm{w}_\ell\}_\ell$, ASLP  $\{\bm{\hat{m}}_\ell\}_\ell$, and input training data $\{({\bf x}_k,{\bf y}_k)\}_k$ 
    \STATE
    $q_{i,j} \gets \text{argmax} 
    \begin{bmatrix}
      \bm{\hat{m}}_{i,j} + g_{i,j} \\
      0 + g'_{i,j}\\
    \end{bmatrix}$ \COMMENT{Sampling of a topology} 
    \STATE $m_{ij} \gets 1_{\{q_{ij}=1\}}$
   \COMMENT{
    Giving the masks $\bm{m}_{i,j}$ their values} 
    \STATE {\bf Return} ${\cal L}\big(f_\theta(\{{\bf x}_k\}_k; \{s_\ell (\bm{m}_\ell \odot \bm{w}_\ell)\}_\ell),\{{\bf y}_k\}_k\big)$
    \COMMENT{Computing the loss with masked weights and SR}
    \end{algorithmic}
  \end{algorithm} 
  % ==============================================================================
  \section{Experiments}\label{sec:experiments}
  
  In this section, we show the performance of our method on the standard CIFAR10
  and CIFAR100 datasets.   They consist of  60k  colored images of $32\times 32$
  pixels each.  Training,  validation and test sets  include  45k,  5k and 10k
  images respectively.  \\  In order to demonstrate the effectiveness of our
  method, we chose the widely used SGD optimizer with a momentum of  0.9  and a
  learning rate of 50.    Faster convergence is obtained with higher learning
  rates,   however,  the latter also lead  to worse observed accuracy.   During
  training,  the maximum number of epochs is set to 1000 and early stopping is
  triggered  if the accuracy on the validation set stops improving during 100
  epochs.   In all these experiments, neither weight decay nor $\ell_2$
  regularisation are applied. See implementation details and our code on the ASLP GitHub
  \cite{Dupont2022}.

\begin{table*}[htbp]
  \centering
  \resizebox{14.0cm}{!} 
{
  \begin{tabular}{@{}llcccccccc|c@{}}
    \cline{3-11}
                       &                                                        & \multicolumn{8}{c|}{Cifar 10}                                                                                                 & Cifar 100 \\ 
                       \cline{3-11}
                            &                                                       & \multicolumn{4}{c}{w/o data augmentation}                                     & \multicolumn{4}{c|}{with data augmentation (w.d.a)}         &  w.d.a  \\
                            &                                                        & $\varnothing$ & WR             & SC            & WR+SC          & $\varnothing$ & WR             & SC            & WR+SC          &           WR+SC   \\ \midrule
    \multirow{4}{*}{Conv2} & \cite{DBLP:conf/nips/ZhouLLY19} (averaging)         &64.4          & 65.0          & 66.3          & 66.0          & -             & -             & -             & -            &               -        \\
                            & \cite{DBLP:conf/cvpr/RamanujanWKFR20}\footnotemark ($k=50\%$)  &               &   -           &         -     &         -     &       -       &      -        & 71.5   &       71.7       &     40.9         \\ 
                            & Our ASLP (averaging)                                     & 68.2          & 66.9          & 68.3          & 66.5          & \textbf{76.0} & \textbf{76.6} & 76.8          & 77.3          &       -         \\
                            & Our ASLP (thresholding)                                     & \textbf{68.7} & \textbf{67.8} & \textbf{68.4} & \textbf{67.1} & 75.9          & 76.4          & \textbf{77.5} & \textbf{77.5} &    \textbf{43.3}          \\
                            \midrule
  
    \multirow{4}{*}{Conv4} & \cite{DBLP:conf/nips/ZhouLLY19} (averaging)       & 65.4          & 71.1          & 66.2          & 72.5          & -             & -             & -             & -             &             -           \\
                            & \cite{DBLP:conf/cvpr/RamanujanWKFR20}\footnotemark[\value{footnote}] ($k=50\%$) &        -      &  -            &            -  &  -            &  -            &  -            & 81.6  &     80.5    &  51.1  \\ 
                            & Our ASLP (averaging)                                          & 70.6          & 71.8          & 69.5          & 71.8          & 83.4          & 84.4          & 83.7          & 84.1          &        -   \\
                            & Our ASLP (thresholding)                                      & \textbf{71.5}  & \textbf{72.8}& \textbf{70.2} & \textbf{72.7} & \textbf{83.7}& \textbf{85.0} & \textbf{84.5} & \textbf{84.8} &         \textbf{51.7}  \\
                            \midrule
    \multirow{4}{*}{Conv6} & \cite{DBLP:conf/nips/ZhouLLY19} (averaging)    & 63.5          & 76.3          & 65.4          & 76.5          & -             & -             & -             & -      &  -       \\
                            & \cite{DBLP:conf/cvpr/RamanujanWKFR20}\footnotemark[\value{footnote}] ($k=50\%$) &      -        &    -          &        -      &     -         &      -        &           -   &  85.4 &    85.1   &   \textbf{53.8}   \\ 
                            & Our ASLP (averaging)                                      & 72.9          & 76.1          & 71.9          & 75.6          & 85.3          & 86.2          & 85.3          & 86.2          &  - \\
                            & Our ASLP (thresholding)                                       & \textbf{73.7} &\textbf{77.0}  &\textbf{72.6}  &\textbf{76.6}  &\textbf{86.0}  &\textbf{86.9}  & \textbf{86.3} &\textbf{86.9}  &   52.8 \\
                            \bottomrule
    \end{tabular}
}

  \caption{\footnotesize Comparison of our method against \cite{DBLP:conf/nips/ZhouLLY19}
  and \cite{DBLP:conf/cvpr/RamanujanWKFR20} on Conv2, Conv4 and Conv6. These
  results are averaged through five independent runs. "WR" (Weight Rescale)
  refers to ``Dynamic Weight Rescale'' or ``Smart Rescale'' depending on which
  methods is used (respectively \cite{DBLP:conf/nips/ZhouLLY19} or our proposed
  ASLP). Again, "SC" refers to the ``Signed Constant'' distribution. The latest results on CIFAR 100
  were recently obtained with data augmentation and WR+SC.}
  \label{tbl:conv_compare}

\end{table*}

    

% ------------------------------------------------------------------------------
\subsection{Performance and comparison}
The accuracy of our method is evaluated on subnetworks whose topology
corresponds to connections with  (trained) probabilities larger than 0.5;  in
other words,  if {\it the binary event of keeping a connection is more likely
than its removal}.  This setting is referred to as  {\it thresholding}.  As a
matter of comparison,  we also consider the setting in
\cite{DBLP:conf/nips/ZhouLLY19} which consists in sampling ten different
subnetworks and evaluating an average accuracy over  these subnetworks.  This
setting  is referred to as {\it averaging}.  In these experiments,  we use the
same networks as
\cite{DBLP:conf/nips/ZhouLLY19,DBLP:conf/cvpr/RamanujanWKFR20} (originally
introduced by  Frankle and Carbin \cite{DBLP:conf/iclr/FrankleC19}) namely
Conv2,  Conv4 and Conv6 which are  variants of  VGG16. \\
\indent   \cref{tbl:conv_compare}  shows a comparison of our method against
\cite{DBLP:conf/nips/ZhouLLY19,DBLP:conf/cvpr/RamanujanWKFR20}.    These
results show means of five independent runs; each run corresponds either to
``thresholding'' or ``averaging''.  These performances show a consistent gain
(in accuracy) of our subnetwork selection.   We also observe that
``thresholding'' is already effective compared to ``averaging''; indeed, our
method reaches a  high accuracy despite learning a single subnetwork topology,
and this makes it also highly efficient for training compared to  the related
work \cite{DBLP:conf/nips/ZhouLLY19,DBLP:conf/cvpr/RamanujanWKFR20}.\\ 
\indent Furthermore, our method and \cite{DBLP:conf/nips/ZhouLLY19} do not
impose a pruning rate. The optimal pruning rate is found during optimisation and
is is arround 51\%, whereas \cite{DBLP:conf/cvpr/RamanujanWKFR20} enforces a 50\%
pruning rate ($k=50\%$). Thus, the networks capacities
are comparable. \\


  % ------------------------------------------------------------------------------
  \subsection{Ablation study}
  In this section, we discuss the impact of all the components of the method
  when taken individually and combined,  namely the use of weight rescaling
  (WR): either DWR or our proposed SR.   We also consider another criterion:
  signed constant  (SC) which consists in replacing weights in a given layer by
  the products of their signs and the standard deviation of their original
  weight distribution.  We show  all these results with and without data
  augmentation, which is composed of the combination of zero-padding,  random crops and
  random  horizontal flips.   Note that  pixel intensities are normalized  from
  their original values in $[0,255]$ to $[0,1]$.
    
  From the results in table~\ref{tbl:conv_compare}, we observe a clear gain of
  our method alone w.r.t.  \cite{DBLP:conf/nips/ZhouLLY19} and the use of SR
  increases further its accuracy (excepting Conv2 w/o data augmentation).  The
  gain in performances increases  significantly with Conv6 and reaches up to 4
  points even when no data augmentation is used.   Note that the use of data
  augmentation attenuates, at some extent, the effect of SR on larger networks
  (conv4 and Conv6). Nonetheless,  as discussed in \cref{sec:SR-impact}, the
  positive impact of SR resides also in training efficiency.   In contrast to
  SR, signed constant improves accuracy by a small margin when combined with
  data augmentation. 
  
  \footnotetext{Performances for \cite{DBLP:conf/cvpr/RamanujanWKFR20} are reported
  with the optimizer described in \cref{sec:experiments}. It is possible to
  improve performances by tuning the learning rate scheduler but this is out of
  the scope of this paper.}
  
  % ------------------------------------------------------------------------------
  \subsection{Computational efficiency}
  \label{sec:SR-impact}
DWR requires rectifying weights layerwise using the inverse of the observed
(computed) pruning rates. These layerwise evaluations introduce a significant
overhead at each training epoch. In contrast, SR consists in simple products
involving one scalar per layer. When training Conv4, we found (on average) that
enabling DWR increases epoch runtime by $0.2s$ while our SR by $0.13s$ only, so
SR speeds up training overhead by 35\% compared to DWR.  When data augmentation
and signed constant are used, SR allows a significant gain in the number of
training epochs. Indeed, enabling SR on Conv4  saves (on average) 19.7\%
training epochs (8.2\%, 14.0\%  on Conv2 and Conv6 respectively) before
converging to its highest accuracy. Finally, our ``thresholding'' setting not
only improves accuracy but makes subnetwork selection (training) and also
inference more efficient compared to the related work
\cite{DBLP:conf/nips/ZhouLLY19,DBLP:conf/cvpr/RamanujanWKFR20},  as this
selection is again achieved  once and thereby only one subnetwork is applied
during inference.
 

% ==============================================================================
\section{Conclusion}

In this paper, we introduce a novel method that extracts effective subnetworks
from larger networks without training its weights. The proposed method optimizes
a probability distribution which measures the relevance of weights, and only
those with the highest relevance define the topology of the selected
subnetworks. An efficient and effective weight rescaling mechanism is also
introduced and allows rectifying the parameters of the selected subnetworks
which improves performances and reduces the number epochs needed to reach
convergence. Experiments conducted on the standard CIFAR10 and  CIFAR100
datasets show the effectiveness of our subnetwork selection method w.r.t. the
related work. Future work includes the study of the scalability of the proposed
method on more complex datasets and other larger networks.\\

\noindent\textbf{Acknowledgement.} 
This work was performed using HPC resources from GENCI-IDRIS (Grant 2021-AD011011427R1). \\
It has been achieved within a partnership between Sorbonne University and Netatmo.

\noindent\textbf{Code.} Our code is available at:\\ \href{https://github.com/N0ciple/ASLP}{\texttt{https://github.com/N0ciple/ASLP}}